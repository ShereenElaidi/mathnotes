\documentclass[11pt]{scrartcl}
\usepackage[margin=2cm]{geometry}
\usepackage{amsmath}
\usepackage{amsfonts}
\usepackage{amssymb,amsmath,amsthm}
\usepackage{xcolor} 
\usepackage{enumitem}
\newcommand{\R}[0]{\mathbb{R}}
\addtokomafont{section}{\rmfamily\centering\scshape}
% math environments 
\usepackage[utf8]{inputenc}
\theoremstyle{definition}
\newtheorem{theorem}{Theorem}
\newtheorem{corollary}{Corollary}
\newtheorem{lemma}[theorem]{Lemma}
\newtheorem{definition}{Definition}
\newtheorem{prop}{Proposition}
\newtheorem{ex}{Example}
\theoremstyle{remark}
\newtheorem*{remark}{Remark}

% definition
\newcommand{\dfn}[1]{\textbf{\underline{#1}}}
\newcommand{\dist}[0]{\mathcal{F}}
\newcommand{\pr}[1]{\mathbb{P}[#1]} 
\newcommand{\stat}[0]{T(X_1, ..., X_n )} 

% converge in probability 
\newcommand{\cvp}[0]{\overset{p}{\to}}

% sample mean
\newcommand{\smean}[0]{\frac{1}{n} \sum_{i=1}^n x_i} 

% sample variance
\newcommand{\svar}[0]{\frac{1}{(n-1)} \sum_{i=1}^n (x_i - \overline{x})^2}

% expected value 
\newcommand{\EX}[1]{\mathbb{E}\left[#1 \right]}  
\newcommand{\EXth}[1]{\mathbb{E}_\theta \left[ #1 \right]}

% integral
\newcommand{\idx}[2]{\int_{#1}^{#2}}

% vector
\newcommand{\vect}[1]{\mathbf{#1}}


\title{\textbf{Math 475: Partial Differential Equations}}
\author{Shereen Elaidi}
\date{Fall 2019 Term}

\begin{document}

\maketitle
\tableofcontents

\section{Introduction}

\textbf{Q:} Where do PDE's come from? PDEs are used in all types of math modelling; they translate phenomena coming from physics, biology, chemistry, etc. into mathematical terms. 
\newline 
\newline 
\textbf{Q:} What goes into a PDE model? 
\begin{center}
	Math model = General Physical Laws (balancing forces and conservation of quantities) + Constitutive Relations (Laws specific to the environment, e.g. Fick's Law of Diffusion) 
\end{center}
A PDE $\leftrightarrow$ a physical description of a system. 

\begin{definition}[PDE]
	A \dfn{PDE} is a mathematical relation involving partial derivatives, i.e., if $\vect{u}: \R^n \rightarrow \R$, $\vect{x} = (x_1, ..., x_n)$ a real-valued function of several variables, then any kth order PDE can be expressed as: 
	\begin{align}
		F(D^k \vect{u}, D^{k-1} \vect{u}, ..., D \vect{u}, \vect{u}, \vect{x} ) = 0 \text{ in } \Omega 	
	\end{align}
	where $\Omega \subseteq \R^n$ is the domain or region where the PDE holds, and $F$ is a function of the placeholders. 
\end{definition}

\begin{definition}[$C^k$ and $C^k(\Omega)$] A function $\vect{u}: \R^n \rightarrow \R$ is $C^k$ at a point $\vect{x} \in \R^n$ if every kth-order partial derivative is continuous as a function of $\R^n$ at $x$. If $u \in C^2(\R^n)$, then in advanced calculus it was proven that $D^2u$ is a symmetric matrix. 
\end{definition}

\begin{ex}[Examples of PDEs] 
	\begin{enumerate}[noitemsep]
		\item \dfn{Heat Equation}: 
		\begin{align}
				u_t - k \Delta u = u_t - k \sum_{i=1}^n u_{x_i x_i} = 0 
		\end{align}
		where $k > 0$ is a constant. Then the solution has a space- and time- component: 
		\begin{align}
			u(\vect{x}, t) = u: \R^{n+1} \rightarrow \R 
		\end{align}
	\item \dfn{Laplace's Equation}: 
	\begin{align}
			- \Delta u = 0 
	\end{align}
	$u(x)$ is the steady state of a solution to the heat equation, since $u_t = 0 \Rightarrow$ ``constant in time.'' The solution $u$ is a function $u: \R^n \rightarrow \R$. 
	\item \dfn{Wave Equation}: 
	\begin{align*}
		& u_{tt} - \Delta u = 0 \\
		& u: \R^{n+1} \rightarrow \R 
	\end{align*}
	$u(x,t)$ is the displacement of an object with wave-like behaviour at location $\vect{x}$ and time $t$. Ex: position of a guitar string. 
	\item \dfn{Transport Equation}: 
	\begin{align*} 
		u_t + cu_x = 0,\ c \in \R, u: \R^2 \rightarrow \R \text{ (space, time) } 
	\end{align*} 
	$u(x,t)$ can be the density of a pollutant at location $\vect{x}$ and time $t$. 
	\item  \dfn{Reaction-Diffusion}: 
	\begin{align*}
		& u_t - k \Delta u = f(u) \\
		& u: \R^{n+1} \rightarrow \R \\
		& f: \R \rightarrow \R 	
	\end{align*}
	$u( \vect{x},t)$ can be the temperature at location $(\vect{x}, t)$ subject to enhancement by $f(u)$ (for example, fire spreading). 
	\item \dfn{Burger's Equation}: 
	\begin{align}
		u_t - u u_x = vu_{xx} 
	\end{align}
	$v > 0$ represents the viscosity. $u: \R^2 \rightarrow \R$, $u(x,t)$ is the concentration of a material in a fluid flow with convection. 
	\end{enumerate}	
\end{ex}

\subsection{Domains and Boundary Conditions}

\begin{definition}[$C^1$ domain] 
	$\Omega \subseteq \R^n$ is a \dfn{$C^1$ domain} if $\partial \Omega$ can locally be expressed as a graph of a $C^1$ function. This means...
	\begin{enumerate}[noitemsep]
		\item $\partial \Omega$ has no corners $\Rightarrow$ smooth. 
		\item $\forall p \in \partial \Omega$, there exists a well-defined and unique tangent plane (whose slope is given by the derivative), which implies that there exists a well-defined inward and outward normal vector. 
		\item Inward and outward normal vectors move continuously along $\partial \Omega$. 
	\end{enumerate}
\end{definition}

\subsubsection{What are the main boundary conditions?}
\begin{enumerate}[noitemsep]
	\item \dfn{Dirichlet Boundary Conditions}: these prescribe what $u$ is on $\partial \Omega$. We assume that $u \in C^k(\Omega) \cap C(\partial \Omega)$. 
	\item \dfn{Neumann Boundary Conditions}: these prescribe what the normal derivative of $u$ on $\partial \Omega$ is. The meaning behind this is: \emph{how does $u$ change along the boundary}? It is specifying:
	\begin{align}
		\frac{\partial u}{\partial n} = \nabla u \cdot n(x) = X \text{ on } \partial \Omega 	
	\end{align}
	\item \dfn{Robin boundary conditions}: a combination of the above: 
	\begin{align}
		\frac{\partial u}{\partial n} + \alpha u = X \text{ on } \partial \Omega 
	\end{align}
\end{enumerate}
In general, in PDEs, we will not be able to identify an explicit solution. Thus, we care about the following four fundamental issues. The first three ensure that we have a \emph{well-posed problem}, and the final one is important when we cannot obtain an explicit solution. 
\begin{enumerate}[noitemsep]
	\item \emph{Existence}: is there a solution? 
	\item \emph{Uniqueness}: is there exactly one solution to the PDE?
	\item \emph{Stability}: does the solution depend continuously on the data? 
	\item \emph{Qualitative Properties}: If I cannot find an explicit solution, but I know that it exists, what else can I tell you about the solution? Some questions we are interested in studying are: 
	\begin{enumerate}[noitemsep]
		\item Does $u(\vect{x}, t) \rightarrow 0$ as $t \rightarrow \infty$? 
		\item What is $\max_\Omega |u(\vect{x})|$? 
	\end{enumerate}
\end{enumerate}

\subsection{Classification of PDEs} 
To do: draw out the chart 

\end{document}