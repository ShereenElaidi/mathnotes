\documentclass[11pt]{scrartcl}
\usepackage[margin=2cm]{geometry}
\usepackage{amsmath}
\usepackage{fancyhdr}
\usepackage{amsfonts}
\usepackage{amssymb,amsmath,amsthm}
\usepackage{xcolor} 
\usepackage{enumitem}
\newcommand{\R}[0]{\mathbb{R}}
\addtokomafont{section}{\rmfamily\centering\scshape}
% math environments 
\usepackage[utf8]{inputenc}
\theoremstyle{definition}
\newtheorem{theorem}{Theorem}
\newtheorem{corollary}{Corollary}
\newtheorem{lemma}[theorem]{Lemma}
\newtheorem{definition}{Definition}
\newtheorem{prop}{Proposition}
\newtheorem{ex}{Example}
\theoremstyle{remark}
\newtheorem*{remark}{Remark}
\usepackage{hyperref}
\hypersetup{
    colorlinks,
    citecolor=black,
    filecolor=black,
    linkcolor=black,
    urlcolor=black
}

% definition
\newcommand{\dfn}[1]{\textbf{\underline{#1}}}
\newcommand{\dist}[0]{\mathcal{F}}
\newcommand{\pr}[1]{\mathbb{P}[#1]} 
\newcommand{\stat}[0]{T(X_1, ..., X_n )} 

% converge in probability 
\newcommand{\cvp}[0]{\overset{p}{\to}}

% sample mean
\newcommand{\smean}[0]{\frac{1}{n} \sum_{i=1}^n x_i} 

% sample variance
\newcommand{\svar}[0]{\frac{1}{(n-1)} \sum_{i=1}^n (x_i - \overline{x})^2}

% expected value 
\newcommand{\EX}[1]{\mathbb{E}\left[#1 \right]}  
\newcommand{\EXth}[1]{\mathbb{E}_\theta \left[ #1 \right]}

% integral
\newcommand{\idx}[2]{\int_{#1}^{#2}}

% vector
\newcommand{\vect}[1]{\mathbf{#1}}


\title{\textbf{Math 458: Differential Geometry}}
\author{Shereen Elaidi}
\date{Winter 2020 Term}

\begin{document}

\maketitle
\tableofcontents
\pagestyle{fancy}
\lhead{Math 458: Differential Geometry}
\chead{Winter 2020 -- Summary}
\rhead{Page \thepage}
\lfoot{}
\cfoot{}
\rfoot{}
\renewcommand{\headrulewidth}{0.4pt}
\renewcommand{\footrulewidth}{0.4pt}
\setlength{\tabcolsep}{0.5em} % for the horizontal padding
{\renewcommand{\arraystretch}{1.2}% for the vertical padding

\section{Introduction}
\subsection{Implicit and Inverse Function Theorems}

\section{Manifolds in $\R^3$}
The aim of this part of the course is to build up to integration on manifolds and the invariant Stokes' theorem. The main purpose of this sections is to develop \emph{coordinate-free} calculus, which clarifies the essence of what is happening (sometimes coordinates can be noisy). 
\subsection{Definitions}

\subsection{Smooth Maps from $M^m \rightarrow N^n$}

\subsection{Change of Coordinates}

\subsection{Multi-Linear Algebra}

\subsection{Differential Forms in $M^n$}

\subsection{Change of Variables for Integrals in $\R^n$}

\subsection{Integrating a $n$-Form on $M^n$ ($\idx{M}{} \omega$)} 

\section{Curves}
There are two subsets of differential geometry: classical differential geometry and global differential geometry. The objective of \dfn{classical differential geometry} is to study the local properties of curves and surfaces. The objective of \dfn{global differential geometry} is to study the influence of local properties on global behaviour. 
\subsection{Definitions}

\begin{definition}[Parameterised Differentiable Curve] 
	A \dfn{parameterised differentiable curve} is a differentiable map $ \alpha: I \rightarrow \R^3$ of an open interval $I = ]a,b[$ of the real line $\R$ into $\R^3$. The image of $\alpha$ is called the \dfn{trace} of $\alpha$. 
\end{definition}

Some examples of parameterised curves include: 
\begin{itemize}[noitemsep]
	\item The helix: $\alpha(t) = (a \cos (t), a \sin(t), bt)$ for $t \in \R$. 
	\item The map $\alpha: \R \rightarrow \R^2$, $t \in \R$, is a parameterised differentiable curve. 
\end{itemize}

\begin{definition}[Norm on $\R^3$]
	Let $u = (u_1, u_2, u_3) \in \R^3$. The \dfn{norm} of $u$ is: 
	\begin{align*}
		|| u || := \sqrt{ u_1^2 + u_2^2 + u_3^3} 
	\end{align*}
\end{definition}

\begin{definition}[Inner Product on $\R^3$] 
	Let $u = (u_1, u_2, u_3)$ and $v= (v_1, v_2, v_3)$ belong to $\R^3$ and let $\theta \in [0, \pi]$ be the angle formed between $u,v$. The \dfn{inner product} is defined by: 
	\begin{align}
		u \cdot v := || u || ||v|| \cos (\theta) 	
	\end{align}
\end{definition}
It satisfies the following properties: 
\begin{enumerate}[noitemsep]
	\item If $u$, $v$ are non-zero, then $u \cdot v = 0$ $\iff$ $u \perp v$. 
	\item $u \cdot v = v \cdot u$. 
	\item $\lambda (u \cdot v ) = \lambda u \cdot v =  u \cdot \lambda v$. 
	\item $u ( v + w) = u \cdot v + u \cdot w$. 
\end{enumerate}
If we have made a choice of basis, then we can formulate the dot product in terms of the components of the vectors as: 
\begin{align}
	u \cdot v = u_1 v_1 + u_2 v_2 + u_3 v_3 	
\end{align}

\subsubsection{Regular Curves and Arclength}
In differential geometry, it is \underline{essential} that our curves have a tangent line at every point. This motivates the following definition. 

\begin{definition}[Regular Curve]
	A parameterised differentiable curve $\alpha: I \rightarrow \R^3$ is \dfn{regular} if $\alpha' (t) \neq 0$ $\forall t \in I$. 
\end{definition}

\begin{definition}[Arc length]
	Given $t_0 \in I$, the \dfn{arc length} of a regular parameterised curve $\alpha: I \rightarrow \R^3$ from $t_0$ to $t$ is defined to be: 
	\begin{align*}
		s(t) := \idx{t_0}{t} | a'(t) | dt 
	\end{align*}
	where
	\begin{align*}
		| \alpha' (t) | := \sqrt{(x'(t))^2 + (y'(t))^2 + (z'(t))^2}
	\end{align*}
	Since we only restrict our attention to regular surfaces, $a'(t) \neq 0$ for all $t$, and so the arlength function is a differentiable function of $t$ and $ds/dt = |a'(t)|$ (by the Fundamental Theorem of Calculus). Arc length parameterisations make life simpler. 
\end{definition}

\subsubsection{The Vector Product in $\R^3$}

\begin{definition}[Vector Product]
	Let $u,v \in \R^3$. Then, the \dfn{vector product} of $u,v$ is the unique vector $u \wedge v$ in $\R^3$ characterised by: 
	\begin{align*}
		(u \wedge v) \cdot w = \text{det}(u,v,w)\ \text{  } \forall w \in \R^3
	\end{align*}
	this is more commonly known as:
	\begin{align*}
		u \wedge v = \text{det} \begin{bmatrix}
			\hat{i} & \hat{j} & \hat{k} \\
			u_1 & u_2 & u_3 \\
			v_1 & v_2 & v_3 
		\end{bmatrix}
	\end{align*}
	where $\hat{i}, \hat{j}, \hat{k}$ are the standard basis vectors in $\R^3$. 
\end{definition}

\underline{Properties of the Vector Product} 
\begin{enumerate}[noitemsep]
	\item (Anti-Commutativity): $u \wedge v = -v \wedge u$. 
	\item (Linear Dependence): $\forall$ $\alpha$, $\beta$ $\in \R$: 
	\begin{align*}
		(\alpha u + \beta v) \wedge v = \alpha u \wedge v + \beta w \wedge v
	\end{align*}
	\item $u \wedge v = 0$ $\iff$ $u$ and $v$ are linearly dependent. 
	\item $(u \wedge v) \cdot u =0$, $(u \wedge v) \cdot v =0$ (this implies that the vector product is normal to the plane generated by $u$ and $v$). 
\end{enumerate}

\subsection{Frenet-Serret Frame} % TO DO: Make this section more comprehensible, just a list of definitions for now. 

\begin{definition}[Curvature]
	Let $\alpha: I \rightarrow \R^3$ be a curve parameterised by arclength $s \in I$. The number $|| \alpha''(s) || = \kappa (s)$ is called the \dfn{curvature} of $\alpha$ at $s$. 
\end{definition}
It's straightforward to check that $\kappa(s) = 0 \iff \alpha(s) = us + v$ (i.e., the curve is actually a straight line). When $\kappa(s) \neq 0$, the \dfn{unit normal} $n(s)$ in the direction $\alpha''(s)$ is well-defined and is given by: 
\begin{align*}
	\alpha''(s) := \kappa(s) \cdot n(s)
\end{align*}
The orthogonality of $n(s)$ to $\alpha'(s)$ can be verified by differentiating both sides of $\alpha'(s) \cdot \alpha'(s) = 1$ since $||\alpha'(s)|| =1$. 

\begin{definition}[Osculating Plane at $s$]
	The \dfn{osculating plane} at $s$ is the plane determined by the unit tangent and normal vectors, $\alpha'(s)$, and $n(s)$. 
\end{definition}

\begin{definition}[Binormal Vector at $s$, $b(s)$] 
	The \dfn{binormal vector} as $s$ is defined as $t(s) \wedge n(s)$, where $t(s)$ is the unit tangent at $s$. The magnitude of this vector, $||b(s)||$, measures how rapidly the curve pulls away from the osculating plane at $s$ in a neighbourhood of $s$. 
\end{definition}

\begin{definition}[Torsion]
	Let $\alpha: I \rightarrow \R^3$ be a curve parameterised by arclength $s$ such that $\alpha''(s) \neq 0$, $s \in I$. The number $\tau(s)$ defined by $b'(s) := \tau(s)  n(s)$ is called the \dfn{torsion} of $\alpha$ at $s$. We have the following useful characterisation: 
	\begin{align*}
		\alpha \text{ is a plane curve} \iff \tau \equiv 0
	\end{align*}
	Thus, torsion measures how much a curve \emph{fails} to be a plane curve. 
\end{definition}

Collecting the orthogonal unit vectors $t(s), n(s), b(s)$ gives us the \dfn{Frenet Trihedron} at $s$. Using the above definitions gives us the \dfn{Frenet Formulae}, which is a set of differential equations: 
\begin{align}
	& t' = \kappa n \\
	& n' = - \kappa t - \tau b \\
	& b' = \tau n 	
\end{align}
\begin{itemize}[noitemsep]
	\item The $tb$ plane is called the \dfn{rectifying plane} 
	\item The $nb$ plane is called the \dfn{normal plane} 
	\item $\kappa$ and $\tau$ completely describe a curve's behaviour. 
	\item Bending $\sim$ curvature; twising $\sim$ torsion. 
\end{itemize}
The Frenet-Serret frame can be concisely expressed as a skew-symmetric matrix: 
\begin{align}
	\begin{bmatrix}
		T' \\
		N ' \\
		B' 
	\end{bmatrix}	 = \begin{bmatrix}
		0 & \kappa & 0 \\
		- \kappa & 0 & \tau \\
		0 & - \tau & 0 
	\end{bmatrix} \cdot \begin{bmatrix}
		T \\
		N \\
		B
	\end{bmatrix}
\end{align}

\begin{theorem}[Fundamental Theorem of the Local Theory of Curves]
	Given differentiable functions  $\kappa(s) > 0$ and $\tau(s)$, $s \in I$, there exists a regular parameterised curve $\alpha: I \rightarrow \R^3$ such that $s$ is the arclength, $\kappa(s)$ is the curvature, and $\tau(s)$ is the torsion of $\alpha$. Moreover, any other curve $\widetilde{\alpha}$ satisfying the same conditions differ from $\alpha$ by a \underline{rigid motion}. 
\end{theorem}

\begin{definition}[Rigid Motion]
	A \dfn{rigid motion} means that $\exists$ an orthogonal map $\rho$ of $\R^3$ with positive determinant and a vector $c$ such that $\widetilde{\alpha} = \rho \circ \alpha + c$. 
\end{definition}

Without loss of generality, we can assume curves to be parameterised by arclength, since we can always re parameterise a parameterised curve by arclength: 

Let $\alpha: I \rightarrow \R^3$ be a regular parameterised curve. Then, it is possible to obtain a curve $\beta: J \rightarrow \R^3$ that is parameterised by arc length with the same trace as $\alpha$: 
\begin{align*}
	s = s(t) = \idx{t_0}{t} | \alpha'(t) | dt
\end{align*}
where $t, t_0 \in I$. 

\subsection{Global Properties of Curves}

\subsubsection{The Isoparametric Inequality}
This is related to the following isoparametric question: 
\begin{center}
	\textbf{\underline{Q}}: Of all the simple closed curves in the plane with a given length, which bounds the largest area? 
\end{center}
We will use the following formula for the area $A$ bounded by a positively oriented simple closed curve $\alpha(t) = (x(t), y(t))$:
\begin{align*}
	A = - \idx{a}{b} y(t) x'(t) dt = \idx{a}{b} x(t)y'(t) dt = \frac{1}{2} (xy' - yx') dt 
\end{align*}

\begin{theorem}[The Isoparametric Inequality]
	Let $C$ be a simple closed plane curve with length $\ell$ and let $A$ be the area of the region bounded by $C$. Then: 
	\begin{align}
		\ell^2 - 4 \pi A \geq 0 	
	\end{align}
	where equality holds $\iff$ $C$ is a circle. 

\end{theorem}

\subsubsection{Cauchy Crofton Formula}

\begin{theorem}[Cauchy Crofton Formula]
	Let $C$ be a regular plane curve with length $\ell$. The measure of the set of straight lines, counted with multiplicities (\dfn{multiplicity} is the number of intersection points of a line with $C$), which meet $C$ is equal to $2 \ell$. 
\end{theorem}

\begin{definition}[Rigid Motion in $\R^2$]
	A \dfn{rigid motion} in $\R^2$ is a map $F: \R^2 \rightarrow \R^2$ given by $(\overline{x}, \overline{y}) \rightarrow (x,y)$, where: 
	\begin{align*}
		& x = a + \overline{x} \cos(\varphi) - \overline{y} \sin( \varphi) \\
		& y = b + \overline{x} \sin( \varphi) + \overline{y} \cos (\varphi) 
	\end{align*}
\end{definition}

\begin{prop}
	Let $f(x,y)$ be a continuous function defined in $\R^2$. For any set $S \subseteq \R^2$, define the \dfn{area $A$ of $S$} by: 
	\begin{align}
	A(S) := \iint_{S} f(x,y) dx dy 	
	\end{align}
	Assume that $A$ is invariant under rigid motions; that is, if $S$ is a set and $\overline{S} = F^{-1}(S)$, where $F$ is a rigid motion, then if: 
	\begin{align*}
		A (\overline{S} ) = \iint_{\overline{S}} f(\overline{x}, \overline{y}) d \overline{x} d \overline{y} 	= \iint_{S} f(x,y) dx dy 	= A(S)
	\end{align*}
	Then, $f(x,y)$ is a constant. 
\end{prop}


\section{Surfaces}

\subsection{Definitions}

\textbf{Motivation:} we want to define a regular surface to be something that is nice enough for us to extend the usual notions of calculus to. 


\begin{definition}[Regular Surface]
	A subset $S \subseteq \R^3$ is called a \dfn{regular surface} if, $\forall$ $p \in S$, there exists a neighbourhood $V \subseteq \R^3$ and a map $\mathbb{X}: U \rightarrow V \cap S$ of an open set $V \subseteq \R^2$ onto $V \cap S \subseteq \R^3$ for which the following conditions hold: 
	\begin{enumerate}[noitemsep]
		\item $\mathbb{X}$ is differentiable; that is, if we write 
		\begin{align*}
			\mathbb{X}(u,v) = (x(u,v), y(u,v), z(u,v)) 
		\end{align*} 
		for $(u,v) \in U$, then the functions $x(u,v)$, $y(u,v)$ and $z(u,v)$ have continuous partial derivatives of all orders in $U$. 
		\item $\mathbb{X}$ is a \dfn{homeomorphism}: there exists an inverse $\mathbb{X}^{-1}: V \cap S \rightarrow U$, which is continuous. 
		\item (Regularity Condition): $\forall q \in U$, the differential $d$x$_q: \R^2 \rightarrow \R^3$ is bijective. 
	\end{enumerate}
	Then, the mapping $\mathbb{X}$ is called a \dfn{parameterisation}  or a \dfn{system of local coordinates} in a neighbourhood of $p$. The neighbourhood $V \cap S$ of $p$ is called a \dfn{coordinate neighbourhood}. 
\end{definition}


\subsection{Regular Surfaces}
\begin{ex}[The Unit Sphere is a Regular Surface] 
	The Unit Sphere is a regular surface. It's parametrised by: 
	\begin{align*}
		S^2 := \{ (x,y,z) \in \R^2\ |\ x^2 + y^2 + z^2 = 1 \} 
	\end{align*}
\end{ex}
In the textbook, they check all three conditions from the above definition. Since this can be quite exhausting, we want some propositions that simplify the task of determining if a surface is regular or not. This is the aim of this section. 

\begin{prop}
	If $f: U \subseteq \R^2 \rightarrow \R$, $U$ open, is a differentiable, then the graph of $f$, that is, the subset of $\R^3$ given by $(x,y, f(x,y))$ for $(x,y) \in U$, is a regular surface. 
\end{prop}

Before introducing the second proposition, we will first need to define critical points, critical values, and regular values for differentiable maps. 

\begin{definition}[Critical Point] 
	Given a differentiable map $F: U \subseteq \R^n \rightarrow \R^m$ defined in an open set $U \subseteq \R^n$, we say that $p \in U$ is a \dfn{critical point} of $F$ id the differential d$F_p: \R^n \rightarrow \R^m$ is not a surjective mapping. The image $F(p) \in \R^m$ of a critical point is called a \dfn{critical value} of $F$. A point $\R^m$ which is not a critical value is called a \dfn{regular value}. 
\end{definition}

The justification for the next proposition comes from the inverse function theorem. 

\begin{prop}
	If $f: U \subseteq \R^3 \rightarrow \R$ is a differentiable function and $a \in f(U)$ is a regular value of $f$, then $f^{-1}(a)$ is a regular surface in $\R^3$. 
\end{prop}

\begin{ex}[Ellipsoid] 
		The ellipsoid is given by: 
		\begin{align*}
			\frac{x^2}{a^2} + \frac{y^2}{b^2} + \frac{z^2}{c^2} = 1
		\end{align*}
		Since it is the set $f^{-1}(0)$ where 
		\begin{align*}
			f(x,y,z) = \frac{x^2}{a^2} + \frac{y^2}{b^2} + \frac{z^2}{c^2} - 1
		\end{align*}
		and $f$ is a differentiable function and $0$ is a regular value of $f$. 
\end{ex}

\begin{definition}[Connected] 
	A surface $S \subseteq \R^3$ is \dfn{connected} if any two of its points can be joined by a continuous curve in $S$. 
\end{definition}

The next proposition is a very useful property that follows from the intermediate value theorem: 

\begin{definition}
	If $f: S \subseteq \R^3 \rightarrow \R$ is a non-zero continuous function defined on a connected surface $S$, then $f$ does not change sign on $S$. 
\end{definition}



\subsection{Differentiable Functions on Surfaces}

\subsection{Tangent Plane}

The third condition of a regular surface guarantees that for any fixed point $p \in S$, the set of tangent vectors to the parameterised curves of $S$ passing through $p$ constitutes a plane. 

\begin{prop}
	Let $\mathbb{X}: U \subseteq \R^2 \rightarrow S$ be a parameterisation of a regular surface $S$ and let $q \in U$. The vector subspace of dimension 2: 
	\begin{align}
		\text{d}x_q (\R^2) \subseteq \R^3 	
	\end{align}
	coincides with the set of tangent vectors to $S$ at $\mathbb{X}(q)$. 
\end{prop}
This plane does not depend on the parameterisation $\mathbb{X}$ and it is called the \dfn{tangent plane} to $S$ at $p$ and is denoted by $T_p(S)$. A choice of parameterisation $\mathbb{X}$ induces a basis on $T_p(S)$: 
\begin{align*}
	\{ (\partial \mathbb{X}/ \partial u)(q), ( \partial \mathbb{X}/ \partial v)(q) \} 
\end{align*}


The next proposition states that a map between two regular surfaces induces a map between the tangent planes, which we can think of as the differential of the map. 

\begin{prop}
	Let $S_1$, $S_2$ be regular surfaces and let $\varphi: V \subseteq S_1 \rightarrow S_2$ be a differentiable mapping of an open set $V$ of $S_1$ into $S_2$. Then, tangent vectors $w \in T_p(S_1)$ are the velocity vectors $\alpha'(0)$ of a differentiable parameterised curve $\alpha: ]-\varepsilon, \varepsilon[ \rightarrow V$ with $\alpha(0) = p$. If we define $\beta:= \varphi \circ \alpha$, then $\beta '(0)$ is a vector of $T_{\varphi(p)}(S_2)$. Given a $w$, the vector $\beta'(0)$ does not depend on the choice of $\alpha$ and the map d$\varphi_p: T_p(S_1) \rightarrow T_{\varphi(p)}(S_2)$  defined by d$\varphi_p(w) = \beta'(0)$ is linear. 
\end{prop}
Before moving onto the next proposition, we first need to define what a local diffeomorphism is. The aim is to build up to a generalisation of the standard inverse function theorem from calculus. 

\begin{definition}[Local Diffeomorphism] 
	A mapping $\varphi: U \subseteq S_1 \rightarrow S_2$ is called a \dfn{local diffeomorphism} at $p \in U$ if there is a neighbourhood $V \subseteq U$ of $p$ such that $\varphi|_U$ is a diffeomorphism onto an open set $\varphi(V) \subseteq S_2$. 
\end{definition}

\begin{prop}
	If $S_1$ and $S_2$ are regular surfaces and $\varphi: U \subseteq S_1 \rightarrow S_2$ is a differentiable mapping of an open set $U \subseteq S_1$ such that the differential d$\varphi_p$ of $\varphi$ at $p \in U$ is an isomorphism, then $\varphi$ is a local diffeomorphism at $p$. 
\end{prop}

For any point on a regular surface, we can find two unit normal vectors. By fixing a parameterisation $\mathbb{X}: U \subseteq \R^2 \rightarrow S$ for $p \in S$, we can make a definite choice of a unit normal at each point $q \in \mathbb{X}(U)$ by the following rule: 
\begin{align}
	N(q) := \frac{\mathbb{X}_u \wedge x_v}{|| x_u \wedge x_v ||} (q) 	
\end{align}
This gives us a differentiable map $N: \mathbb{X}(U) \rightarrow \R^3$. 



\subsection{First Fundamental Form: Area}
\textbf{Motivation:} the natural inner product on $\R^3$ induces on each regular surface $S \subseteq \R^3$'s tangent plane $T_p(S)$ an inner product, $\langle \cdot, \cdot \rangle_p$. The aim of the First Fundamental Form is to express how a surface inherits the natural inner product of $\R^3$. This allows us to make metric measurements of the surface, such as lengths of curves, angles of tangent vectors, and areas of regions without referring to the ambient space in which they reside.  


\begin{definition}[First Fundamental Form]
	Let $w_1$, $w_2 \in T_p(S) \subseteq \R^3$. Then, the quadratic form given by $I_p: T_p(S) \rightarrow \R$: 
	\begin{align}
		I_p(w) := \langle w, w \rangle_p = || w ||^2 > 0 	
	\end{align}
	is called the \dfn{First Fundamental Form} of the regular surface $S \subseteq \R^3$ at $p \in S$. 
\end{definition}

\subsubsection{Deriving the First Fundamental Form Given a Basis and a Parameterisation}

Let $\mathbb{X}(u,v)$ be a parametrisation. We will now express the first fundamental form in the basis $\{ \mathbb{X}_u, \mathbb{X}_v \}$ associated to a parameterisation $\mathbb{X}(u,v)$ at $p$. Recall that a tangent vector $w \in T_p(S)$ is equivalent to a tangent vector to a parameterised curve $\alpha(t) = \mathbb{X}(u(t), v(t) ) $ for $t \in ]-\varepsilon, + \varepsilon[$ for which $p = \alpha(0) = \mathbb{X}(u_0, v_0)$. 

From the definition of the first fundamental form, we have:
\begin{align*}
	 I_p(\alpha'(0)) & = \langle \alpha'(0), \alpha'(0) \rangle_p \\
	 				 & = \langle \mathbb{X}_u u' + \mathbb{X}_v v', \mathbb{X}_u u' + \mathbb{X}_v v' \rangle_p \\
	 				 & = \langle \mathbb{X}_u, \mathbb{X}_u \rangle_p (u')^2 + 2 \langle \mathbb{X}_u, \mathbb{X}_v \rangle u' v' + \langle \mathbb{X}_v, \mathbb{X}_v \rangle_p (v')^2
\end{align*}
If we define 
\begin{align*}
	& E(u_0, v_0) := \langle \mathbb{X}_u, \mathbb{X}_u \rangle_p  \\
	& F(u_0, v_0) := \langle \mathbb{X}_u, \mathbb{X}_v \rangle_p  \\
	& G(u_0, v_0) := \langle \mathbb{X}_v, \mathbb{X}_v \rangle_p 
\end{align*}
then the first fundamental form can be expressed as: 
\begin{align*}
	I_p = E (u')^2 + 2 F u' v' + G(v')^2
\end{align*}

\subsubsection{Examples of First Fundamental Forms}

\begin{enumerate}[noitemsep]
	\item Recall that the \dfn{plane} going through $p_0 = (x_0, y_0, z_0)$ containing the orthonormal vectors $w_1 = (a_1, a_2, a_3)$ and $w_2 = (b_1, b_2, b_3)$ is given by: 
	 \begin{align*}
		\mathbb{X}(u,v)  = p_0 + uw_1 + vw_2 
	\end{align*}
	for $(u,v) \in \R^2$. Then, $E = 1$, $F=0$, and $G=1$. 
	\item The \dfn{cylinder} over the circle $x^2 + y^2 = 1$ parameterised by $\mathbb{X} (u,v) = (\cos (u) , \sin(v), v) $ where $u \in ]0, 2 \pi[$ and $v \in \R$. Then: $E = \sin^2 (u) + \cos^2(u) =1$, $F =0$, and $G=1$. 
	\item The \dfn{Helicoid} is given by: $\mathbb{X}(u,v) := (v \cos (u), v \sin(u) au)$. $ u \in ]0, 2 \pi[$, $v \in \R$. The first fundamental form is given by: $ E = v^2 + a^2$, $F(u,v) = 0$, and $G(u,v) = 1$. 
\end{enumerate}
We can express arclength in terms of the terms of the functions of the first fundamental form. Let $s$ be an arclength-parameterised curve $\alpha: I \rightarrow s$. Then, the arc-length is: 
\begin{align*}
	s(t) = \idx{0}{t} | \alpha'(t) | dt = \idx{0}{t} \sqrt{I(\alpha'(t))} dt
\end{align*}
Substituting in the derivation gives us: 
\begin{align*}
	s(t) = \idx{0}{t} \sqrt{E(u')^2  + 2Fu' v' + G(v')^2} dt 
\end{align*}
We can also represent angles of intersections of parameterised curves using the coefficients of the first fundamental form. Let $\alpha: I \rightarrow S$ and $\beta: I \rightarrow S$ be two parameterised curves. The angle $\theta$ at which they intersect at $t=t_0$ is given by: 
\begin{align} 
\cos(\theta) = \frac{\langle a'(t_0), \beta'(t_0) \rangle}{|| \alpha'(t_0) || || \beta'(t_0) ||}
\end{align} 
In terms of the coefficients of the first fundamental form, we have: 
\begin{align*}
	\cos(\theta) = \frac{\langle x_u, x_v \rangle}{||x_u|| ||x_v || } = \frac{F}{\sqrt{EG}}
\end{align*}
A special type of parameterisation is called an \dfn{orthogonal parameterisation}, which is a parameterisation where the coordinate curves of a parameterisation are orthogonal. By the above, this happens if and only if $F(u,v) = 0$ for all $u,v \in S$. Moreover, from the arc length formula, an \dfn{element of arclength} is given by: 
\begin{align*}
	ds^2 = E du^2 + 2F du dv + G dv^2 
\end{align*}
One final classic example of computing first fundamental forms is that of a sphere. If we parameterise a sphere as: 
\begin{align*}
	\mathbb{X}(\theta, \varphi) = ( \sin \theta \cos \varphi, \sin \theta, \sin \varphi, - \sin \theta ) 
\end{align*}
Then, the coefficients of the first fundamental form become: 
\begin{align*}
	& E(\theta, \varphi) = 1 \\
	& F(\theta, \varphi) = 0 \\
	& G(\theta, \varphi) = \sin^2 (\theta) 
\end{align*}
Then, for a vector $w \in T_p(S)$ at the point $p$ with the coordinates based on the basis associated to the parametrisation $\mathbb{X}( \theta, \varphi)$, we write: 
\begin{align*}
	w = a \mathbb{X}_\theta + b \mathbb{X}_\varphi 
\end{align*}
and so
\begin{align*}
	||w||^2 = I(w) = Ea^2 + 2Fab + Gb^2 = a^2 + b^2 \sin^2 \theta 
\end{align*}
We can use the first fundamental form to compute areas. 
\begin{definition}[Area]
	Let $R \subseteq S$ be a bounded region of a regular surface contained in the coordinate neighbourhood of the parameterisation $\mathbb{X}: U \subseteq \R^2 \rightarrow S$. Then, the positive number: 
	\begin{align*}
		A(R) := \iint_Q || \mathbb{X}_u \wedge \mathbb{X}_v || du dv 
	\end{align*}
	where $Q = \mathbb{X}^{-1}(R)$ is called the \dfn{area} of $R$. This is equivalent to, in terms of the first fundamental form: 
	\begin{align*}
		& = \iint_Q \sqrt{EG - F^2} du dv 
	\end{align*}
\end{definition}

\section{The Gauss Map}
\textbf{Motivation:} try to measure how rapidly a surface $S$ pulls away from the tangent plane $T_p(S)$ in a neighbourhood of a point $p \in S \leftrightarrow$ measuring the rate of change at $p$ of a unit normal vector field $N$ on a neighbourhood of $p$. This gives rise to a linear map on $T_p(S)$ that is self-adjoint. This map happens to give us a lot of information about local properties of the surface $S$ at $p$. 

\subsection{The Definition of the Gauss Map and its Fundamental Properties}

\begin{itemize}[noitemsep]
	\item $N$ is said to be a \dfn{differentiable field of unit normal vectors on} an open set $V \subseteq S$ if $N: V \rightarrow \R^3$ is a differentiable map which associates to each $q \in V$ a unit normal vector at $q$. 
	\item A regular surface $V$ is called \dfn{orientable} if it admits a differentiable field of unit normal vectors defined on the whole surface. 
	\begin{itemize}[noitemsep]
		\item The Möbius strip is an example of a non-orientable surface. 
		\item The choice of such a field $N$ is called an \dfn{orientation} of $S$. 
		\item Every surface is locally orientable. 
		\item Orientation is a global property in the sense that it involves the \emph{whole} surface. 
	\end{itemize}
\end{itemize}

The Gauss map is the map which assigns unit normals to points on surfaces. We derived this map in homework 1. 

\begin{definition}[Gauss Map] 
	Let $S \subseteq \R^3$ be a surface with orientation $N$. The map $N: S \rightarrow \R^3$ takes its values in the unit sphere: 
	\begin{align}
		S^2 := \{ (x,y,z) \in \R^3\ |\ x^2 + y^2 + z^2 = 1 \} 	
	\end{align}
	This map $N: S \rightarrow S^2$ as defined is called the \dfn{Gauss Map} of $S$. 
\end{definition}
The differential induced by the Gauss Map, d$N_p: T_p(S) \rightarrow T_{N(p)}(S)$ , is a linear map. Restricting the map to a parameterised curve $\alpha(t)$ in $S$ provides for us a measure of how $N$ pulls away from $N(p)$ in a neighbourhood of $p$. For curves, this information is encoded in the curvature, a scalar. For surfaces, the ``notion'' of curvature is encoded as a linear map. 

Here are several examples of what $dN$ would be for some surfaces. 
\begin{enumerate}[noitemsep]
	\item The \dfn{plane} has zero ``curvature.'' Parameterise this plane by $ax + by + cz + d =0$. Then, the unit normal vector is given by: 
	\begin{align*}
		N = \frac{(a,b,c)}{\sqrt{a^2 + b^2 + c^2}}
	\end{align*}
\end{enumerate}



\subsection{Ruled Surfaces and Minimal Surfaces}


\section{The Intrinsic Geometry of Surfaces}

\subsection{Isometries and Conformal Maps}





\end{document}