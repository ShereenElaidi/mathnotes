\documentclass[11pt]{scrartcl}
\usepackage[margin=2cm]{geometry}
\usepackage{amsmath}
\usepackage{amsfonts}
\usepackage{amssymb,amsmath,amsthm}
\usepackage{xcolor} 
\usepackage{enumitem}
\newcommand{\R}[0]{\mathbb{R}}
\addtokomafont{section}{\rmfamily\centering\scshape}
% math environments 
\usepackage[utf8]{inputenc}
\theoremstyle{definition}
\newtheorem{theorem}{Theorem}
\newtheorem{corollary}{Corollary}
\newtheorem{lemma}[theorem]{Lemma}
\newtheorem{definition}{Definition}
\newtheorem{prop}{Proposition}
\newtheorem{ex}{Example}
\theoremstyle{remark}
\newtheorem*{remark}{Remark}

% definition
\newcommand{\dfn}[1]{\textbf{\underline{#1}}}
\newcommand{\dist}[0]{\mathcal{F}}
\newcommand{\pr}[1]{\mathbb{P}[#1]} 
\newcommand{\stat}[0]{T(X_1, ..., X_n )} 

% converge in probability 
\newcommand{\cvp}[0]{\overset{p}{\to}}

% sample mean
\newcommand{\smean}[0]{\frac{1}{n} \sum_{i=1}^n x_i} 

% sample variance
\newcommand{\svar}[0]{\frac{1}{(n-1)} \sum_{i=1}^n (x_i - \overline{x})^2}

% expected value 
\newcommand{\EX}[1]{\mathbb{E}\left[#1 \right]}  
\newcommand{\EXth}[1]{\mathbb{E}_\theta \left[ #1 \right]}

% integral
\newcommand{\idx}[2]{\int_{#1}^{#2}}

% vector
\newcommand{\vect}[1]{\mathbf{#1}}


\title{\textbf{Math 458: Differential Geometry}}
\author{Shereen Elaidi}
\date{Winter 2020 Term}

\begin{document}

\maketitle
\tableofcontents

\section{Introduction}
\subsection{Implicit and Inverse Function Theorems}

\section{Manifolds in $\R^3$}
The aim of this part of the course is to build up to integration on manifolds and the invariant Stokes' theorem. The main purpose of this sections is to develop \emph{coordinate-free} calculus, which clarifies the essence of what is happening (sometimes coordinates can be noisy). 
\subsection{Definitions}

\subsection{Smooth Maps from $M^m \rightarrow N^n$}

\subsection{Change of Coordinates}

\subsection{Multi-Linear Algebra}

\subsection{Differential Forms in $M^n$}

\subsection{Change of Variables for Integrals in $\R^n$}

\subsection{Integrating a $n$-Form on $M^n$ ($\idx{M}{} \omega$)} 

\section{Curves}

\subsection{Definitions}

\subsection{Frenet-Serret Frame}

\subsection{Global Properties of Curves}

\subsubsection{The Isoperimetric Inequality}

\subsubsection{Cauchy Crofton Formula}




\section{Surfaces}

\subsection{Definitions}

\textbf{Motivation:} we want to define a regular surface to be something that is nice enough for us to extend the usual notions of calculus to. 


\begin{definition}[Regular Surface]
	A subset $S \subseteq \R^3$ is called a \dfn{regular surface} if, $\forall$ $p \in S$, there exists a neighbourhood $V \subseteq \R^3$ and a map $\mathbb{X}: U \rightarrow V \cap S$ of an open set $V \subseteq \R^2$ onto $V \cap S \subseteq \R^3$ for which the following conditions hold: 
	\begin{enumerate}[noitemsep]
		\item $\mathbb{X}$ is differentiable; that is, if we write 
		\begin{align*}
			\mathbb{X}(u,v) = (x(u,v), y(u,v), z(u,v)) 
		\end{align*} 
		for $(u,v) \in U$, then the functions $x(u,v)$, $y(u,v)$ and $z(u,v)$ have continuous partial derivatives of all orders in $U$. 
		\item $\mathbb{X}$ is a \dfn{homeomorphism}: there exists an inverse $\mathbb{X}^{-1}: V \cap S \rightarrow U$, which is continuous. 
		\item (Regularity Condition): $\forall q \in U$, the differential $d$x$_q: \R^2 \rightarrow \R^3$ is bijective. 
	\end{enumerate}
	Then, the mapping $\mathbb{X}$ is called a \dfn{parameterisation}  or a \dfn{system of local coordinates} in a neighbourhood of $p$. The neighbourhood $V \cap S$ of $p$ is called a \dfn{coordinate neighbourhood}. 
\end{definition}

\subsection{Differentiable Functions on Surfaces}

\subsection{Tangent Plane}

\subsection{First Fundamental Form: Area}


\section{The Gauss Map}

\subsection{Ruled Surfaces and Minimal Surfaces}


\section{The Intrinsic Geometry of Surfaces}

\subsection{Isometries and Conformal Maps}





\end{document}