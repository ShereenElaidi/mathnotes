\documentclass[11pt]{article}
\usepackage{titlesec}
\titleformat{\section}[hang]{\normalfont\scshape}{\thesection.}{1em}{}
\titleformat{\subsection}[hang]{\normalfont\scshape}{\thesubsection.}{1em}{}
\usepackage[T1]{fontenc}
\usepackage[dvipsnames]{xcolor}
\usepackage[margin=2cm]{geometry}
\usepackage{amssymb} 
\usepackage{amsmath}
\usepackage{graphicx}
\usepackage{float}
\usepackage[normalem]{ulem} 
\usepackage{enumitem} 
\setlist[enumerate]{itemsep=0mm}
\usepackage{xcolor}
\setenumerate{label=(\arabic*)}
\usepackage[utf8]{inputenc}
\usepackage[T1]{fontenc}
\usepackage{babel}
\usepackage{mathtools}
\usepackage{amsthm}
\usepackage{thmtools}
\usepackage{etoolbox}
\usepackage{fancybox}
\usepackage{framed}
\usepackage{tcolorbox}
\usepackage{xcolor} 
% open 
\newcommand{\open}[0]{\mathcal{O}}
\newcommand{\topo}[0]{\mathcal{T}}
\newcommand{\hood}[0]{\mathcal{U}}
\newcommand{\base}[0]{\mathcal{B}} 
% example environmentq
\theoremstyle{definition} 
\newtheorem{exmp}{Example}[section]

% linear operators
\newcommand{\lop}[2]{\mathcal{L}(#1, #2)}

% question environment
\theoremstyle{definition}
\newtheorem{question}{Question}

%rd 
\newcommand{\rd}[0]{\mathbb{R}^d}
\newcommand{\R}[0]{\mathbb{R}}
\newcommand{\N}[0]{\mathbb{N}} 
% let E in R be measurable 
\newcommand{\EinR}[0]{Let $E \subseteq \R$ be measurable}

%integral 
\newcommand{\idx}[2]{\int_{#1}^{#2}}

% weak convergence 
\newcommand{\warrow}[0]{\rightharpoonup}
\newcommand{\fcvw}[0]{ \{f_n \} \warrow f \text{ in } L^p(E)} 

% Probability 
\DeclareRobustCommand{\bbone}{\text{\usefont{U}{bbold}{m}{n}1}}
\newcommand{\Var}[1]{\mathrm{Var[#1]}}			% variance
\newcommand{\EX}[1]{\mathbb{E}\mathrm{[#1]}}	 % expected value 
\newcommand{\seq}[1]{\{ #1_n	\}_{n \in \bb{N}}} % sequence of events
\newcommand{\pspace}[0]{( \Omega, F, P)}		% probability space
\newcommand{\msp}[0]{( \Omega, F)}		% measurable space
	
% Exercise environment 
\newenvironment{myleftbar}{%
\def\FrameCommand{\hspace{0.6em}\vrule width 2pt\hspace{0.6em}}%
\MakeFramed{\advance\hsize-\width \FrameRestore}}%
{\endMakeFramed}
\declaretheoremstyle[
spaceabove=6pt,
spacebelow=6pt
headfont=\normalfont\bfseries,
headpunct={} ,
headformat={\cornersize*{2pt}\ovalbox{\NAME~\NUMBER\ifstrequal{\NOTE}{}{\relax}{\NOTE}:}},
bodyfont=\normalfont,
]{exobreak}

\declaretheorem[style=exobreak, name=Exercise,%
postheadhook=\leavevmode\myleftbar, %
prefoothook = \endmyleftbar]{exo}

% Solution environment 
\newenvironment{mysolbar}{%
\def\FrameCommand{\hspace{0.6em}\vrule width 2pt\hspace{0.6em}}%
\MakeFramed{\advance\hsize-\width \FrameRestore}}%
{\endMakeFramed}
\declaretheoremstyle[
spaceabove=6pt,
spacebelow=6pt
headfont=\normalfont\bfseries,
headpunct={} ,
headformat={\cornersize*{2pt}\ovalbox{\NAME~\NUMBER\ifstrequal{\NOTE}{}{\relax}{\NOTE}:}},
bodyfont=\normalfont,
]{solbreak}

\declaretheorem[style=solbreak, name=Solution,%
postheadhook=\leavevmode\mysolbar, %
prefoothook = \endmysolbar]{sol}

% HEADERS
\usepackage{fancyhdr}
 
\pagestyle{fancy}
\fancyhf{}
\fancyhead[LE,RO]{Page \thepage}
\fancyhead[RE,LO]{Math 455: Analysis 4}
\fancyhead[CE,CO]{Winter 2020 -- Final Summary}
\fancyfoot[LE,RO]{}

% Definitions
\newcommand{\dfn}[1]{\underline{\textbf{#1}}}
\newcommand{\im}[1]{\textbf{\textcolor{red}{#1}}}

% lower integral
\usepackage{accents}

\newcommand{\ubar}[1]{\underaccent{\bar}{#1}}
\def\avint{\mathop{\,\rlap{-}\!\!\int}\nolimits} 

% custom commands 
\newcommand{\bb}[1]{\mathbb{#1}}
\newcommand{\vc}[1]{\mathbf{#1}}
\newcommand{\step}[1]{\textbf{#1}\textbf{. Step:}}
\newcommand{\pdv}[2]{\frac{\partial #1}{\partial #2}}
\newcommand{\sets}[2]{ \left\{ #1\ |\ #2 \right\}}
\DeclareMathOperator{\Tr}{Tr}

% Proofs
\newcommand{\claim}[1]{\textbf{#1}\textbf{. Claim:}}

	% iff proofs
	\newcommand{\rhs}[0]{(\Rightarrow )}
	\newcommand{\lhs}[0]{(\Leftarrow )}

% sequence of functions
\newcommand{\funcseqx}{(f_n(x))_{n \in \bb{N}}}
\newcommand{\funcseq}{(f_n)_{n \in \bb{N}}}

% measurable sets 
\newcommand{\measurable}{f^{-1}([-\infty, c[)} 

% heat equation 
\newcommand{\pbdry}[2]{C^{(#1, #2)} (\Omega_T) \cap C (\overline{\Omega_T})}
\DeclareMathOperator\erf{erf}
\newcommand{\mbf}[1]{\mathbf{#1}}

% Laplace Equation 
\newcommand{\lapbdry}[1]{C^{#1} (\Omega) \cap C (\overline{\Omega})}


% math environments 
\usepackage[utf8]{inputenc}
\newtheorem{theorem}{\textcolor{blue}{Theorem}}
\newtheorem{corollary}{Corollary}
\newtheorem{lemma}[theorem]{Lemma}
\theoremstyle{definition}
\newtheorem{definition}{\textcolor{OliveGreen}{Definition}}
\newtheorem{prop}{\textcolor{red}{Proposition}}
\newtheorem{ex}{\textcolor{Maroon}{Example}}
\theoremstyle{remark}
\newtheorem*{remark}{Remark}

% cookbook proofs 
\newcommand{\cb}[3]{\underline{(#1 #2): #3:}}

\usepackage{tcolorbox}
\tcbuselibrary{theorems}

% theorems 
\newtcbtheorem[number within=section]{mytheo}{Theorem}%
{colback=blue!5,colframe=blue!35!black,fonttitle=\bfseries}{th}

% definitions 
\newtcbtheorem[number within=section]{defn}{Definition}%
{colback=black!5,colframe=black!35!black,fonttitle=\bfseries}{th}

% axioms
\newtcbtheorem[number within=section]{ax}{Axioms}%
{colback=OliveGreen!5,colframe=black!35!OliveGreen,fonttitle=\bfseries}{th}


% important examples
\newtcbtheorem[number within=section]{examp}{Example}%
{colback=Mahogany!5,colframe=black!35!Mahogany,fonttitle=\bfseries}{th}

% upper and lower riemann integrals
\newcommand{\upRiemannint}[2]{
  \overline{\int_{#1}^{#2}}
}
\newcommand{\loRiemannint}[2]{
  \underline{\int_{#1}^{#2}}
}

\usepackage{hyperref}
\hypersetup{
    colorlinks,
    citecolor=black,
    filecolor=black,
    linkcolor=black,
    urlcolor=black
}

\begin{document}

\begin{center}
	\textbf{Math 455: Analysis IV Summary} \\
	\textbf{Midterm Date: 27 April 2020 14.00 - 17.00} \\
	\textbf{Key Results, Theorems, Definitions, etc.} \\
	\textbf{Shereen Elaidi}
\end{center}

\begin{abstract}
	This document contains a summary of all the key definitions, results, and theorems from class. There are probably typos, and so I would be grateful if you brought those to my attention :-). 
	
	Syllabus: $L^p$ space, duality, weak convergence, Young, Holder, and Minkowski inequalities, point-set topology, topological space, dense sets, completeness, compactness, connectedness, path-connectedness, separability, Tychnoff theorem, Stone-Weierstrass Theorem, Arzela-Ascoli, Baire category theorem, open mapping theorem, closed graph theorem, uniform boudnedness principle, Hahn Banch theorem. 
\end{abstract}

\tableofcontents

\section{$L^p$ Spaces: Completeness and Approximation}
\subsection{Normed Vector Spaces}

\begin{definition}[$\ell^p$ space]
	Let $(a_1, a_n, ...)$ be a sequence. Then, the $\ell^p$-space is: 
	\begin{align}
		\ell^p := \sets{(a_1, a_2, ...)}{\sum_{n=1}^\infty |a_n|^p < +\infty}
	\end{align}
\end{definition}

\begin{theorem}[Riesz-Fisher] 
	$L^p(X)$ is complete. 
\end{theorem}

\begin{definition}[$L^p$ space]
	Let $E$ be a measurable set and let $1 \leq p < \infty$. Then, $L^p(E)$ is the collection of measurable functions $f$ for which $|f|^p$ is Lebesgue integrable over $E$. 
\end{definition}

\begin{definition}[Equivalent Functions]
	Let $\mathcal{F}$ be the collection of all measurable extended real-valued functions on $E$ that are finite a.e. on $E$. Define two functions $f$ and $g$ to be equivalent, and write $f \sim g$ if $g(x) = f(x)$ a.e. on $E$.
\end{definition}

\begin{definition}[Essentially Bounded]
	We call a function $f \in \mathcal{F}$ to be \textbf{essentially bounded} if there exists some $M \geq 0$, called the \textbf{essential upper bound} for $f$, for which 
	\begin{align*}
		|f(x)| \leq M 	
	\end{align*}
	for almost every $x \in E$. $L^\infty(E)$ is the collection of equivalence classes $[f]$ for which $f$ is essentially bounded. 
\end{definition}

\begin{definition}[Norm]
	Let $X$ be a linear space. A real-valued functional $|| \cdot ||$ on $X$ is called a \textbf{norm} provided that for each $f$ and $g$ in $X$ and each real number $\alpha$, 
	\begin{enumerate}[noitemsep] 
		\item (The Triangle Inequality). 
		\begin{align*}
			|| f + g || \leq ||f|| + ||g|| 	
		\end{align*}
		\item (Positive Homogeneity). 	
		\begin{align*}
			|| \alpha f || = |\alpha | || f || 	
		\end{align*}
		\item (Non-Negativity). 
		\begin{align*}
			|| f || \geq 0 \text{ and } ||f|| = 0 \text{ if and only if } f = 0	
		\end{align*}
	\end{enumerate}
\end{definition}

\begin{definition}[Normed Linear Space]
	$X$ is said to be a \textbf{normed linear space} if $X$ is equipped with a norm. 
\end{definition}

\begin{definition}[Essential Supremum]
	Let $f \in L^\infty (E)$. $||f||_\infty$ is called the \textbf{essential supremum} and is defined as: 
	\begin{align*}
		||f ||_\infty := \sets{M}{M \text{ is an essential upper bound for } f}	
	\end{align*}
	\textbf{Theorem:} $|| \cdot ||_\infty$ is a norm on $L^\infty (E)$. 
\end{definition}

\subsection{The Inequalities of Young, Hölder, and Minkowski}

\begin{definition}[p-norm]
	Let $E$ be a measurable set, $1 < p < \infty$, and let $f \in L^p(E)$. Then, define the \textbf{p-norm} to be: 
	\begin{align}
		||f ||_p := \left[	\int_E |f|^p		\right]^{\frac{1}{p}} 
	\end{align}
\end{definition}
	
\begin{definition}[Conjugate]
	The \textbf{conjugate} of a number $p \in ]1, \infty[$ is the number $q = p/(p-1)$, which is the unique number $q \in ]1, \infty[$ for which 
	\begin{align}
		\frac{1}{p} + \frac{1}{q} = 1
	\end{align}
\end{definition}
The conjugate of 1 is defined to be $\infty$ and the conjugate of $\infty$ is defined to be 1. 

\begin{definition}[Young's Inequality]
	For $1 < p < \infty$, $q$ the conjugate of $p$, and any two positive numbers $a$ and $b$, we have: 
	\begin{align}
			ab \leq \frac{a^p}{p} + \frac{b^q}{q}
	\end{align}
\end{definition}

\begin{theorem}[Hölder's Inequality]
	Let $E \subseteq \bb{R}$ be measurable, $1 \leq p < \infty$, and $q$ the conjugate of $p$. If $f$ belongs to $L^p(E)$, and $g$ belongs to $L^q(E)$, then their product $f \cdot g$ is integrable over $E$ and: 
	\begin{align}
		\int_E | f \cdot g | \leq || f ||_p \cdot ||g||_q. 
	\end{align}
	Moreover, if $f \neq 0$, then the function defined as: 
	\begin{align}
		f^* := ||f||_p^{1-p} \cdot \text{sgn}(f) \cdot |f|^{p-1}
	\end{align}
	belongs to $L^q(E)$, 
	\begin{align*}
		\int_E f \cdot f^* = ||f||_p \text{ and } ||f^*||_q = 1	
	\end{align*}
	We call $f^*$ defined as above to be called the \textbf{conjugate function} of $f$. 
\end{theorem}

\begin{theorem}[Minkowski's Inequality]
	Let $E$ be a measurable set and $1 \leq p \leq \infty$. If the functions $f$ and $g$ belong to $L^p(E)$, then so does their sum $f+g$. Moreover, 
	\begin{align}
		||f+g||_p \leq ||f||_p + ||g||_p 
	\end{align}
\end{theorem}

\begin{theorem}[Cauchy-Schwarz Inequality]
	Let $E$ be a measurable set and let $f$ and $g$ be measurable functions over $E$ for which $f^2$ and $g^2$ are integrable over $E$. Then, $f \cdot g$ is integrable over $E$ and 
	\begin{align}
		\int_E |f \cdot g| \leq \sqrt{\int_E f^2} \cdot  \sqrt{\int_E g^2}	
	\end{align}
\end{theorem}

\begin{corollary}
	Let $E$ be a measurable set and $1 < p < \infty$. Suppose $\mathcal{F}$ is a family of functions  in $L^p(E)$ that is bounded in $L^p(E)$ in the sense that there is a constant $M$ for which 
	\begin{align*}
		|| f ||_p \leq M \text{ for all } f \in \mathcal{F}	
	\end{align*}
	Then, the family $\mathcal{F}$ is uniformly integrable over $E$. 
\end{corollary}

\begin{corollary}
	Let $E$ be a measurable set of finite measure and $1 \leq p_1 < p_2 \leq \infty$. Then, $L^{p_2}(E) \subseteq L^{p_1} (E)$. Furthermore, 
	\begin{align*}
		||f ||_{p_1} \leq c ||f||_{p_2} 	
	\end{align*}
	for all $f$ in $L^{p_2}(E)$, where $ c = [m(E)]^{\frac{p_2 - p_1}{q_1 p_2}}$ if $p_2 < \infty$ and $c = [m(E)]^{\frac{1}{p_1}}$ if $p_2 = \infty$. 
\end{corollary}

\subsection{$L^p$ is complete: the Reisz-Fischer Theorem}

\begin{definition}[Converge]
	A sequence $\{ f_n \}$ in a linear space $X$ normed by $|| \cdot ||$ is said to \textbf{converge to $f$ in $X$} provided: 
	\begin{align*}
		\lim_{n \rightarrow \infty} || f - f_n || = 0 	
	\end{align*}
\end{definition}

\begin{definition}[Cauchy]
	A sequence $\{ f_n \}$ in a linear space $X$ that is normed by $|| \cdot ||$ is said to be \textbf{Cauchy} in $X$ provided for each $\varepsilon > 0$, there exists a $N \in \bb{N}$ such that 
	\begin{align}
		|| f_n - f_m || < \varepsilon\ \forall\ m, n \geq N
	\end{align}
\end{definition}

\begin{definition}[Complete]
	A normed linear space $X$ is called \textbf{complete} if every Cauchy sequence in $X$ converges to a function in $X$. A complete normed linear space is called a \textbf{Banach space}. 
\end{definition}


\begin{prop}
	Let $X$ be a normed linear space. Then, every convergent sequence in $X$ is Cauchy. Moreover, a Cauchy sequence in $X$ converges if it has a convergent subsequence. 
\end{prop}

\begin{definition}
	Let $X$ be a linear space normed by $|| \cdot ||$. A sequence $\{ f_n \}$ in $X$ is said to be \textbf{rapidly Cauchy} if there is a convergent series of positive numbers $\sum_{k=1}^\infty \varepsilon_k$ for which 
	\begin{align*}
		|| f_{k+1} - f_k || \leq \varepsilon_k^2 	\text{ for all $k$} 
	\end{align*}
	
\end{definition}

\begin{prop}
	Let $X$ be a normed linear space. Then, every rapidly Cauchy sequence in $X$ is Cauchy. Furthermore, every Cauchy sequence has a rapidly Cauchy subsequence. 
\end{prop}

\begin{prop}
	Let $E$ be a measurable set and $1 \leq p \leq \infty$. Then, every rapidly Cauchy sequence in $L^p(E)$ converges with respect to the $L^p(E)$ norm and pointwise a.e. on $E$ to a function in $L^p(E)$. 
\end{prop}

\begin{theorem}[Riesz-Fischer Theorem] 
	Let $E$ be a measurable set and $1 \leq p \leq \infty$. Then $L^p(E)$ is a Banach space. Moreover, if $\{f_n \} \rightarrow f$ in $L^p(E)$, a subsequence of $\{ f_n \}$ converges pointwise a.e. on $E$ to $f$. 
\end{theorem}

\begin{theorem}
	Let $E$ be a measurable set and $1 \leq p < \infty$. Suppose $\{ f_n \}$ is a sequence in $L^p(E)$ that converges pointwise a.e. on $E$ to the function $f$ which belongs to $L^p(E)$. Then: 
	\begin{align*} 
		\{ f_n \} \rightarrow f \text{ in } L^p(E) \iff \lim_{n \rightarrow \infty } \int_E |f_n | ^p = \int_E |f|^p 	
	\end{align*}
\end{theorem}

\begin{definition}[Tight]
	A family $\mathcal{F}$ of measurable functions on $E$ is said to be \textbf{tight} over $E$ provided that for each $\varepsilon > 0$, there exists a subset $E_0$ of $E$ of finite measure for which 
	\begin{align*}
		\int_{E \setminus E_0 } |f| < \varepsilon \text{ for all } f \in \mathcal{F}	
	\end{align*}

\end{definition}
\begin{theorem}
	Let $E$ be a measurable set and let $1 \leq p < \infty$. Suppose $\{ f_n \}$ is a sequence in $L^p(E)$ that converges pointwise a.e. on $E$ to the function $f$ which belongs to $L^p(E)$. Then, $\{ f_n \} \rightarrow f$ in $L^p(E)$ $\iff$ $\{ |f_n|^p \}$ is uniformly integrable and tight over $E$. 
\end{theorem}

\subsection{Approximation and Separability}

\begin{definition}[Dense]
	Let $X$ be a normed linear space with norm $|| \cdot ||$. Given two subsets $\mathcal{F}$ and $\mathcal{G}$ of $X$ with $\mathcal{F} \subseteq \mathcal{G}$, we say that $\mathcal{F}$ is \textbf{dense} in $\mathcal{G}$ provided for each function $g$ in $\mathcal{G}$ and $\varepsilon > 0$, there is a function $f \in \mathcal{F}$ for which $||f - g || < \varepsilon$. 
\end{definition}

\begin{prop}
	Let $E$ be a measurable set and let $1 \leq p \leq \infty$. Then, the subspace of simple functions in $L^p(E)$ is dense in $L^p(E)$. 
\end{prop}

\begin{prop}
	Let $[a,b]$ be a closed, bounded interval and $1 \leq p < \infty$. Then, the subspace of step functions on $[a,b]$ is dense in $L^p[a,b]$. 
\end{prop}

\begin{definition}[Separable] A normed linear space $X$ is said to be \textbf{separable} provided there is a countable subset that is dense in $X$. 
\end{definition}

\begin{theorem}
	Let $E$ be a measurable set and $1 \leq p < \infty$. Then, the normed linear space $L^p(E)$ is separable. 
\end{theorem}

\begin{theorem}
	Suppose $E$ is measurable and let $1 \leq p < \infty$. Then, $C_c(E)$ (the set of all continuous functions with compact support on $E$) is dense in $L^p(E)$. 
\end{theorem}

\subsection{Results from the Homework}
\begin{enumerate}[noitemsep]
	\item (When Hölder's inequality $\rightarrow$ equality):  There is equality in Hölder's Inequality $\iff$ there exists constants $\alpha$, $\beta$, both of which non-zero, for which: 
	\begin{align*}
		\alpha |f|^p = \beta |g|^q 
	\end{align*}
	a.e. on $E$. 
	\item (Extension of Hölder's Inequality for 3 functions): Let $E \subseteq \R$ be measurable, let $1 \leq p < \infty$, $1 \leq q < \infty$, $1 \leq r < \infty$ such that: 
	\begin{align*}
		\frac{1}{p} + \frac{1}{q} + \frac{1}{r} = 1
	\end{align*}
	If $f \in L^p(E)$, $q \in L^q(E)$, and $h \in L^r(E)$, then $fgh \in L^(E)$ and: 
	\begin{align*}
		\idx{E}{}  |f g h | \leq ||f||_p ||g||_q ||h||_r 
	\end{align*}
	\item For $1 \leq p \leq \infty$, $q$ conjugate of $p$, $f \in L^p(E)$: 
	\begin{align*}
		||f||_p = \max_{g \in L^q(E), ||g||_q \leq 1} \idx{E}{} f g 
	\end{align*}
	\item ($L^p$ dominated convergence theorem): Let $\{ f_n \}$ be a sequence of measurable functions that converge pointwise a.e. on $E$ to $f$. For $1 \leq p < \infty$, suppose $\exists$ a function $g \in L^p(E)$ such that $\forall$ $n \in \N$, $|f_n| \leq g$ a.e. on $E$. Then, $\{ f_n \} \rightarrow f$ in $L^p(E)$. 
	\item Assume $1 \leq p < \infty$, if $E \subseteq \R$ has finite measure, $1 \leq p < \infty$, and $\{ f_n \}$ is a sequence of measurable functions which converge pointwise a.e. on $E$ to $f$, then $\{ f_n \} \rightarrow f$ in $L^p(E)$ if $\exists$ a $\theta > 0$ such that $\{ f_n \}$ belongs to and is \underline{bounded} as a subset of $L^{p+ \theta} (E)$. 
	\item The space $c$ of all convergent sequences of real numbers and the space $c_0$ of all sequences which converge to zero are Banach spaces with respect to the $\ell^\infty$ norm. 
	\item Let $E \subseteq \R$ be measurable, $1 \leq p \leq \infty$, $q$ the conjugate of $p$, and $\mathcal{S}$ a dense subset of $L^q(E)$. If $g \in L^p(E)$ and $\idx{E}{} g \cdot g =0$ for all $g \in \mathcal{S}$, then $g = 0$. 
	\item (Separability of $\ell^p$): For $1 \leq p < \infty$, $\ell^p$ is separable. $\ell^\infty$ is \underline{not} separable. 
\end{enumerate}

\section{$L^p$ Spaces: Duality and Weak Convergence}

\subsection{Riesz Representation Theorem for the Dual of $L^p$, $1 \leq p < \infty$}
\begin{definition}[Linear Functional]
	A \textbf{linear functional} on a linear space $X$ is a real-valued function $T$ on $X$ such that for $f$ and $g$ in $X$ and $\alpha$ and $\beta$ real numbers, 
	\begin{align}
		T(\alpha \cdot g + \beta \cdot h ) = \alpha \cdot T(g) + \beta \cdot T(h) 
	\end{align}
\end{definition}

\begin{definition}[Bounded]
	For a normed linear space $X$, a linear functional $T$ on $X$ is said to be \textbf{bounded} provided there is an $M \geq 0$ for which 
	\begin{align}
		|T(f)| \leq M \cdot ||f|| \text{ for all } f \in X
	\end{align}
	The infimum of all such $M$ is called the \textbf{norm} of $T$ and is denoted by $||T||_*$. 
\end{definition}

\begin{prop}[Continuity Property of a Bounded Linear Functional] 
	Let $T$ be a bounded linear functional on the normed space $X$. Then, if $\{ f_n \} \rightarrow f$ in $X$, then $\{ T(f_n) \} \rightarrow \{ T(f) \} $. 
\end{prop}

\begin{prop}
	Let $X$ be a normed vector space. Then, the collection of bounded linear functionals on $X$ is a linear space which is normed by $|| \cdot ||_*$. This normed vector space is called the \textbf{dual space} of $X$, and is denoted by $X^*$. 
\end{prop}

\begin{prop}
	Let $E \subseteq \R$ be measurable, $1 \leq p < \infty$, q the conjugate of $p$, $g \in L^q(E)$. Define the functional $T$ on $L^p(E)$ by: 
	\begin{align}
		T(f) := \int_E g \cdot f \text{ 		} \forall f \in L^p(E) 
	\end{align}
	Then, $T$ is a bounded linear functional on $L^p(E)$ and $||T||_* = ||g||_q$. 
\end{prop}

\begin{prop}
	Let $T$, $S$ be bounded linear functionals on the normed vector space $X$. If $T=S$ on a dense subset $X_0$ of $X$, then $T=S$. 
\end{prop}

\begin{lemma}
	\EinR, $1 \leq p < \infty$. Suppose that $g$ is integrable over $E$ and there exists a $M \geq 0$ for which 
	\begin{align*}
		\left| 		\int_E g \cdot f \right| \leq M || f||_p 	\text{ 		} \forall f \in L^p(E),\ f \text{ simple} 
	\end{align*}
	Then, $g \in L^q(E)$, where $q$ is the conjugate of $p$. Moreover, $||g||_q \leq M$. 
\end{lemma}

\begin{theorem}
	Let $[a,b]$ be a closed, bounded interval, and $1 \leq p < \infty$. Suppose that $T$ is a bounded linear functional on $L^p[a,b]$. Then, there is a functional $g \in L^q[a,b]$, where $q$ is the conjugate of $p$, for which: 
	\begin{align}
		T(f)= \idx{a}{b} g \cdot f \text{ 	} \forall f \in L^p[a,b] 
	\end{align}
\end{theorem}

\begin{theorem}[Riesz-Representation Theorem for the Dual of $L^p(E)$] 
	\EinR, $1 \leq p < \infty$, and q the conjugate of p. For all $g \in L^q(E)$, define the bounded linear functional $\mathcal{R}_g$ on $L^p(E)$ by: 
	\begin{align}
		\mathcal{R}_g := \idx{E}{} g \cdot f \text{ 		} \forall f \in L^p(E) 
	\end{align}
	Then, for each bounded linear functional $T$ on $L^p(E)$, there exists a unique $g \in L^q(E)$ for which 
	\begin{enumerate}[noitemsep]
		\item $\mathcal{R}_g = T$ and 
		\item $||T||_* = ||g||_q$
	\end{enumerate}
\end{theorem}

\subsection{Weak Sequential Convergence in $L^p$}

\begin{definition}[Converge Weakly]
	Let $X$ be a normed vector space. A sequence $\{ f_n \}$ in $X$ is said to \textbf{converge weakly} in $X$ to $f$ provided that 
	\begin{align}
		\lim_{n \rightarrow \infty} T(f_n) = T(f) \text{ 		} \forall T \in X^* 
	\end{align}
	we write 
	\begin{align*}
		\{ f_n \} \warrow f 
	\end{align*}
	to mean that $f$ and each $f_n$ belong to $X$ and $\{ f_n \}$ converges weakly in $X$ to $f$. 
\end{definition}

\begin{definition}
	\EinR, $1 \leq p < \infty$, $q$ the conjugate of $p$. Then, $\fcvw$ $\iff$ 
	\begin{align}
		\lim_{n \rightarrow \infty} \idx{E}{} g \cdot f_n = \idx{E}{} g \cdot f \text{ 		} \forall g \in L^q(E) 
	\end{align}
\end{definition}

\begin{theorem}
	\EinR, $1 \leq p < \infty$. Suppose that $\fcvw$. Then: 
	\begin{align*}
		\text{ $\{f_n\}$ is bounded and } ||f||_p \leq \liminf ||f_n||_p 	
	\end{align*}
\end{theorem}

\begin{corollary}
	\EinR, $1 \leq p < \infty$, $q$ the conjugate of $p$. Suppose $\{f_n\}$ converges weakly to $f$ in $L^p(E)$ and $\{ g_n \}$ converges strongly to $g \in L^q(E)$. Then: 
	\begin{align}
		\lim_{n \rightarrow \infty} \idx{E}{} g_n \cdot f_n = \idx{E}{} g \cdot f
	\end{align}
\end{corollary}

\begin{definition}[Linear Span] 
	Let $X$ be a normed vector space, and let $S \subseteq X$. Then, the \textbf{linear span of $S$} is the vector space consisting of all linear functionals of the form: 
	\begin{align}
		f = \sum_{k=1}^n \alpha_k \cdot f_k 
	\end{align}
	where each $\alpha_k \in \R$ and $f_k \in S$. It is the set of all \emph{finite linear combinations of elements in $S$}. We care about this since $L^p$ is an infinite dimensional  space, so we want to find a way to approximate it with finitely many elements. 
\end{definition}

\begin{prop}[Characterisation of Weak Convergence in $L^p(E)$] 
	\EinR, $1 \leq p < \infty$, q the conjugate of $P$. Assume that $\mathcal{F} \subseteq L^q(E)$ whose linear span is dense in $L^q(E)$. Let $\{f_n \}$ be a bounded sequence in $L^p(E)$, and let $f \in L^p(E)$. Then, $\fcvw$ $\iff$ 
	\begin{align}
		\lim_{n \rightarrow \infty} \idx{E}{} f_n \cdot g = \idx{E}{} f \cdot g \text{ 		} \forall g \in \mathcal{F} 
	\end{align}
\end{prop}

\begin{theorem}
	\EinR, $1 \leq p < \infty$. Suppose that $\{f_n \}$ is a bounded sequence in $L^p(E)$ and $f$ belongs to $L^p(E)$. Then, $\fcvw$ $\iff$ $\forall$ measurable sets $A \subseteq E$: 
	\begin{align}
		\lim_{n \rightarrow \infty} \idx{A}{} f_n = \idx{A}{} f
	\end{align}
	if $p > 1$, then it is sufficient to consider sets $A$ of finite measure. 
\end{theorem}

\begin{theorem}
	Let $[a,b]$ be a closed and bounded interval, $1 < p < \infty$. Suppose that $\{ f_n \}$ is a bounded sequence in $L^p[a,b]$ and $f \in L^p[a,b]$. Then, $\fcvw$ in $L^p[a,b]$ $\iff$ 
	\begin{align}
		\lim_{n \rightarrow \infty} \left[		\idx{a}{x} f_n	\right] = \idx{a}{x} f \text{ 			} \forall x \in [a,b] 
	\end{align}
\end{theorem}

\begin{lemma}[Riemann-Lebesgue Lemma; used in Fourier Series :-)] 
	Let $I= [-\pi, \pi]$, $1 \leq p < \infty$. $\forall n \in \mathbb{N}$, define $f_n(x):= \sin(nx)$ for $x \in I$. Then, $\{f_n\}$ converges weakly in $L^p(I)$ to $f \equiv 0$. 
\end{lemma}

\begin{theorem}
	\EinR, $1 < p < \infty$. Suppose that $\{ f_n \}$ is a bounded sequence in $L^p(E)$ that converges pointwise a.e. on $E$ to $f$. Then, $\fcvw$. 
\end{theorem}
This theorem was used in the proof but was not covered in Analysis 3: 
\begin{quote}
	\begin{theorem}[Vitali Convergence Theorem] 
		\EinR\  and of finite measure. Suppose that the sequence of functions $\{ f_n \}$ is uniformly integrable over $E$. Then, if $\{f_n \} \rightarrow f$ pointwise a.e. on $E$, then $f$ is integrable over $E$  and $\lim_{n \rightarrow \infty} \idx{E}{} f_n = f$. 
	\end{theorem}
\end{quote}

\begin{theorem}[Radon-Riesz Theorem]
	\EinR, $1 < p < \infty$. Suppose that $\fcvw$. Then: 
	\begin{align}
		\{f_n \} \rightarrow f \text{ in } L^p(E) \iff \lim_{n \rightarrow \infty} ||f_n||_p = ||f||_p 
	\end{align}
\end{theorem}

\begin{corollary} (Not Covered in Class): \EinR and $1 < p < \infty$. Suppose that $\fcvw$. Then, a subsequence of $\{f_n \}$ converges strongly to $f$ $\iff$ $||f||_p = \liminf||f_n||_p$. 
\end{corollary}

\subsection{Weak Sequential Compactness (``Compactness Found!'')} 

\begin{theorem}
	\EinR, $1 < p < \infty$. Then, every bounded sequence in $L^p(E)$ has a subsequence that converges weakly in $L^p(E)$ to a function in $L^p(E)$. 
\end{theorem}

\begin{theorem}[Helly's Theorem]
	Let $X$ be a \emph{SEPARABLE} normed vector space and $\{ T_n \}$ a sequence in its dual space $X^*$ that is bounded; that is, $\exists$ a $M > 0$ for which 
	\begin{align*}
		|T_n(f)| \leq M \cdot ||f|| \text{ 		} \forall f \in X,\ \forall n \in \mathbb{N} 	
	\end{align*}
	Then, there is a subsequence $\{ T_{n_k} \}$ of $\{ T_n \}$ and $T \in X^*$ for which 
	\begin{align}
		\lim_{k \rightarrow \infty} T_{n_k} (f) = T(f) \text{ 		} \forall f \in X 
	\end{align}
\end{theorem}

\begin{definition}[Weakly Sequentially Compact (Compact in the ``weak topology''] Let $X$ be a normed vector space. Then, a subset $K \subseteq X$ is \textbf{weakly sequentially compact} in $X$ provided that every sequence $\{ f_n \}$  in $K$ has a subsequence that converges weakly to $f \in K$. 
\end{definition}

\begin{theorem}[The Unit Ball is Weakly Sequentially Compact] \EinR, $1 < p < \infty$. Define: 
\begin{align}
	B_1:= \sets{f \in L^p(E)}{||f||_p \leq 1}. 
\end{align}
$B_1$ is weakly sequentially compact in $L^p(E)$. 

	
\end{theorem}

\subsection{Results from the Homework}
\begin{enumerate}
	\item (Reisz-Representation Theorem for the Dual of $\ell^p$): Let $1 \leq p < \infty$, $q$ the conjugate of $p$. Then for all $\{ g_n \} \in \ell^q$, define the bounded linear functional $\mathcal{R}_g$ on $\ell^p$ by: 
	\begin{align}
		\mathcal{R}_g := T ( \{ f_n \}) = \sum_{n=1}^\infty g_n f_n 	
	\end{align}
 	$\forall$ $\{ f_n \} \in \ell^p$. Then, for each bounded linear functional $T$ on $\ell^p$, there exists a unique $\{ g_n \} \in \ell^q$ for which: 
 	\begin{enumerate}[noitemsep]
 		\item $\mathcal{R}_g = T$
 		\item $||T||_* = || \{ g_n \} ||_q$ 
 	\end{enumerate}
 	\item Let $c$ be the vector space of all real sequences that converge to a real number and let $c_0$ be the subspace of $c$ comprising of all sequences that converge to zero. Norm each vector space with the $\ell^\infty$ norm. Then, $c^* = \ell^1$ and $c_0^* = \ell^1$. 
 	\item Assume that $h$ is a continuous function defined on all of $\R$ that is periodic with period $T$ and $\idx{0}{T}h =0$. Let $[a,b]$ be a closed + bounded interval. For each $n \in \N$, define $f_n(x) := h(nx)$. Define $f \equiv 0$ on $[a,b]$. Then, $\{ f_n \} $ converges weakly to $f$ in $L^p[a,b]$. 
 	\item Let $1 < p < \infty$, assume $f_0 \in L^p( \R)$. For each $n \in \N$, define $f_n(x) := f_0 (x-n)$. Define $f \equiv 0$ on $\R$. Then, $\{ f_n \}$ converges weakly to $f$ in $L^p(\R)$. Not true for $p=1$!
 	\item For $1 \leq p < \infty$, for each $n \in \N$, let $e_n \in \ell^p$ be the standard basis sequence. If $p > 1$, then $\{ e_n \}$ converges weakly to zero in $\ell^p$, but no subsequence converges strongly to zero. $\{ e_n \}$ does not converge at all in $\ell^1$. 
 	\item (Uniform Boundedness Principle): Let $E \subseteq \R$ be measurable, $1 \leq p < \infty$, and $q$ the conjugate of $p$. Suppose that $\{ f_n \}$ is a sequence in $L^p(E)$ for which for each $g \in L^q(E)$, the sequence $\{ \idx{E}{} g f_n \}$ is bounded. Show that $\{ f_n \}$ is bounded in $L^p(E)$. 
 	\item $\{ x^n \}$ in $C[0,1]$ fails to have a strongly convergent subsequence. Suitably modify this to work in any $C[a,b]$ by: 
 	\begin{align*}
 		f_n := \left( \frac{x-a}{b-a} \right)^n 
 	\end{align*}
 	\item In $\ell^p$, $1 < p < \infty$, every bounded sequence in $\ell^p$ has a weakly convergent subsequence. 
 	\item Let $X$ be a normed vector space, and let $\{ T_n \}$ be a sequence in $X^*$ for which there exists an $M \geq 0$ such that $||T_n ||_* \leq M$ for all $n \in \N$. Let $\mathcal{S} \subseteq X$ be a dense subset such that $\{ T_n (g) \}$ is Cauchy for all $g \in \mathcal{S}$. Then: 
 	\begin{enumerate}[noitemsep]
 		\item $\{ T_n (g) \}$ is Cauchy for all $g \in X$. 
 		\item The limiting functional is linear and bounded. 
 	\end{enumerate}
 	\item Helly's theorem fails when $X = L^\infty [0,1]$. To see why, consider  a sequence of linear functionals induced by the Rademacher functions. 
\end{enumerate}

\section{Metric Spaces}
\textbf{This section was not covered in class, but since we have homework on this chapter I figured having this as a review from analysis 2 might be helpful. Also, there are a few terms/results that I don't think we covered in analysis 2.}

\subsection{Examples of Metric Spaces}

\begin{definition}[Metric Space]
	Let $X$ be a non-empty set. A function $\rho: X \times X \rightarrow \R$ is called a \textbf{metric} if $\forall$ $x, y \in X$: 
	\begin{enumerate}[noitemsep]
		\item $\rho(x,y) \geq 0$ 
		\item $\rho(x,y) = 0$ $\iff$ $x=y$ 
		\item $\rho(x,y) = \rho(y,x)$
		\item $\rho(x,z) \leq \rho(x,y) + \rho(y,z)$ (\textbf{Triangle Inequality)}. 
	\end{enumerate}
	A non-empty set together with a metric, denoted $(X, \rho)$ is called a \textbf{metric space}. 
\end{definition}

\begin{definition}[Discrete Metric]
	For any non-empty set $X$, the \textbf{discrete metric} $\rho$ is defined by setting $\rho(x,y) = 0$ if $x = y$ and $\rho(x,y) = 1$ if $x \neq y$. 
\end{definition}

\begin{definition}[Metric Subspace]
	For any metric space $(X, \rho)$, let $Y \subseteq X$ be non-empty. Then, the restriction of $\rho$ to $Y \times Y$ defines a metric on $Y$. We define this induced metric space as a \textbf{metric subspace}. 
\end{definition}

\begin{exmp}[Examples of metric spaces] The following are examples of metric spaces: 
\begin{enumerate}[noitemsep]
	\item Every non-empty subset of a Euclidean space. 
	\item $L^p(E)$, where $E \subseteq \R$ is a measurable set. 
	\item $C[a,b]$. 
\end{enumerate}
\end{exmp}

\begin{definition}[Product Metric] For metric spaces $(X_1, \rho_1)$ and $(X_2, \rho_2)$, we define the \textbf{product metric} $\tau$ on the cartesian product $X_1 \times X_2$ by setting, for $(x_1, x_2)$ and $(y_1, y_2)$ in $X_1 \times X_2$:
	\begin{align}
		\tau ( (x_1, x_2), (y_1, y_2) ) := \{ 	[\rho_1(x_1, x_2)]^2 + [\rho_2(y_1, y_2)]^2	\}^{1/2} 
	\end{align}
\end{definition}

\begin{definition}
	Two metrics $\rho$ and $\sigma$ on a set $X$ are said to be \textbf{equivalent} if there are positive numbers $c_1$ and $c_2$ such that $\forall$ $x_1, x_2 \in X$, 
	\begin{align*}
		c_1 \sigma (x_1, x_2) \leq \rho(x_1, x_2) \leq c_2 \sigma(x_1, x_2) 	
	\end{align*}
\end{definition}

\begin{definition}[Isometry] 
	A mapping $f: (X, \rho) \rightarrow (Y, \sigma)$ between two metric spaces is called an \textbf{isometry} provided that $f$ is surjective and $\forall x_1, x_2 \in X$: 
	\begin{align}
		\sigma(f(x_1), f(x_2)) = \rho(x_1, x_2) 
	\end{align}
	We say that two metric spaces are \textbf{isometric} if there is an isometry from one to another. 
\end{definition}

\subsection{Open Sets, Closed Sets, and Convergent Sequences}

\begin{definition}[Open Ball]
	Let $(X, \rho)$ be a metric space. For a point $x \in X$ and $r>0$, the set: 
	\begin{align}
		B(x,r) := \sets{x' \in X}{\rho(x', x) < r}
	\end{align}
	is called the \textbf{open ball} centred at $x$ of radius $r$. A subset $\open \subseteq X$ is said to be \textbf{open} if $\forall x \in \open$, there exists an open ball centred at $x$ and contained in $\open$. For a point $x \in X$, an open set containing $x$ is called a \textbf{neighbourhood} of $x$.  
\end{definition}

\begin{prop}
	Let $X$ be a metric space. The whole set $X$ and the empty set $\emptyset$ are open. The intersection of any two open sets is open. The union of any collection of open sets is open. 
\end{prop}

\begin{prop}
	Let $X$ be a subspace of a metric space $Y$ and $E \subseteq X$. Then, E is \textbf{open in $X$} $\iff$ $E = X \cap \open$, where $\open$ is open in $Y$. 
\end{prop}

\begin{definition}[Closure]
	For a subset $E \subseteq X$, a point $x \in X$ is called a \textbf{point of closure} of $E$ provided that every neighbourhood of $x$ contains a point in $E$. The collection of the points of closure of $E$ is called the \textbf{closure} of $E$ and is denoted by $\overline{E}$. 
\end{definition}

\begin{prop}
	For $E \subseteq X$, where $X$ is a metric space, its closure $\overline{E}$ is closed. Moreover, $\overline{E}$ is the smallest closed subset of $X$ containing $E$ in the sense that if $F$ is closed and if $E \subseteq F$, then $\overline{E} \subseteq F$. 
\end{prop}

\begin{definition}[Converge]
	A sequence $\{ x_n \}$ in a metric space $(X, \rho)$ is said to \textbf{converge} to the point $x \in x$ provided that: 
	\begin{align*}
		\lim_{n \rightarrow \infty} \rho(x_n, x) = 0 	
	\end{align*}
	that is, $\forall$ $\varepsilon > 0$, $\exists$ an index $N$ such that $\forall n \geq N$, $\rho(x_n, x) < \varepsilon$. 
\end{definition}

\begin{prop}
	Let $\rho$ and $\sigma$ be equivalent metrics on a non-empty set $X$. Then, a subset $X$ is open in a metric space $(X, \rho)$ $\iff$ it is open in $(X, \sigma)$. 
\end{prop}
	
\subsection{Continuous Mappings Between Metric Spaces}

\begin{definition}[Continuous]
	A mapping $f$ from a metric space $X$ to a metric space $Y$ is continuous at the point $x \in X$ if $\forall$ $\{ x_n \} \in X$, if $\{ x_n \} \rightarrow x$, then $\{ f(x_n) \} \rightarrow f(x)$. $f$ is said to be \textbf{continuous} if it is continuous at every point in $X$. 
\end{definition}

\begin{prop}[$\varepsilon$-$\delta$ criteria for continuity] 
	A mapping from a metric space $(X, \rho)$ to a metric $(Y, \sigma)$ is continuous at the point $x \in X$ $\iff$ $\forall$ $\varepsilon > 0$, $\exists$ $\delta > 0$ such that if $\rho(x, x') < \delta$, then $\sigma(f(x), f(x')) < \varepsilon$. That is: 
	\begin{align}
		f(B(x, \delta)) \subseteq B(f(x), \varepsilon) 
	\end{align}
\end{prop}

\begin{prop}
	A mapping $f$ from a metric space $X$ to a metric space $Y$ is continuous $\iff$ $\forall$ open subsets $\open \subseteq Y$, the inverse image under $f$ of $\open$, $f^{-1}(\open)$, is an open subset of $X$. 
\end{prop}

\begin{prop}
	The composition of continuous mappings between metric spaces, when defined, is continuous. 
\end{prop}

\begin{definition}[Uniformly Continuous] 
	A mapping from a metric space $(X, \rho)$ to a metric space $(Y, \sigma)$ is said to be \textbf{uniformly continuous} if $\forall$ $\varepsilon > 0$, $\exists$ $\delta > 0$ such that $\forall u, v \in X$, if $\rho(u, v) < \delta$, $\sigma(f(u), f(v)) < \varepsilon$. 
\end{definition}

\begin{definition}[Lipschitz] 
	A mapping $f: (X, \rho) \rightarrow (Y, \sigma)$ is said to be \textbf{Lipschitz} if $\exists$ a $c \geq 0$ such that $\forall$ $u, v \in X$: 
	\begin{align*}
		\sigma(f (u), f(v) ) \leq c \rho(u,v) 	
	\end{align*}
\end{definition}

\subsection{Complete Metric Spaces} 

\begin{definition}[Cauchy] A sequence $\{ x_n \}$ in a metric space $(X, \rho)$ is said to be a \textbf{Cauchy sequence} if $\forall \varepsilon > 0$, there exists a $N \in \mathbb{N}$ such that if $m, n \geq N$, then $\rho(x_n, x_m) < \varepsilon$. 
\end{definition}

\begin{definition}[Complete]
	A metric space $X$ is said to be \textbf{complete} if every Cauchy sequence in $X$ converges to a point in $X$. 
\end{definition}

\begin{prop}
	Let $[a,b]$ be a closed and bounded interval of real numbers. Then, $C[a,b]$ with the metric induced by the max norm is complete. 
\end{prop}

\begin{prop}[Characterisation of Complete Subspaces of Metric Spaces]
	Let $E \subseteq X$, where $X$ is a complete metric space. Then, the metric subspace $E$ is complete $\iff$ $E$ is a closed subset of $X$. 
\end{prop}

\begin{theorem}
	The following are complete metric spaces: 
	\begin{enumerate}[noitemsep]
		\item Every non-empty closed subset of $\R^n$. 
		\item $E \subseteq \R$ measurable, $1 \leq p \leq \infty$, each non-empty closed subset of $L^p(E)$. 
		\item Each non-empty closed subset of $C[a,b]$. 
	\end{enumerate}
\end{theorem}

\begin{definition}[Diameter]
	Let $E$ be a non-empty subset of a metric space $(X, \rho)$. We define the \textbf{diameter} of $E$, denoted by diam$(E)$, by: 
	\begin{align}
		\text{diam}(E):= \sup \sets{\rho(x,y)}{x, y \in E}
	\end{align}
	We say that $E$ is \textbf{bounded} if it has finite diameter. 
\end{definition}

\begin{definition}[Contracting Sequence]
	A decreasing sequence $\{ E_n \}$ of non-empty subsets of $X$ is called a \textbf{contracting sequence} if: 
	\begin{align}
		\lim_{n \rightarrow \infty} \text{diam}(E_n) = 0 
	\end{align}
\end{definition}

\begin{theorem}[Cantor Intersection Theorem]
	Let $X$ be a metric space. Then, $X$ is complete $\iff$ whenever $\{ F_n \}$ is a contracting sequence of non-empty closed subsets of $X$, there is a point $x \in X$ for which: 
	\begin{align}
		\bigcap_{n=1}^\infty F_n = \{ x \}
	\end{align}
\end{theorem}

\begin{theorem}
	Let $(X, \rho)$ be a metric space. Then, there is a complete metric space $(\widetilde{X}, \tilde{\rho})$ for which $X$ is a dense subset of $\widetilde{X}$ and 
	\begin{align}
		\rho(u,v) = \tilde{\rho}(u,v) \text{ 		} \forall\ u, v \in X
	\end{align}
	we call such a space the \textbf{completion} of $(X, \rho)$. 
\end{theorem}

\subsection{Compact Metric Spaces}

\begin{definition}[Compact Metric Space]
	A metric space $X$ is called \textbf{compact} if every open cover of $X$ has a finite sub-cover. A subset $K \subseteq X$ is compact if $K$, considered as a metric subspace of $X$, is compact. 
\end{definition}

\underline{\textbf{Formulation of compactness in terms of closed sets:}} Let $\mathcal{T}$ be a collection of open subsets of a metric space $X$. Define $\mathcal{F}$ to be the collection of the complements of elements in $\mathcal{T}$. Since the elements of $\mathcal{T}$ are open, the elements of $\mathcal{F}$ are closed. Thus, $\mathcal{T}$ is a cover $\iff$ the elements of $\mathcal{F}$ have \emph{empty intersection}. By deMorgan's law, we can formulate compactness in terms of closed sets as: 
\begin{quote}
	A metric space $X$ is compact $\iff$ every collection of closed sets with empty intersection has a finite sub-collection whose intersection is non-empty. 
\end{quote}
	This property is called the \textbf{finite intersection property}.
	
\begin{definition}[Finite Intersection Property]
	A collection of sets $\mathcal{F}$ is said to have the \textbf{finite intersection property} if any finite sub-collection of $\mathcal{F}$ has a non-empty intersection.
\end{definition}

\begin{prop}[Compactness in terms of closed sets]
	A metric space $X$ is compact $\iff$ every collection $\mathcal{F}$ of closed subsets of $X$ with the finite intersection property has a non-empty intersection. 
\end{prop}

\begin{definition}[Totally Bounded]
	A metric space $X$ is \textbf{totally bounded} if $\forall$ $\varepsilon > 0$,  the space $X$ can be covered by a finite number of open balls of radius $\varepsilon$. A subset $E \subseteq X$ is said to be \textbf{totally bounded} if $E$, a s a subspace of the metric space $X$, is totally bounded. 
\end{definition}

\begin{definition}[$\varepsilon$-net] 
	Let $E$ be a subset of a metric space $X$. A $\varepsilon$-\textbf{net} for $R$ is a finite collection of open balls $\{ B(x_k, \varepsilon) \}_{k=1}^n$ with centres $x_k \in X$ whose union covers $E$. 
\end{definition}

\begin{prop}
	A metric space $E$ is totally bounded $\iff$ $\forall$ $\varepsilon > 0$, there is a finite $\varepsilon$-net for $E$. 
\end{prop}

\begin{prop}
	A subset of Euclidean space $\R^n$ is bounded $\iff$ it is totally bounded. 
\end{prop}

\begin{definition}[Sequentially Compact]
	A metric space $X$ is \textbf{sequentially compact} if every sequence in $X$ has a subsequence that converges to a point in $X$. 
\end{definition}

\begin{theorem}[Characterisation of Compactness for a metric space]. Let $X$ be a metric space. Then, TFAE: 
\begin{enumerate}[noitemsep]
	\item $X$ is complete and totally bounded. 
	\item $X$ is compact. 
	\item $X$ is sequentially compact. 
\end{enumerate}
\end{theorem}
	The following three propositions of this chapter are just breaking down these equivalences, so I will not write them.
	
\begin{theorem}
	Let $K \subseteq \mathbb{R}^n$. Then, TFAE: 
	\begin{enumerate}[noitemsep]
		\item $K$ is closed and bounded. 
		\item $K$ is compact. 
		\item $K$ is sequentially compact. 
	\end{enumerate}
	\textbf{Observe}: The equivalence $(1) \iff (2)$ is the Heine-Borel theorem. The  equivalence $(2) \iff (3)$ is the Bolzano-Weierstrass theorem.  
\end{theorem} 

\begin{prop}
	Let $f$ be a continuous mapping from a compact metric space $X$ to a compact metric space $Y$. Then, its image $f(X)$ is compact. 
\end{prop}

\begin{theorem}[Extreme Value Theorem]
	Let $X$ be a metric space. Then, $X$ is compact $\iff$ every continuous real-valued function on $X$ attains a minimum and maximum value. 
\end{theorem}

\begin{definition}[Lebesgue Number]
	Let $X$ be a metric space, and let $\{ \open_\lambda \}_{\lambda \in \Lambda}$ be an open cover of $X$. Thus, each $x \in X$ is contained in a member of the cover, $\open_\lambda$. Since $\open_\lambda$ is open, $\exists$ $\varepsilon > 0$ such that: 
	\begin{align*}
		B(x, \varepsilon) \subseteq \open_\lambda 	
	\end{align*}
	In general, $\varepsilon$ on $X$, but for compact metric spaces we can get \emph{uniform control}. This $\varepsilon$ that uniformly works is called the \textbf{Lebesgue number} for the cover $\{ \open_\lambda \}_{\lambda \in \Lambda }$. 
\end{definition}

\begin{lemma}
	Let $\{ \open_\lambda \}_{\lambda \in \Lambda}$ be an open cover of a compact metric space $X$. Then, there is a number $\varepsilon > 0$ such that for each $x \in X$, the open ball $B(x, \varepsilon) $ is contained in some member of the cover. 
\end{lemma}

\begin{prop}
	A continuous mapping from a compact space $(X, \rho)$ to a metric space $(Y, \sigma)$ is uniformly continuous. 
\end{prop}

\subsection{Separable Metric Spaces}

\begin{definition}[Dense \& Separable] A subset $D $ of a metric space $X$ is \textbf{dense} in $X$ if every non-empty subset of $X$ contains a point of $D$. A metric space is \textbf{separable} if there is a countable subset of $X$ that is dense in $X$. 	
\end{definition}

The \textbf{Weierstrass Approximation Theorem} states that polynomials are dense in $C[a,b]$. So, $C[a,b]$ is separable, with the countable dense set being the set of polynomials with rational coefficients. 

\begin{prop}
	A compact metric space is separable. 
\end{prop}

\begin{prop}
	A metric space $X$ is separable $\iff$ there is a countable collection of $\{ \open_n \}$ of open subsets of $X$ such that any open subset of $X$ is the union of a sub-collection of $\{ \open_n \}$. 
\end{prop}

\begin{prop}
	Every subspace of a separable metric space is separable. 
\end{prop}

\begin{theorem}
	Each of the following are separable metric spaces: 
	\begin{enumerate}[noitemsep]
		\item Every non-empty subset of Euclidean space $\R^n$. 
		\item $1 \leq p < \infty$, $L^p(E)$ and all non-empty subsets of $L^p(E)$. 
		\item Each non-empty subset of $C[a,b]$. 
	\end{enumerate}
\end{theorem}


\subsection{Results from the Homework}

\begin{enumerate}[noitemsep]
	\item $\{ (X_n, \rho_n) \}_{n=1}^\infty$ a countable collection of metric spaces. Then, the following is a metric on the Cartesian product: 
	\begin{align*}
		\rho_*(x,y) = \sum_{n=1}^\infty \frac{1}{2^n} \cdot \frac{\rho_n(x_n, y_n)}{1 + \rho_n(x_n + y_n)}
	\end{align*}
	\item A continuous mapping between metric spaces remains continuous if an equivalent metric is imposed on the domain and an equivalent metric is imposed on the domain. 
	\item The distance function (from a point to a set) is continuous. 
	\item $\{ x \in X\ |\ \text{dist}(x,E) = 0 \} = \overline{E}$. 
	\item (Sequential Definition of Uniform Continuity): For a mapping $f$ of a metric space $(X, \rho)$ to the metric space $(Y, \sigma)$, $f$ is uniformly continuous $\iff$ for all sequences $\{ u_n \}$ and $\{ v_n \}$ in $X$: 
	\begin{align*}
		\text{ if } \lim_{n \rightarrow \infty} \rho(u_n, v_n) = 0 \text{ then } \lim_{n \rightarrow \infty} \sigma (f(u_n), f(v_n)) = 0 
	\end{align*}
	\item If $X$ and $Y$ are metric spaces, with $Y$ complete, and $f$ a uniformly continuous mapping from $E \subseteq X \rightarrow Y$, then $f$ has a uniquely uniformly continuous extension mapping $\overline{f}$ of $\overline{E}$ to $Y$. 
	\item Let $E \subseteq X$, $X$ a compact metric space. Then, the metric subspace $E$ is compact $\iff$ $E$ is a closed subset of $X$. 
	\item $E \subseteq X$, $X$ complete. Then, $E$ is totally bounded $\iff$ $\overline{E}$ is totally bounded. 
	\item The closed unit ball in $\ell^2$ is not compact. 
\end{enumerate}


\section{Topological Spaces}
\subsection{Open Sets, Closed Sets, Bases, and Sub-bases}

\begin{definition}[Open Sets] 
	Let $X$ be a non-empty set. A \textbf{topology} $\topo$ for $X$ is a collection of subsets of $X$, called \textbf{open sets}, posessing the following properties: 
	\begin{enumerate}[noitemsep]
		\item The entire set $X$ and the empty set $\emptyset$ are open. 
		\item The finite intersection of open sets are open. 
		\item The union of any collection of open sets is open. 
	\end{enumerate}
	A non-empty set $X$, together with a topology on $X$, is called a \textbf{topological space}. For a point $x \in X$, an open set that contains $x$ is called a \textbf{neighbourhood} of $x$. 
\end{definition}

\begin{prop}
	A subset $E \subseteq X$ is open $\iff$ for each $x \in E$, there exists a neighbourhood of $x$ that is contained in $E$. 
\end{prop}

\begin{ex}[Metric Topology]
	Let $(X, \rho)$ be a metric space. Let $\open \subseteq X$ be  open if for all $x \in \open$, $\exists$ an open ball at $x$ that is contained in $\open$. This collection of open sets forms a topology; we call this the \textbf{metric topology} induced by $\rho$. 
\end{ex}

\begin{ex}[Discrete Topology] 
	This topology is ``too much.'' Let $X$ be a non-empty subset. Let $\topo := \mathcal{P}(X)$. Then, every set containing a point is a neighbourhood of that point. This is induced by the discrete metric. 
\end{ex}

\begin{ex}[Trivial Topology]
	Let $X$ be non-empty. Define $\topo := \{ X, \emptyset \}$. The only neighbourhood of any point is the whole set $X$. 	
\end{ex}

\begin{definition}[Topological Subspaces]
	Let $(X, \topo)$ be a topological space and let $E$ be a non-empty subset of $X$. The inherited topology $\mathcal{S}$ for $E$ is the set of all sets of the form $E \cap \topo$, where $\open \in \topo$. The topological space $(E, \mathcal{S})$ is called a \textbf{subspace} of $(X, \topo)$. 
\end{definition}

\begin{definition}[Base for the Topology]
	The building blocks of a topology is called a \textbf{base}. Let $(X, \topo)$ be a topological space. For a point $x \in X$, a collection of neighbourhoods of $x$, $B_x$, is called a \textbf{base for the topology at $X$} if $\forall$ neighbourhoods $\mathcal{U}$ of $x$, there exists a set $B$ in the collection $B_x$ for which $B \subseteq \mathcal{U}$. 
	
	A collection of open sets $\mathcal{B}$ is called a \textbf{base for the topology} $\topo$ provided it contains a base for the topology at each point. 
\end{definition}

\begin{center} 
\textbf{A base for a topology completely determines a topology, alongside $\emptyset$ and $X$.}
\end{center} 

\begin{prop}
	For a non-empty set $X$, let $\mathcal{B}$ be a collection of subsets of $X$. Then, $\mathcal{B}$ is a base for a topology for $X$ $\iff$: 
	\begin{enumerate}[noitemsep]
		\item $\mathcal{B}$ covers $X$. That is:
		\begin{align}
			X = \bigcup_{B \in \mathcal{B}} B
		\end{align}
		\item If $B_1, B_2 \in \mathcal{B}$, and $x \in B_1 \cap B_2$, then there is a set $B_3 \in \mathcal{B}$ for which $x \in B_3 \subseteq B_1 \cap B_2$. 
	\end{enumerate}
	The unique topology that has $\mathcal{B}$ as its base consists of $\emptyset$ and unions of sub-collections of $\mathcal{B}$. 
\end{prop}

\begin{definition}[Product Topology]
	Let $(X, \topo)$ and $(Y, \mathcal{S})$ be two topological spaces. In the cartesian product $X \times Y$, consider the collection of sets $\mathcal{B}$ containing the products $\open_1 \times \open_2$, where $\open_1$ is open in $X$ and $\open_2$ is open in $Y$. Then, $\mathcal{B}$ is a base for a topology on $X \times Y$, which we call the \textbf{product topology.}
\end{definition}

\begin{definition}[Sub-base]
	Let $(X, \topo)$ be a topological space. The collection of $\mathcal{S}$ of $\topo$ that covers $X$ is called a \textbf{sub-base} for the topology $\topo$ provided intersections of finite collections of $\mathcal{S}$ are a base for $\topo$. 
\end{definition}

\begin{definition}[Closure]
	Let $E \subseteq X$ be a subset of a topological space. A point $x \in E$ is called a \textbf{point of closure} of $E$ if every neighbourhood of $x$ contains a point in $E$. The collection of the points of closure of $E$ is called the \textbf{closure} of $E$, denoted $\overline{E}$. 
\end{definition}

\begin{prop}
	Let $X$ be a topological space, $E \subseteq X$. Then, $\overline{E}$ is closed. Moreover, $\overline{E}$ is the smallest closed subset of $X$ containing $E$ in the sense that if $F$ is closed and $E \subseteq F$, then $\overline{E} \subseteq F$. 
\end{prop}

\begin{prop}
	A subset of a topological space $X$ is open $\iff$ its complement is closed. 
\end{prop}

\begin{prop}
	Let $X$ be a topological space. Then, (a) $\emptyset$ and $X$ are closed, (b) the union of a finite collection of closed sets is closed, (c) the intersection of any collection of closed sets in $X$ is closed. 
\end{prop}

\subsection{Separation Properties}
\textbf{Motivation:} Separation properties for a topology allow us to discriminate between which topologies discriminate between certain disjoint pairs of sets, which will then allow us to study a robust collection of cts real-valued functions on $X$. 

\begin{definition}[Neighbourhood]
	A \textbf{neighbourhood} of $K$ for a subset $K \subseteq X$ is an open set that contains $K$. 
\end{definition} 

\begin{definition}[Separated by Neighbourhoods]
	We say that two disjoint sets $A$ and $B$ in $X$ can be separated by disjoint neighbourhoods provided that there exists neighbourhoods of $A$ and $B$, respectively, that are disjoint. 
\end{definition}

\begin{definition}[Separation Properties of Topological Spaces]. In the order of most general to least general, they are: 
\begin{enumerate}[noitemsep]
	\item \underline{\textbf{Tychonoff Separation Property}}: For each two points $u, v \in X$, there exists a neighbourhood of $u$ that does not contain $v$ and a neighbourhood of $v$ that does not contain $u$. 
	\item \underline{\textbf{Hausdorff Separation Property}}: Each two points in $X$ can be separated by disjoint neighbourhoods. 
	\item \underline{\textbf{Regular Separation Property}}: Tychonoff $+$ each closed set and a point not in the set can be separated by disjoint neighbourhoods.
	\item \underline{\textbf{Normal Separation Property}}: Tychonoff $+$ each two disjoint closed sets can be separated by disjoint neighbourhoods. 
\end{enumerate}
\end{definition}

\begin{prop}
	A topological space is Tychonoff $\iff$ every set containing a single point, $\{ x \}$, is closed. 
\end{prop}

\begin{prop}
	Every metric space is normal.
\end{prop}

\begin{lemma}
	$F$ is closed $\iff$ dist$(x, F) > 0$ $\forall$ $ x \notin F$. 
\end{lemma}

\begin{prop}
	Let $X$ be a Tychonoff topological space. Then, $X$ is normal $\iff$ whenever $\hood$ is a neighbourhood of a closed subset of $F$ of $X$, there is another neighbourhood of $F$ whose closure is contained in $\hood$. that is, there is an open set $\open$ for which: 
	\begin{align}
		F \subseteq \open \subseteq \overline{\open} \subseteq \hood 
	\end{align}
\end{prop}


\subsection{Countability and Separability}
\begin{definition}[Converge, Limit]
	A sequence $\{ x_n \}$ in a topological space $X$ is said to \textbf{converge} to the point $x \in X$ if for each neighbourhood $\hood$ of $x$, there exists an index $N \in \mathbb{N}$ such that if $n \geq N$, then $x_n$ belongs to $\hood$. This point is called a \textbf{limit} of the sequence. 
\end{definition}

\begin{definition}[First and Second Countable]
	A topological space $X$ is \textbf{first countable} if there is a countable base at each point. A space $X$ is said to be \textbf{second countable} if there is a countable base for the topology.
\end{definition}

\begin{ex}
Every metric space is first countable. 	
\end{ex}

\begin{prop}
	Let $X$ be a first countable topological space. For a subset $E \subseteq X$, a point $x \in X$ is called a point of closure of $E$ $\iff$ it is a limit of a sequence in $E$. Thus, a subset $E$ of $X$ is closed $\iff$ whenever a sequence in $E$ converges to $x \in X$, we have that $x \in E$. 
\end{prop}

\begin{definition}[Dense/Separable]
	A subset $E \subseteq X$ is \textbf{dense} in $X$ if every open set in $X$ contains a point of $E$. We call $X$ \textbf{separable} if it has a countable dense subset. 
\end{definition}

\begin{definition}[Metrisable]
	A topological space $X$ is said to be \textbf{metrisable} if the topology is induced by the metric. 
\end{definition}

\begin{theorem}
	Let $X$ be a second countable topological space. Then, $X$ is metrisable $\iff$ it is normal. 
\end{theorem}


\subsection{Continuous Mappings between Topological Spaces}
\begin{definition}[Continuous]
	For topological spaces $(X, \topo)$, $(Y, \mathcal{S})$, a mapping $f: X \rightarrow Y$ is said to be \textbf{continuous} at the point $x_0$ in $X$ if, for every neighbourhood $\open$ if $f(x_0)$, there is a neighbourhood $\hood$ of $x_0$ for which $f(\hood) \subseteq \open$. We say that $f$ is continuous provided it is continuous at each point in $X$.  
\end{definition}

\begin{prop}
	A mapping $f: X \rightarrow Y$ between topological spaces $X$ and $Y$ is continuous $\iff$ for any open subset $\open$ in $Y$, its inverse image under $f$, $f^{-1}(\open)$, is an open subset of $X$. 
\end{prop}

\begin{prop}
	The composition of continuous mappings between topological spaces, when defined, is continuous. 
\end{prop}

\begin{definition}[Stronger]
	Given two topologies $\topo_1$ and $\topo_2$ for a set $X$, if $\topo_2 \subseteq \topo_1$, then we say that $\topo_2$ is \textbf{weaker} than $\topo_1$, and that $\topo_1$ is \textbf{stronger} than $\topo_2$. 
\end{definition}

\begin{prop}
	Let $X$ be a non-empty set and let $\mathcal{S}$ be a collection of subsets of $X$ that covers $X$. The collection of subsets of $X$ consisting of intersections of finite collections of $\mathcal{S}$ is a base for a topology $\topo$ of $X$. It is the weakest topology containing $\mathcal{S}$ in the sense that if $\topo'$ is any other topology for $X$ containing $\mathcal{S}$, then $\topo \subseteq \topo'$. 
\end{prop}

\begin{definition}[Weak Topology]
	Let $X$ be a non-empty set and $\mathcal{F} := \{ f_\alpha\ |\ X \rightarrow X_\alpha \}_{\alpha \in \Lambda }$ a collection of mappings, where each $X_\alpha$ is a topological space. The weakest topology for $X$ that contains the collection of sets
	\begin{align}
		\{ f_\alpha^{-1} ( \open_\alpha)\ |\ f_\alpha \in \mathcal{F},\ \open_\alpha \text{ open in } X_\alpha \} 
	\end{align}
	is called the \textbf{weak topology for $X$ induced by $\mathcal{F}$.}
\end{definition}

\begin{prop} Let $X$ be a non-empty set, $\mathcal{F} := \{ f_\lambda\ |\ X \rightarrow X_\lambda \}_{\lambda \in \Lambda } $ a collection of mappings where each $X_\lambda$ is a topological space. The weak topology for $X$ induced by $\mathcal{F}$ is the topology on $X$ that has the fewest number of sets covering the topologies on $X$ for which each mapping $f_\lambda: X \rightarrow X_\lambda$ is continuous. 
\end{prop}

\begin{definition}[Homeomorphism]
	A mapping from a topological space $X \rightarrow Y$ is said to be a \textbf{homeomorphism} if it is bijective and has a continuous inverse $f^{-1}: Y \rightarrow X$. Two topological spaces are said to be \textbf{homeomorphic} if there exists a homeomorphism between them. The notion of homeomorphism induces a notion of an equivalence relation between spaces. 
\end{definition}

\subsection{Compact Topological Spaces}

\begin{definition}[Cover]
	A collection of sets $\{ E_\lambda \}_{\lambda \in \Lambda }$ is said to be a \textbf{cover} of a set $E$ if $E \subseteq \bigcup_{\lambda \in \Lambda } E_\lambda$. 
\end{definition}

\begin{definition}[Compact]
	A topological space $X$ is said to be \textbf{compact} if every open cover of $X$ has a finite sub-cover. A subset $K \subseteq X$ is compact if $K$, considered as a topological space with the subspace topology inherited from $X$, is compact. 
\end{definition}

\begin{prop}
	A topological space $X$ is compact $\iff$ every collection of closed subsets of $X$ that posesses the finite intersection property has non-empty intersection. 
\end{prop}

\begin{prop}
	A closed subset $K$ of a compact topological space is compact. 
\end{prop}

\begin{prop}
	A compact subspace $K$ of a Hausdorff topological space is a closed subset of $X$. 
\end{prop}

\begin{definition}[Sequentially Compact] 
	A topological space $X$ is said to be \textbf{sequentially compact} if every sequence in $X$ has a subsequence that converges to a point in $X$. 
\end{definition}

\begin{prop}
	Let $X$ be a second countable topological space. Then, $X$ is compact $\iff$ it is sequentially compact. 
\end{prop}

\begin{theorem}
	A compact Hausdorff space is normal. 
\end{theorem}

\begin{prop}
	A continuous one-to-one mapping $f$ of a compact space $X$ onto a Hausdorff space $Y$ is a homeomorphism. 
\end{prop}

\begin{prop}
	The continuous image of a compact topological space is compact. 
\end{prop}

\begin{corollary}
	A continuous real-valued function on a compact topological space takes on a minimum and maximum functional value. 
\end{corollary}

\begin{definition}[Countably Compact] 
	A topological space is \textbf{countably compact} if every countable open cover has a finite subcover.
\end{definition}

\subsection{Connected Topological Space}

\begin{definition}[Separated]
	Two non-empty subsets of a topological space \textbf{separate} $X$ if they are disjoint and their union is $X$. 
\end{definition}

\begin{definition}[Connected]
	A topological space which cannot be separated by open sets is said to be \textbf{connected}. A subset $E \subseteq X$ is \textbf{connected} if there do NOT exist open subsets $\open_1$, $\open_2$ of $X$ for which: 
	\begin{align*}
		& \open_1 \cap E \neq \emptyset \\
		& \open_2 \cap E \neq \emptyset \\
		& E \subseteq \open_1 \cup \open_2, \\
		& E \cap \open_1 \cap \open_2 = \emptyset 	
	\end{align*}
\end{definition}

\begin{prop}
	Let $f$ be a continuous mapping of a connected space $X$ to a topological space $Y$. Then, its image $f(X)$ is connected. 
\end{prop}

\begin{prop}
	For A set $C \in \R$, the following are equivalent. 
	\begin{enumerate}[noitemsep]
		\item $C$ is an interval. 
		\item $C$ is convex. 
		\item $C$ is connected. 
	\end{enumerate}
\end{prop}

\begin{definition}[Intermediate Value Property]
	A topological space $X$ has the \textbf{intermediate value property} if the image of any continuous real-valued function on $X$ is an interval. 
\end{definition}

\begin{prop}
	A topological space has the intermediate value property $\iff$ it is connected. 
\end{prop}

\begin{definition}[Arcwise connected]
	A topological space $X$ is \textbf{arcwise connected} if, for each pair $u, v \in X$, there exists a continuous map $f: [0,1] \rightarrow X$ for which $f(0) = u$ and $f(1) = v$. Note: 
	\begin{enumerate}[noitemsep]
		\item Connected $\iff$ arcwise connected in $\R^n$. 
		\item Arcwise connected $\Rightarrow$ connected (in general) 
		\item There exist connected but non-arcwise connected spaces (in general). 
	\end{enumerate}
\end{definition}

\subsection{Results from Homework}
\begin{enumerate}[noitemsep]
	\item Let $X$ be a topological space. Then, $X$ is Hausdorff $\iff$ the diagonal $D:= \{ (x_1, x_2) \in X \times X\ |\ x_1 = x_2 \}$ is closed as a subset of $X \times X$. 
	\item The Moore plane is separable. The subspace $\R \times \{ 0 \}$ is not separable. Thus, the Moore plane is not metrisable and not second countable. 
	\item Let $X$ and $Y$ be topological spaces. Then, you can construct a continuous map from a Hausdorff space to a non-Hausdorff space, and you can do the same for a normal space to a non-normal space.
	\item If $\rho_1$ and $\rho_2$ are metrics on a set $X$ that induce topologies $\topo_1$ and $\topo_2$, respectively, then if they generate the same topology $\topo_1 = \topo_2$, then they are \underline{NOT} necessarily equivalent. A counter example would be: 
	\begin{align*}
		& \rho_1 := |x-y| \\
		& \rho_2 := \frac{|x-y|}{1 + |x-y|}
	\end{align*} 
\end{enumerate}

\section{Metric Spaces: Three Fundamental Theorems}

\subsection{The Arzela-Ascoli Theorem}

\begin{prop}
	Let $X$ be a compact metric space. Then $C(X)$ is complete. 
\end{prop}

\begin{definition}[Epicontinuous] 
	A collection $\mathcal{F}$ of real-valued functions on a metric space $X$ is said to be \dfn{epicontinuous} at the point $x \in X$ provided that $\forall \varepsilon >0$, $\exists$ $\delta >0$ such that $\forall f \in \mathcal{F}$, $x' \in X$: 
	\begin{align*}
		\text{ if } \rho (x', x) < \delta \text{ then } | f(x') - f(x) | < \varepsilon
	\end{align*}
	The collection $\mathcal{F}$ is said to be epicontinuous on $X$ if it is epicontinuous at every point in $X$. 
\end{definition}

\begin{itemize}[noitemsep]
	\item Any finite collection of continuous functions will be epicontinuous. 
	\item In general, infinite collections of epicontinuous functions are not epicontinuous. For example, consider $f_n := x^n$ where $x \in [0,1]$. This collection fails to be epicontinuous at $x=1$. 
\end{itemize}

\begin{definition}[Pointwise bounded] 
	A sequence $\{ f_n \}$ of real-valued functions on a set $X$ are said to be \dfn{pointwise bounded} if $\forall x \in X$, the sequence $\{ f_n \}$ is bounded. A sequence is \dfn{uniformly bounded} if $\exists M \geq 0$ for which 
	\begin{align*}
		| f_n | \leq M \text{ on } X \text{ for all } n \in \mathbb{N} 
	\end{align*}
\end{definition}

\begin{lemma}[The Arzela-Ascoli Lemma] 
	Let $X$ be a separable metric space. Let $\{ f_n \}$ be an equicontinuous sequence in $X$ that is pointwise bounded. Then a subsequence $\{ f_n \}$ converges pointwise on all of $X$ to a real-valued function $f$ on $X$. 
\end{lemma}

\begin{definition}[Uniformly Epicontinuous]
	Let $X$ be a compact metric space, $\mathcal{F}$ an epicontinuous collection of real-valued functions on $X$. Then, $\mathcal{F}$ is \dfn{uniformly equicontinuous} in the sense that $\forall$ $\varepsilon > 0$, $\exists$ $\delta >0$ such that $\forall $ $u, v \in X$, $\forall f \in \mathcal{F}$ if 
	\begin{align*}
		\rho (u,v) < \delta\ \Rightarrow |f(u) - f(v) | < \varepsilon
	\end{align*}
\end{definition}


\begin{theorem}[Arzela-Ascoli Theorem]
	Let $X$ be a compact metric space, $\{ f_n \}$ a uniformly bounded, equicontinuous sequence of real-valued functions on $X$. Then $\{ f_n \}$ has a subsequence that converges uniformly on $X$ to a continuous function $f$ on $X$. 
\end{theorem}

\begin{theorem}
	Let $X$ be a compact metric space and $\mathcal{F} \subseteq C(X)$. Then, $\mathcal{F}$ is a compact subspace of $C(X)$ $\iff$ $\mathcal{F}$ is closed, uniformly bounded, and epicontinuous. 
\end{theorem}

\subsection{Baire Category Theorem}

\begin{definition}[Hallow]
	A subset of a metric space $X$ is \dfn{hallow} if it has an empty interior. 
	\begin{itemize}[noitemsep]
		\item For $E \subseteq X$, $E$ is \dfn{hallow}  in $X$ $\iff$ its complement is dense in $X$. 
		\item Let $X$ be a metric space. Let $0 < r_1 < r_2$. Bu the continuity of the metric, $\overline{B(x, r_1)} \subseteq B(x, r_2) $. Thus, $\overline{B(x, r_1)}$ is a closed set for which the following holds: 
		\begin{align}
			B(x, r_1) \subseteq \overline{B(x, r_1) } \subseteq B(x, r_2) 	
		\end{align}

	\end{itemize}
\end{definition}

\begin{theorem}[Baire Category Theorem]
	Let $X$ be a complete metric space. 
	\begin{enumerate}[noitemsep]
		\item Let $\{ \open_n \}_{n=1}^\infty$ be a countable collection of open, dense subsets of $X$. Then, the intersection $\bigcap_{n=1}^\infty \open_n$ is also dense. 
		\item Let $\{ F_n \}_{n=1}^\infty$ be a countable collection of closed, hallow subsets of $X$. Then, their union $\bigcup_{n=1}^\infty F_n$ is also hallow. 
	\end{enumerate}
	Equivalent formulation:
	\begin{itemize}[noitemsep]
		\item In a complete metric space, the union of a countable collection of nowhere dense sets is hallow. 
	\end{itemize}
\end{theorem}

\begin{definition}[Nowhere Dense]
	A subset $E \subseteq X$, $X$ a metric space, is called \dfn{nowhere dense} provided that its closure $\overline{E}$ is hallow. We have the following equivalence: 
	\begin{itemize}[noitemsep]
		\item A subset $E \subseteq X$ is nowhere dense $\iff$ for each open subset $\open $ of $X$, $E \cap \open$ is not dense in $X$. 
	\end{itemize}
\end{definition}

\begin{corollary}
	Let $X$ be a complete metric space and $\{ F_n \}_{n=1}^\infty$ a countable collection of closed subsets of $X$. If $\cup_{n=1}^\infty F_n$ has a non-empty interior, then at least one of the $F_n$'s has a non-empty interior. In particular, if $X = \bigcup_{n=1}^\infty F_n$, then at least one of the $F_n's$ has empty interior. 
\end{corollary}


\begin{corollary}
	Let $X$ be a complete metric space and $\{ F_n \}_{n=1}^\infty$ a countable collection of closed subsets of $X$. Then $\bigcup_{n=1}^\infty \partial F_n$ is hallow. 
\end{corollary}

\begin{theorem}
	Let $\mathcal{F}$ be. family of continuous real-valued functions on a complete metric space $X$ that is pointwise bounded in the sense that $\forall x \in X$, $\exists$ a constant $M_x$ for which 
	\begin{align}
		|f(x) | \leq M_x \text{ } \forall f \in \mathcal{F} 	
	\end{align}
	Then, there is a non-empty open subset $\open$ of $X$ on which $\mathcal{F}$ is uniformly bounded in the sense that $\exists$ a constant $M$ such that 
	\begin{align}
		|f| \leq M \text{ on } \open\ \forall f \in \mathcal{F} 	
	\end{align}
\end{theorem}


\begin{theorem}
	Let $X$ be a complete metric space and let $\{ f_n \}$ be a sequence of continuous real-valued functions on $X$ that converges pointwise on $X$ to the real-valued function $f$. Then, there is a dense set $D \subseteq X$ for which $\{ f_n \}$ is epicontinuous and $f$ is continuous at each point in $D$. 
\end{theorem}

Some standard terminology: 
\begin{itemize}[noitemsep]
	\item \dfn{First Category/Meger}: a subset $E \subseteq X$ is of the \dfn{first category} if $E$ is the union of a countable collection of nowhere dense subsets of $X$. 
	\item \dfn{Second Category/Non-Meger} a set that is not of the second category. 
	\item \dfn{Residual}: the complement of a set of the first category. 
\end{itemize}
Equivalent formulation of the Baire Category Theorem: \emph{A non-empty set of a complete metric space is of the second category}

\subsection{The Banach Contraction Principle}

\begin{definition}[Fixed Point]
	A point $x \in X$ is called a \dfn{fixed point} of the mapping $T: X \rightarrow X$ provided that $T(x) = x$. 
\end{definition}

\begin{definition}[Convex]
	A subset $K \subseteq \R^n$ is said to be \dfn{convex} provided that whenever $u, v \in K$, the segment $\{ tu + (1-t) v\ |\ 0 \leq t \leq 1 \} \subseteq K$. 
\end{definition}

\begin{theorem}[Brouwer's Fixed Point Theorem]
	If $K \subseteq \R^n$ is a compact, convex subset of $\R^n$, and if the mapping $T: K \rightarrow K$ is continuous, then $T$ has a fixed point. 
\end{theorem}

\begin{definition}[Lipschitz] 
	A mapping $T$ from a metric space $(X, \rho)$ to itself is said to be \dfn{Lipschitz} provided that there is a number $c \geq 0$, called a Lipschitz constant for the mapping, for which 
	\begin{align}
		\rho(T(u), T(v) ) \leq c \rho (u,v)\ \forall u,v \in X	
	\end{align}
	If $c<1$, then the Lipschitz mapping is called a \dfn{contraction}
\end{definition}


\begin{theorem}[Banach Contraction Principle]
	Let $X$ be a complete metric space and the mapping $T: X \rightarrow X$ a contraction. Then, $T: X \rightarrow X$ has exactly one fixed point. 
\end{theorem}

\begin{theorem}[Picard Local Existence Theorem]
	Let $\open \subseteq \R^n$ be open, $(x_0, y_0) \in \open$. Let $g: \open \rightarrow \R$ be a function. Suppose we want to find an open interval of real numbers $I$ containing the point $x_0$ and a differentiable function $f: I \rightarrow \R$ such that
	\begin{align}
		\begin{cases}
			f'(x) = g(x, f(x))\ \forall x \in I \\
			f(x_0) = y_0 
		\end{cases}	
	\end{align}
	Suppose the function $g: \open \rightarrow \R$ is continuous and there is a positive number $M$ for which the following Lipschitz property holds: 
	\begin{align*}
		| g(x, y_1) - g(x, y_2) | \leq M |y_1 - y_2 |\ \forall (x, y_1), (x, y_2) \in \open
	\end{align*}
	Then, $\exists$ an open interval $I$ containing $x_0$ on which the ODE above has a unique solution. 
\end{theorem}

\section{Topological Spaces: Three Fundamental Properties}

\subsection{Ursohn's Lemma and the Tietze Extension Theorem}
\begin{lemma}[Urysohn's Lemma]
	Let $A$, $B$, be two non-empty, disjoint closed subsets of a normal space $X$. Then, for any closed, bounded interval of real numbers $[a,b]$, $\exists$ a continuous real-valued function $f$ defined on $X$ that takes values in $[a,b]$, while $f=A$ on $A$ and $f=b$ on $B$.  
\end{lemma}

\begin{definition}[Normally Ascending]
	Let $X$ be a topological space and $\Lambda$ a set of numbers. A collection of open subsets of $X$ indexed by $\Lambda$, $\{ \open_\lambda \}_{\lambda \in \Lambda} $, is said to be \dfn{normally ascending} provided that for any $\lambda_1 < \lambda_2$: 
	\begin{align*}
		\overline{\open_{\lambda_1}} \subseteq \open_{\lambda_2} 
	\end{align*}
\end{definition}

\begin{lemma}
	Let $X$ be a topological space. For $\Lambda$ a dense subset of the open, bounded interval $]a,b[$, let $\{ \open_{\lambda}\}_{\lambda \in \Lambda}$, be a normally ascending collection of open subsets of $X$. Define the function $f: X \rightarrow \R$ by setting $f=b$ on $X \setminus \bigcup_{\lambda \in \Lambda} \open_\lambda$ and otherwise setting: 
	\begin{align}
		f(x) := \inf \{ \lambda \in \Lambda\ |\ x \in \open_\lambda \} 	
	\end{align}
	Then, $f: X \rightarrow [a,b]$ is continuous. 
\end{lemma}

\begin{lemma}
	Let $X$ be a normal topological space, $F$ a closed subset of $X$, and $\mathcal{U}$ a neighbourhood of $F$. For any open, bounded interval $]a,b[$ there exists a dense subset $\Lambda$ of $]a,b[$ and a normally ascending collection of open subsets of $X$, $\{ \open_\lambda \}_{\lambda \in \Lambda }$ for which 
	\begin{align}
		F \subseteq \open_\lambda \subseteq \overline{\open}_\lambda \subseteq \mathcal{U} 	
	\end{align}
\end{lemma}

\begin{theorem}[Tietze Extension Theorem]
	Let $X$ be a normal topological space, $F$ a closed subset of $X$, and $f$ a continuous real-valued function on $F$ that takes values in the closed, bounded interval $[a,b]$. Then, $f$ has a continuous extension ot all of $X$ that also takes values in $[a,b]$. 
\end{theorem}

\begin{theorem}[Urysohn Metrization Theorem] 
	Let $X$ be a second countable topological space. Then $X$ is metrisable $\iff$ $X$ is normal. 
\end{theorem}


\subsection{The Tychnoff Product Theorem}

\begin{definition}[Product Topology]
	Let $\{ (X_\lambda, \topo_\lambda ) \}_{\lambda \in \Lambda} $ be a collection of topological spaces indexed by a set $\Lambda$. The \dfn{product topology} on the cartesian product $\prod_{\lambda \in \Lambda} \open_\lambda$, where $\open_\lambda \in \topo_\lambda$ and $\open_\lambda = X_\lambda$ except for finitely many $\lambda$. 
\end{definition}

\begin{prop}
	Let $X$ be a topological space. A sequence $\{ f_n\ | \Lambda \rightarrow X \}$ converges to $f$ in the product space $X^\Lambda$ $\iff$  $\{ f_n (\lambda) \}$ converges to $f(\lambda)$ for all $\lambda \in \Lambda$. Thus, convergence of a sequence wrt the product topology is pointwise convergence. 
\end{prop}

\begin{prop}
	The product topology on the cartesian product of topological spaces $\prod_{\lambda \in \Lambda} X_\lambda$ is the weak topology associated to the collection of projections $\{ \pi_\lambda\ |\ \prod_{\lambda \in \Lambda} X_\lambda \rightarrow X_\lambda \}_{\lambda \in \Lambda} $, that is, the topology on the cartesian product that has the fewest number of sets among the topologies for which all projection mappings are continuous. 
\end{prop}

\begin{lemma}
	Let $\mathcal{A}$ be a collection of subsets of a set $X$ that possesses the finite intersection property. Then, there exists a collection $\mathcal{B}$ of subsets of $X$ which contains $\mathcal{A}$, has the finite intersection property, and is maximal with respect to this property. 
\end{lemma}

\begin{lemma}
	Let $\mathcal{B}$ be a collection of subsets of $X$ of a set that is maximal with respect to the finite intersection property. Then, each intersection of a finite number of sets in $\mathcal{B}$ is again in $\mathcal{B}$, and each subset of $X$ that has non-empty intersection with each set in $\mathcal{B}$ is itself in $\mathcal{B}$. 
\end{lemma}

\begin{theorem}[Tychnoff Product Theorem]
	Let $\{ X_\lambda \}_{\lambda \in \Lambda}$ be a collection of compact topological spaces indexed by the set $\Lambda$. Then, the cartesian product $\prod_{\lambda \in \Lambda} X_\lambda$ with the product topology is compact. 
\end{theorem}
\subsection{The Stone-Weierstrass Theorem}
\begin{theorem}[Weierstrass Approximation Theorem]
	Let $f$ be a continuous real-valued function on a closed, bounded interval $[a,b]$. Then, $\forall$ $\varepsilon > 0$, $\exists$ a polynomial $p$ for which 
	\begin{align}
		|f(x) - p(x) | < \varepsilon\ \forall x \in [a,b] 	
	\end{align}

\end{theorem}

\begin{definition}[Algebra]
	A vector subspace $\mathcal{A} \leq C(X)$ is called an \dfn{algebra} provided that the product of any two functions in $\mathcal{A}$ is in $\mathcal{A}$. 
\end{definition}


\begin{definition}[Separate Points]
	A collection $\mathcal{A}$ of real-valued functions  on $X$ is said to \dfn{separate points} in $X$ provided that for any two distinct points $u, v \in X$, $\exists f \in \mathcal{A}$ such that $f(u) \neq f(v)$. 
\end{definition}

\begin{theorem}[Stone-Weierstrass Approximation Theorem]
	Let $X$ be a compact Hausdorff space. Suppose that $\mathcal{A}$ is an algebra of continuous real-valued functions on $X$ that separates points in $X$ and contains the constant functions. Then, $\mathcal{A}$ is dense in $C(X)$. 
\end{theorem}
Lemmas used in the proof of the Stone-Weierstrass theorem: 
\begin{lemma}
	Let $X$ be a compact Hausdorff space and $\mathcal{A}$ an algebra of continuous functions on $X$ that separates points and contains the constant functions. Then, for each closed subset $F$ of $X$ and point $x_0 \in X \setminus F$ there exists a neighbourhood $\hood$ of $x_0$ that is disjoint from $F$ with the following property: $\forall$ $\varepsilon >0$, $\exists h \in \mathcal{A}$ for which
	\begin{align}
		h < \varepsilon \text{ on } \hood, h > 1 - \varepsilon \text{ on } F, \text{ and } 0 \leq h \leq 1 \text{ on } X 	
	\end{align}
\end{lemma}

\begin{lemma}
	Let $X$ be a compact Hausdorff space and $\mathcal{A}$ an algebra of continuous functions on $X$ that separates points and contains the constant functions. Then, for each pair of disjoint and closed subsets $A$ and $B$ of $X$ and $\varepsilon > 0$, $\exists h \in \mathcal{A}$ for which 
	\begin{align}
		h < \varepsilon \text{ on } A, h > 1 - \varepsilon \text{ on } B, \text{ and } 0 \leq h \leq 1 \text{ on } X 	
	\end{align}
\end{lemma}

\begin{theorem}[Borsuk's Theorem]
	Let $X$ be a compact Hausdorff topological space. Then, the normed vector space $C(X)$ is separable $\iff$ the topology on $X$ is metrisable. 
\end{theorem}

\section{Continuous Linear Operators Between Banach Spaces}

\subsection{Normed Vector Space}

\begin{definition}[Linear Space]
	A \dfn{linear space} $X$ is an abelian group with the group operation of addition, denoted by $+$, for which, $\forall $ $\alpha \in \R$ and $u \in X$, $\exists$ a scalar product $\alpha \cdot u \in X$ for which the following four properties hold: 
	\begin{enumerate}[noitemsep]
		\item $(\alpha + \beta) \cdot u = \alpha \cdot u + \beta \cdot u $
		\item $\alpha \cdot (u+v) = \alpha \cdot u + \alpha \cdot v$ 
		\item $(\alpha \beta) \cdot u = \alpha \cdot ( \beta \cdot u) $
		\item $1 \cdot u =u$
	\end{enumerate}
\end{definition}

\begin{definition}[Norm]
	A non-negative real-valued function $|| \cdot ||$ defined on the vector space $X$ is called a \dfn{norm} if $\forall u, v \in X$ $\alpha \in \R$: 
	\begin{enumerate}[noitemsep]
		\item $|| u || =0 \iff u=0$
		\item $|| u + v || \leq ||u || + ||v||$
		\item $ || \alpha u || = | \alpha | || u || $
	\end{enumerate}
\end{definition}

\begin{definition}[Equivalent Norms]
	Two norms $|| \cdot ||_1$ and $|| \cdot ||_2$ on a vector space $X$ are said to be \dfn{equivalent} provided that there $\exists$ constants $c_1, c_2 \geq 0$ such that 
	\begin{align}
		c_1 || x ||_1 \leq || x ||_2 \leq c_2 || x_1 || \text{ 		} \forall x \in X	
	\end{align}

\end{definition}

\begin{definition}[Direct Sum]
	Let $Y$, $Z$ be subspaces of $X$. Then, $Y+Z$ is a subspace. If $Y \cap Z = \{ 0 \}$, then $Y \bigoplus Z$ is the \dfn{direct sum} of $Y$ and $Z$. 	
\end{definition}

\begin{definition}[Banach Space]
	A normed vector apace is called a \dfn{Banach space} provided it is complete as a metric space with the metric induced by the norm. 
\end{definition}

\subsection{Linear Operators}

\begin{definition}[Linear Mapping]
	Let $X$ and $Y$ be linear spaces. A mapping $T: X \rightarrow Y$ is said to be \dfn{linear} provided that for each $u, v \in X$, real numbers $\alpha, \beta \in \R$, 
	\begin{align}
		T (\alpha u + \beta v) = \alpha T(v) + \beta T(v) 
	\end{align}
	These are called \dfn{linear operators} 
\end{definition}

\begin{definition}[Bounded Operator]
	Let $X$ and $Y$ be normed linear spaces. A linear operator $T: X \rightarrow Y$ is said to be \dfn{bounded} provided that $\exists$ a constant $M \geq 0$ for which: 
	\begin{align}	
		|| T(u) || \leq M || u || \text{ 		} \forall u \in X 	
	\end{align}
	The inf of all such $M$ is called the \dfn{operator norm} of $T$ and is denoted by $||T||$. The collection of bounded linear operators from $X$ to $Y$ is denoted by $\mathcal{L}(X,Y)$. 
\end{definition}

\begin{theorem}
	A linear operator between normed linear spaces is continuous $\iff$ it is bounded. 
\end{theorem}

\begin{definition}[Linear Combination]
	Let $X$ and $Y$ be linear spaces. For $T: X \rightarrow Y$ and $S: X \rightarrow Y$ linear operators and real numbers $\alpha$ and $\beta$, define the \dfn{linear combination} $\alpha T + \beta S: X \rightarrow Y$ pointwise by: 
	\begin{align*}
		(\alpha T + \beta S) (u) := \alpha T(u) + \beta S(u)\ \forall u \in X
	\end{align*}
\end{definition}

\begin{prop}
	Let $X$ and $Y$ be normed linear spaces. The collection of bounded linear operators from $X$ to $Y$, $\lop{X}{Y}$, is a normed linear space. 
\end{prop}

\begin{theorem}
	Let $X$ and $Y$ be normed linear spaces. If $Y$ is a Banach space, then so is $\lop{X}{Y}$. 
\end{theorem}

\begin{definition}[Isomorphism]
	Let $X$ and $Y$ be normed spaces, $T \in \lop{X}{Y}$ is called an \dfn{isomorphism} if it is bijective and has a continuous inverse (we require that $T^{-1} \in \lop{X}{Y}$). 	
\end{definition}

\begin{definition}[Kernel]
	Let $T: X \rightarrow Y$ be an operator. The subspace of $X$ ker$[T] := \{ x \in X\ |\ T(x) = 0 \}$ is called the \dfn{kernel} of $T$. 
\end{definition}


\subsection{Compactness Lost: Infinite-Dimensional Normed Vector Spaces}

\begin{theorem}
	Any two norms of a finite-dimensional linear space are equivalent. 
\end{theorem}

\begin{corollary}
	Any two normed linear spaces of the same finite dimension are isomorphic. 
\end{corollary}

\begin{corollary}
	Any finite-dimensional normed vector space is complete and therefore any finite-dimensional subspace of a normed linear space is closed. 
\end{corollary}

\begin{corollary}
	The closed unit ball in a finite-dimensional normed linear space is compact. 
\end{corollary}

\begin{theorem}[Riesz's Theorem]
	The closed unit ball of a normed vector space is compact $\iff$ $X$ is finite-dimensional.
\end{theorem}

The following lemma is used to establish the above theorem. 

\begin{lemma}[Riesz's Lemma]
	Let $Y$ be a closed proper linear subspace of the normed linear space $X$. Then, $\forall$ $\varepsilon > 0$, $\exists$ a unit vector $x_0 \in X$ for which 
	\begin{align}
		|| x_0 - y || > 1 - \varepsilon\ \forall  y \in Y 	
	\end{align}
\end{lemma}

\subsection{The Open Mapping and Closed Graph Theorems}
The following theorems are results of the Baire Category Theorem. They apply to the analysis of linear operators between infinite-dimensional Banach spaces. 

\begin{theorem}
	Let $X$ and $Y$ be Banach spaces and the linear operator $T: X \rightarrow Y$ be continuous. Then, $T(X)$ is a closed subspace of $Y$ $\iff$ $\exists$ a constant $M> 0$ for which given any $y \in T(X)$, $\exists$ $x \in x$ such that
	\begin{align*}
		T(x) = y\ \text{ and } ||x|| \leq M ||y||
	\end{align*}
\end{theorem}

\begin{definition}[Open Mapping]
	A mapping $f: X \rightarrow Y$ from the topological space $X$ to the topological space $Y$ is said to be \dfn{open} provided that the image of each open set in $X$ is open in the topological space $f(X)$, where $f(X)$ has the subspace topology induced from $Y$. 
\end{definition}

\begin{theorem}[Open Mapping Theorem]
	Let $X$ and $Y$ be Banach spaces and the linear operator $T: X \rightarrow Y$ be continuous. Then, its image $T(X)$ is a closed subspace of $Y$ $\iff$ the operator $T$ is open. 
\end{theorem}

\begin{corollary}
	Let $X$ and $Y$ be Banach spaces and $T \in \lop{X}{Y}$ be bijective. Then, $T^{-1}$ is continuous. 
\end{corollary}

\begin{corollary}
	Let $|| \cdot ||_1$ and $|| \cdot||_2$ be norms on a linear space $X$ for which both $(X, || \cdot ||_1)$ and $(X, || \cdot ||_2 )$ are Banach spaces. Suppose that there is a $c \geq 0$ for which 
	\begin{align}
		|| \cdot ||_2 \leq c || \cdot ||_1 \text{ on } X 	
	\end{align}
	Then, the norms are equivalent. 
\end{corollary}

\begin{definition}[Closed]
	A linear operator $T: X \rightarrow Y$ between normed linear spaces $X$ and $Y$ is said to be \dfn{closed} provided that whenever $\{ x_n \}$ is a sequence in $X$, if $\{ x_n \} \rightarrow x_0$ and $\{ T(x_n) \} \rightarrow y_0$, then $T(x_0) = y_0$. 
\end{definition}

\begin{definition}[Graph]
	The \dfn{graph} of a mapping $T: X \rightarrow Y$ is the set $\{ (x, T(x)) \in X \times Y\ |\ x \in X \}$. 
\end{definition}
Thus, an operator is closed $\iff$ its graph is a closed subspace of the product space $X \times Y$. 

\begin{theorem}[Closed Graph Theorem]
	Let $T: X \rightarrow Y$ be a linear operator between the Banach spaces $X$ and $Y$. Then, $T$ is continuous $\iff$ it is closed. 
\end{theorem}

\begin{definition}[Linear Complement]
	Let $V \leq X$, $X$ a linear space. By Zorn's Lemma, there exists a $W \leq X$ such that there exists a direct sum decomposition: 
	\begin{align}
		X = V \oplus W	
	\end{align}
	$W$ is called the \dfn{linear complement} of $V$ in $X$. If a subspace of $X$ has a finite-dimensional linear complement in $X$, then it is said to have \dfn{finite co-dimension} in $X$. 
\end{definition}

\begin{definition}[Projection]
	For $x \in X$, and the linear complement decomposition, set $x:= v + w$ for $v \in V$ and $w \in W$. Define $P(x) := v$. Then, $P: X \rightarrow X$ is a mapping called the \dfn{projection} of $X$ onto $V$ along $W$; it has the following properties: 
	\begin{enumerate}[noitemsep]
		\item $P^2 = P$ on $X$ 
		\item $P(X) = V$ 
		\item $(ID - P)(X) = W$
	\end{enumerate}
\end{definition}

\begin{definition}[Closed Linear Complement]
	Let $X$ be a normed vector space. A closed subspace $W$ of $X$ for which $X = V \oplus W$ is called the \dfn{closed linear complement} of $V$ in $X$. 
\end{definition}

\begin{theorem}
	Let $V$ be a closed subspace of the Banach space $X$. Then, $V$ has a closed linear complement in $X$ $\iff$ $\exists$ a continuous projection of $X$ onto $V$. 
\end{theorem}

\begin{theorem}
	Let $X$ and $Y$ be Banach spaces and the linear operator $T: X \rightarrow Y$ continuous. If $T(X)$ has a closed linear complement in $Y$, then $T(X)$ is closed in $Y$. In particular, if $T(X)$ has finite co-dimension in $Y$, then $T(X)$ is closed in $Y$. 
\end{theorem}

\subsection{The Uniform Boundedness Principle}

\begin{theorem}[Uniform Boundedness Principle]
	For $X$ a Banach space and $Y$ a normed vector space, consider a family $\mathcal{F} \subseteq \lop{X}{Y}$. Suppose the family $\mathcal{F}$ is pointwise bounded in the sense that $\forall x \in X$, $\exists$ a constant $M_x \geq 0$ for which 
	\begin{align}
		|| T(x) || \leq M x \text{ 		} \forall T \in \mathcal{F} 	
	\end{align}
		Then, the family $\mathcal{F}$ is uniformly bounded in the sense that $\exists$ a constant $M \geq 0$ for which $ || T|| \leq M$ $\forall T \in \mathcal{F}$. 
\end{theorem}

\begin{theorem}[Banach-Saks-Steinhaus Theorem]
	Let $X$ be a Banach space, $Y$ a normed vector space, and $\{ T_n |\ X \rightarrow Y \}$ a sequence of continuous linear operators. Suppose that for each $x \in X$: 
	\begin{align}
		\lim_{n \rightarrow \infty} T_n(x) \text{ exists in $Y$ }	
	\end{align}
	Then, the sequence of operators $\{ T_n\ |\ X \rightarrow Y \}$ is uniformly bounded. Moreover, the operator $T: X \rightarrow Y$ defined by
	\begin{align}
		T(x) := \lim_{n \rightarrow \infty} T_n(x) \text{ 	} \forall x \in X 	
	\end{align}
	is linear, continuous, and $|| T || \leq \liminf_{n \rightarrow \infty} ||T_n||$. 
\end{theorem}

\end{document}
