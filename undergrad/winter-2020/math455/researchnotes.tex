\documentclass[11pt]{article}
\usepackage{titlesec}
\titleformat{\section}[hang]{\normalfont\scshape}{\thesection.}{1em}{}
\titleformat{\subsection}[hang]{\normalfont\scshape}{\thesubsection.}{1em}{}
\usepackage[T1]{fontenc}
\usepackage[dvipsnames]{xcolor}
\usepackage[margin=2cm]{geometry}
\usepackage{amssymb} 
\usepackage{amsmath}
\usepackage{graphicx}
\usepackage{float}
\usepackage[normalem]{ulem} 
\usepackage{enumitem} 
\setlist[enumerate]{itemsep=0mm}
\usepackage{xcolor}
\setenumerate{label=(\arabic*)}
\usepackage[utf8]{inputenc}
\usepackage[T1]{fontenc}
\usepackage{babel}
\usepackage{mathtools}
\usepackage{amsthm}
\usepackage{thmtools}
\usepackage{etoolbox}
\usepackage{fancybox}
\usepackage{framed}
\usepackage{tcolorbox}
\usepackage{xcolor} 
% open 
\newcommand{\open}[0]{\mathcal{O}}
\newcommand{\topo}[0]{\mathcal{T}}
\newcommand{\hood}[0]{\mathcal{U}}
\newcommand{\base}[0]{\mathcal{B}} 
% example environmentq
\theoremstyle{definition} 
\newtheorem{exmp}{Example}[section]


% question environment
\theoremstyle{definition}
\newtheorem{question}{Question}

%rd 
\newcommand{\rd}[0]{\mathbb{R}^d}
\newcommand{\R}[0]{\mathbb{R}}
\newcommand{\N}[0]{\mathbb{N}} 
% let E in R be measurable 
\newcommand{\EinR}[0]{Let $E \subseteq \R$ be measurable}

%integral 
\newcommand{\idx}[2]{\int_{#1}^{#2}}

% weak convergence 
\newcommand{\warrow}[0]{\rightharpoonup}
\newcommand{\fcvw}[0]{ \{f_n \} \warrow f \text{ in } L^p(E)} 

% Probability 
\DeclareRobustCommand{\bbone}{\text{\usefont{U}{bbold}{m}{n}1}}
\newcommand{\Var}[1]{\mathrm{Var[#1]}}			% variance
\newcommand{\EX}[1]{\mathbb{E}\mathrm{[#1]}}	 % expected value 
\newcommand{\seq}[1]{\{ #1_n	\}_{n \in \bb{N}}} % sequence of events
\newcommand{\pspace}[0]{( \Omega, F, P)}		% probability space
\newcommand{\msp}[0]{( \Omega, F)}		% measurable space
	
% Exercise environment 
\newenvironment{myleftbar}{%
\def\FrameCommand{\hspace{0.6em}\vrule width 2pt\hspace{0.6em}}%
\MakeFramed{\advance\hsize-\width \FrameRestore}}%
{\endMakeFramed}
\declaretheoremstyle[
spaceabove=6pt,
spacebelow=6pt
headfont=\normalfont\bfseries,
headpunct={} ,
headformat={\cornersize*{2pt}\ovalbox{\NAME~\NUMBER\ifstrequal{\NOTE}{}{\relax}{\NOTE}:}},
bodyfont=\normalfont,
]{exobreak}

\declaretheorem[style=exobreak, name=Exercise,%
postheadhook=\leavevmode\myleftbar, %
prefoothook = \endmyleftbar]{exo}

% Solution environment 
\newenvironment{mysolbar}{%
\def\FrameCommand{\hspace{0.6em}\vrule width 2pt\hspace{0.6em}}%
\MakeFramed{\advance\hsize-\width \FrameRestore}}%
{\endMakeFramed}
\declaretheoremstyle[
spaceabove=6pt,
spacebelow=6pt
headfont=\normalfont\bfseries,
headpunct={} ,
headformat={\cornersize*{2pt}\ovalbox{\NAME~\NUMBER\ifstrequal{\NOTE}{}{\relax}{\NOTE}:}},
bodyfont=\normalfont,
]{solbreak}

\declaretheorem[style=solbreak, name=Solution,%
postheadhook=\leavevmode\mysolbar, %
prefoothook = \endmysolbar]{sol}

% HEADERS
\usepackage{fancyhdr}
 
\pagestyle{fancy}
\fancyhf{}
\fancyhead[LE,RO]{Page \thepage}
\fancyhead[RE,LO]{Math 455: Analysis 4}
\fancyhead[CE,CO]{Winter 2020 -- Midterm Summary}
\fancyfoot[LE,RO]{}

% Definitions
\newcommand{\dfn}[1]{\textbf{\textcolor{blue}{#1}}}
\newcommand{\im}[1]{\textbf{\textcolor{red}{#1}}}

% lower integral
\usepackage{accents}

\newcommand{\ubar}[1]{\underaccent{\bar}{#1}}
\def\avint{\mathop{\,\rlap{-}\!\!\int}\nolimits} 

% custom commands 
\newcommand{\bb}[1]{\mathbb{#1}}
\newcommand{\vc}[1]{\mathbf{#1}}
\newcommand{\step}[1]{\textbf{#1}\textbf{. Step:}}
\newcommand{\pdv}[2]{\frac{\partial #1}{\partial #2}}
\newcommand{\sets}[2]{ \left\{ #1\ |\ #2 \right\}}
\DeclareMathOperator{\Tr}{Tr}

% Proofs
\newcommand{\claim}[1]{\textbf{#1}\textbf{. Claim:}}

	% iff proofs
	\newcommand{\rhs}[0]{(\Rightarrow )}
	\newcommand{\lhs}[0]{(\Leftarrow )}

% sequence of functions
\newcommand{\funcseqx}{(f_n(x))_{n \in \bb{N}}}
\newcommand{\funcseq}{(f_n)_{n \in \bb{N}}}

% measurable sets 
\newcommand{\measurable}{f^{-1}([-\infty, c[)} 

% heat equation 
\newcommand{\pbdry}[2]{C^{(#1, #2)} (\Omega_T) \cap C (\overline{\Omega_T})}
\DeclareMathOperator\erf{erf}
\newcommand{\mbf}[1]{\mathbf{#1}}

% Laplace Equation 
\newcommand{\lapbdry}[1]{C^{#1} (\Omega) \cap C (\overline{\Omega})}


% math environments 
\usepackage[utf8]{inputenc}
\newtheorem{theorem}{\textcolor{blue}{Theorem}}
\newtheorem{corollary}{Corollary}
\newtheorem{lemma}[theorem]{Lemma}
\theoremstyle{definition}
\newtheorem{definition}{\textcolor{OliveGreen}{Definition}}
\newtheorem{prop}{\textcolor{red}{Proposition}}
\newtheorem{ex}{\textcolor{Maroon}{Example}}
\theoremstyle{remark}
\newtheorem*{remark}{Remark}

% cookbook proofs 
\newcommand{\cb}[3]{\underline{(#1 #2): #3:}}

\usepackage{tcolorbox}
\tcbuselibrary{theorems}

% theorems 
\newtcbtheorem[number within=section]{mytheo}{Theorem}%
{colback=blue!5,colframe=blue!35!black,fonttitle=\bfseries}{th}

% definitions 
\newtcbtheorem[number within=section]{defn}{Definition}%
{colback=black!5,colframe=black!35!black,fonttitle=\bfseries}{th}

% axioms
\newtcbtheorem[number within=section]{ax}{Axioms}%
{colback=OliveGreen!5,colframe=black!35!OliveGreen,fonttitle=\bfseries}{th}


% important examples
\newtcbtheorem[number within=section]{examp}{Example}%
{colback=Mahogany!5,colframe=black!35!Mahogany,fonttitle=\bfseries}{th}

% upper and lower riemann integrals
\newcommand{\upRiemannint}[2]{
  \overline{\int_{#1}^{#2}}
}
\newcommand{\loRiemannint}[2]{
  \underline{\int_{#1}^{#2}}
}

\usepackage{hyperref}
\hypersetup{
    colorlinks,
    citecolor=black,
    filecolor=black,
    linkcolor=black,
    urlcolor=black
}

\begin{document}

\begin{center}
	\textbf{Math 455: Analysis IV Summary} \\
	\textbf{Midterm Date: 12 March 2020 18.00 - 20.00} \\
	\textbf{Key Results, Theorems, Definitions, etc.} \\
	\textbf{Shereen Elaidi}
\end{center}

Meeting notes: In conformal geometry, there is a conformal laplacian. Euclidean space is flat. The laplacian is simple there. If a surface is curved, then you can still define div and grad, and you can still take $\text{div} \text{grad}$. You can write this w geodesic-normal coordinates. Locally, it looks like a normal laplacian. 

\section{Conformal Laplacian}
It looks like the usual laplacian. If a manifold has dimension $n$, $n \geq 3$, then it looks like the following:  
\begin{align*}
	P_g =  \nabla_g + \underbrace{\frac{(n-2)}{4(n-1)}R_g}_{:= (*)}
\end{align*}
Where $R_g$ is the scalar curvature. The three solns to big bang eqn come from the three geometries (see home midterm). Positive curvature is ``like gold''; positively curved metrics are rare and easy. Negative scalar curvature is like walking up burnside?? It's difficult to increase scalar curvature. 

\begin{theorem}[Kazdan-Warner]
	Manifolds on higher dimensions have three types: 
	\begin{enumerate}[noitemsep]
		\item Type 1: there exist metrics with positive scalar curvature.
		\item Type 2: $R_g > 0$ do not exist, but there exist metrics with identically zero $R_g \equiv 0$.  
		\item Type 3: No metrics of positive curvature; none with zero. Only exist negative curvature metrics. Most manifolds are of this class. 
	\end{enumerate}
	On every conformal class, there is a metric of constant scalar curvature. E, S, H: 0, 1, $-1$. Suppose the scalar curvature is constant. 
\end{theorem}

Suppose the scalar curvature is positive. Then, you are talking a Laplacian where all the eigenvalues are non-negative. Then you are adding an operator that is essentially an identity. This maintains the old eigenfunctions. The eigenvalue will shift by the constant added to $\nabla_g$. Conformal Laplacian will have all eigenvalues non-zero. 

For scalar curvature negative, we are subtracting. It stays an eigenfunction with negative eigenvalue. If zero, it will have eigenvalue zero. 

\section{What have we noticed so far?}
It may happen that some eigenvalues of the usual laplacian may equal that number  (*) in a given conformal class. Then, when you subtract, the conformal laplacian's eigenvalue will be zero. Nice transformation formula: suppose two metrics $g_1$ and $g_0$ are conformally equivalent. Then, $g_1(x) = e^{2 \omega(x)} \cdot g_0(x)$ where $e^{2 \omega(x)}$ is a positive function. The $g_1$ and $g_1$ are Grahm-Schmitt matrices. Matrix of inner product. The conformal laplacian for the metric $g_1$ transforms as: 
\begin{align*}
	(P_{g_1}) u(x) e^{a \omega (x)} \cdot P_{g_{0}} \left[ e^{k w(x)} u(x) \right] 
\end{align*}
Suppose that $u$ is an eigenfunction: 

\begin{align*}
	P_{g_0}u \equiv 0 
\end{align*}
Then, 
\begin{align*}
	P_{g_1}( e^{kw}u) 
\end{align*}
So, instead of the function $u$, we divide it by the positive function which is inside formula. So, old conformal laplacian acting on old function with eigenvalue zero, which gives zero. Take a zero function, mult by something positive, which gives zero again. So, the kernel of $P_{g_1} + e^{-b \omega} \text{ker} P_{g_0}$ A very similar thing happens for solns of conformally invariant wave eqns in relativity. 

Nodal set of $u$ stays invariant. Usually dimension one less. The pieces that are left are called the nodal domains. They make up the whole manifold. Divided by hypersurfaces where the function vanishes. Eigenvalue must be zero. Transformation is only this simple if eigenvalue is zero. You can show that the number of negative eigenvalues can be made as large as you'd like if you change the conformal class. By continuity, get movement past zero. Suppose $u_1, u_2 \in $ ker$O_{g_0}$. Then: 
\begin{align*}
	\frac{u_1(x)}{u_2(x)}
\end{align*}
suppose you change your metrics conformally. Get $v_1, v_2$: 
\begin{align*}
	\frac{v_1(x)}{v_2(x)} = \frac{e^{-b\omega}u_1}{e^{-b \omega} u_2}
\end{align*}
So, the ratio of the two fxns remain the same. There is always a finite basis $u_1, ..., u_k$. So, you can use eigenfunctions as projective coordinates. Projective spaces: 
\begin{align}
	\Phi(x) = [u_1(x)\ |\ u_2(x) \| \cdots \| u_k(x) ]	
\end{align}
At least one of these eigenfunctions need to not be zero: 
\begin{align*}
	x \in M \setminus \bigcap_{j=1}^k \text{nod}(u_k)
\end{align*}
Then, the map is well-defined. 

\subsection{Relation to relativity}
There are some models of the univ. Univ is a 3d manifold. Then, the metric evolves in time in a certain way. There are some solutions of this equation, and in three dimensions there is a conformally-invariant version of this equation. If you take a soln of this other eqn in relativity, and if you change some metric conformally, what happens is that the solns of the equation get multiplied by a positive function (ex setup remains true for these solns). One-line observation, but maybe it could be interesting to explore a bit further. 

In the paper, they only looked at modal problems (the most symmetric possible geometries. 3d space is very summetric, 3d sphere, rotate. 3d hyperbolic, more difficult but still can move and quite symmetric). You can write this equation starting with a metric which is not so nice. You could still write down the soln of this equation, and then you can compare. Some things will be conformally invariant. 

One thing in GR that is conformally invariant is light cone. Something similar happens to the soln of the conformal box equation. Something which could maybe be interesting to do is to try to write down the solution of FRW starting with not maybe inf dim hyperbolic. Compact hyperbolic manifolds. Is the universe compact??

Toy modek: try smth small which is compact; try to solve equations on this thing, compare to solns on toher manifolds. There will be some conformally invariant objects of this type. 

3:20: ISM application. 

\end{document}