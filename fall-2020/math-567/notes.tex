\documentclass[psamsfonts]{amsart}
\newcommand\hr{\par\vspace{-.5\ht\strutbox}\noindent\hrulefill\par}
\usepackage{hyperref}
\usepackage{amssymb,amsfonts}
\usepackage[all,arc]{xy}
\usepackage{enumerate}
\usepackage{mathrsfs}
\usepackage{stackengine}
\usepackage{mathtools}
\stackMath
\usepackage{graphicx}
\newcommand{\given}{\ |\ }
\usepackage{graphics}
\usepackage[margin=2cm]{geometry}
\newtheorem{thm}{Theorem}[section]
\newtheorem{cor}[thm]{Corollary}
\newtheorem{prop}[thm]{Proposition}
\newtheorem{lem}[thm]{Lemma}
\newtheorem{conj}[thm]{Conjecture}
\newtheorem{quest}[thm]{Question}

\theoremstyle{definition}
\newtheorem{defn}[thm]{Definition}
\newtheorem{defns}[thm]{Definitions}
\newtheorem{con}[thm]{Construction}
\newtheorem{exmp}[thm]{Example}
\newtheorem{exmps}[thm]{Examples}
\newtheorem{notn}[thm]{Notation}
\newtheorem{notns}[thm]{Notations}
\newtheorem{addm}[thm]{Addendum}
\newtheorem{exer}[thm]{Exercise}
\newtheorem{ques}[thm]{Question}

\theoremstyle{remark}
\newtheorem{rem}[thm]{Remark}
\newtheorem{rems}[thm]{Remarks}
\newtheorem{warn}[thm]{Warning}
\newtheorem{sch}[thm]{Scholium}

\newcommand{\T}[0]{\mathbb{T}}
\newcommand{\N}[0]{\mathbb{N}}
\newcommand{\Q}[0]{\mathbb{Q}}
\newcommand{\R}[0]{\mathbb{R}}
\newcommand{\Z}[0]{\mathbb{Z}}
\newcommand{\Sph}[0]{\mathbb{S}}
\newcommand{\Hyp}[0]{\mathbb{H}}
\newcommand{\christoffel}[4][\Gamma]{#1^{\hspace{5pt}#3}_{#2 \hspace{5pt} #4}}
\newcommand{\no}{\noindent}
\usepackage{xcolor,cancel}
\newcommand\Ccancel[2][black]{\renewcommand\CancelColor{\color{#1}}\cancel{#2}}
\newcommand{\M}{\mathcal{M}}
\usepackage{framed}
\usepackage{appendix}
\newcommand{\eps}[0]{\varepsilon}

\makeatletter
\let\c@equation\c@thm
\makeatother
\numberwithin{equation}{section}

\newcommand{\holdernorm}[0]{||f||_{C^{0, \gamma}(\Omega)}}

\bibliographystyle{plain}

\title[Math 567: Functional Analysis]{Math 567: Functional Analysis (Fall 2020 Semester)}
\author{Shereen Elaidi as taught by Prof. lin; Last Updated: \today}
\date{Fall 2020 Semester}

\begin{document}

\maketitle

This class is about linear functional analysis. This has a lot in common with linear algebra in infinite-dimensional spaces. We can think of this as infinite-dimensional linear algebra. There are two main applications of this: (a) geometry and topology in infinite dimensions and (b) solving PDEs. Recall that in regular linear algebra, we used those tools to solve linear systems. The infinite-dimensional equivalent to this is a PDE. In this class, we'll focus on the second application of functional analysis.


\section{Basic Functional Analysis}
This corresponds to Chapters 4 and 5 of the textbook.
\subsection{Banach Spaces and General Topology}
Let \( X \) be a vector space. Recall that this means that it is closed under addition and scalar multiplication. 

\begin{defn}[Norm]
	A \textbf{norm} on a vector space \( X \), \( || \cdot ||: X \rightarrow [0, \infty [ \), satisfies the following three properties: 
	\begin{enumerate}
		\item \( || x || = 0 \iff x = 0 \). 
		\item \textbf{(Homogeneity)}: \( || \lambda x || = |\lambda | || x || \) for each \( x \in X \), \( \lambda \in \R \).
		\item \textbf{(Triangle Inequality):} \( || x+y || \leq || x || + || y || \).
	\end{enumerate}
\end{defn}

\begin{defn}[Completeness / Banach Space]
	\( X \) is \textbf{complete} if every Cauchy sequence converges. \( (X, || \cdot ||) \) is a \textbf{Banach space} if it is a complete normed vector space.
\end{defn}

\begin{defn}[Dense Subset]
	\( Y \subseteq X \) is \textbf{dense} if 
	\begin{enumerate}
		\item \( \overline{Y} = X \) (one thing we need to note: \( \overline{Y} \) is the closure, but we need to ask ourselves ``in which topology''? ).
	\end{enumerate}
	This is equivalent to: 
	\begin{align*}
		\forall \eps > 0,\ \forall x \in X,\ \exists y \in Y\ \text{ s.t. } || x - y || < \eps.
	\end{align*}
	And also equivalent to, 
	\begin{align*}
		\forall x \in X,\ \exists \{ y_n \} \subseteq Y\ \text{ s.t. } y_n \rightarrow y \in X. 
	\end{align*}
\end{defn}

\begin{defn}[Strong Topology]
	The \textbf{strong topology} is the topology induced by the norm, \( || \cdot || \) (the open sets are characterized by the balls, \( B_r := \{ x\ |\ || x || < r \} \)). In this topology, the definitions of density given above are equivalent.
\end{defn}

\begin{defn}[Separable]
	\( X \) is \textbf{separable} if \( \exists \) a countable dense subset. 
\end{defn}

We have the following equivalent definitions of compactness. 

\begin{defn}[Compactness 1]
	\( E \subseteq X \) is \textbf{compact} if every open cover of \( E \) admits a finite subcover.
\end{defn}

\begin{defn}[Compactness 2]
	Every sequence has a convergent sub-sequence. 
\end{defn}

\begin{defn}[Compactness 3]
	For any sequence  \( \{ x_n \} \subseteq E \), there exists \( \{ x_{n_k} \} \) and \( x^* \in E \) such that \( x_{n_k} \rightarrow x^* \in E \).
\end{defn}

\begin{defn}[Pre-Compact]
	\( E \subseteq X \) is \textbf{pre-compact} if \( \overline{E} \) is compact.
\end{defn}

\subsection{Euclidean Space \( \R^n \)}
Let \( x \in \R^n \). This is denoted by \( (x_1, ..., x_n) \). Then, recall, 
\begin{align*}
	||x || = || x ||_{\ell^2} = \left( \sum_{j=1}^n x_j^2 \right)^{1/2}. 
\end{align*}
We also have these other typical norms on Euclidean space:
\begin{align*}
	|| x||_{\ell^1}  &= \sum_{j=1}^n |x_j | \\
	||x||_{\ell^p} & = \left( \sum_{j=1}^n |x_j|^p \right)^{1/p} \\
	|| x ||_{\ell^\infty} & = \max_{1 \leq j \leq n } |x_j |.
\end{align*}

\begin{defn}[Equivalent Norms]
	We say that two norms, \( | \cdot | \) and \( || \cdot || \), are equivalent if and only if there exist two constants \( a \) and \( b \) such that
	\begin{align}
		\boxed{a||x|| \leq |x | \leq b || x|| \hspace{1cm} \forall x \in X.}	
	\end{align}
\end{defn}
In words, this is saying that you can't be big on one norm but small in another. These norms are comparable; they are bounded by constants on either side. 

\begin{thm}
	All norms on \( \R^n \) are equivalent (all norms in finite dimensions are equivalent).
\end{thm}

\begin{proof}
	Let \( || \cdot || \) be the Euclidean norm, and let \( | \cdot | \) be another norm. Let \( \{ e_i \}_{i=1}^n \) be the standard basis of \( \R^n \); recall that this is \( e_i = (0, ..., 1, ... , 0 ) \) where the \( 1 \) is in the ith slot. Since this is a basis, for \( x \in X \): 
	\begin{align*}
		x = \sum_{i=1}^n x_i e_i.
	\end{align*}
	By the reverse triangle inequality, 
	\begin{align*}
		| |x| - |y| | & \leq |x-y | \\
					 & = \left| \sum_{i=1}^n (x_i - y_i) e_i \right| \\
					 & \leq \sum_{i=1}^n |x_i - y_i| |e_i| \\
					 & \leq \underbrace{ \left( \sum_{i=1}^n |e_i|^2  \right)^{1/2} }_{:= C} \left( \sum_{i=1}^n |x_i - y_i|^2   \right)^{1/2} \hspace{0.5cm} \text{(Cauchy-Schwarz) } \\
					 & \leq C ||x-y|| \hspace{1cm} (*),
	\end{align*}
	where \( C \) is some number. Norms are continuous; \( x \mapsto || x || \) is continuous \( S = \{ x\ |\ || x|| = 1 \} \) (the unit ball). By \( (*) \), \( x \mapsto | x | \) is continuous on \(  S \). \( S \) is closed and bounded on \( \R^n \) which means that \( S \) is compact. By the extreme value theorem, this means that there exist two constants \( a, b \in \R \) such that
	\begin{align}
		a \leq |x | \leq b \hspace{1cm} \forall x \in S.	
	\end{align}
	Observe that \( | x | = 0 \iff x = 0 \), which implies that \( a > 0 \). For any \( y \in \R^n \), let \( x := \frac{y}{||y||} \in S \). Then, 
	\begin{align*}
		a \leq \left| \frac{y}{||y||}  \right| \leq b \iff a \leq \frac{1}{||y ||} |y| \leq b \iff a || y || \leq |y| \leq b || y || \hspace{0.5cm} \forall y \in \R^n \setminus \{ 0 \}.
	\end{align*}
	The case of \( y = 0 \) is straightforward. This proves that any norm in a finite-dimensional vector space are equivalent. Note that this proof rests on the fact that we have a basis.
\end{proof}

\begin{rem}
	\( \R^n \) is separable in any norm. The typical countable dense subset of \( \R^n \) is \( \Q^n \). We will see in infinite-dimensions that all norms are not equivalent.
\end{rem}


\subsection{The Spaces of \( C^r \), \( C^{r, \gamma} \) of Continuous Functions}

\begin{defn}[\(C^0 \)]
	Let \( \Omega \subseteq \R^n \) be open. Then, 
	\begin{align*}
		C^0(\Omega) & \coloneqq \{ f\ |\ \Omega \to \R \text{ s.t. } f \text{ is continuous on } \Omega \}  \\
		C^0(\overline{\Omega}) & \coloneqq \{ f\ |\ \overline{\Omega} \rightarrow \R \text{ s.t. } f \text{ is continuous on } \overline{\Omega} \}. 
	\end{align*}
\end{defn}
This implies that \( f \in C^0(\overline{\Omega}) \) is bounded and uniformly continuous.

\begin{defn}[\( || \cdot ||_\infty \)]
		The standard norm on \( C^0(\Omega) \) is
		\begin{align}
			|| u ||_\infty \coloneqq \sup_{x \in \Omega} |u(x) | \leftrightarrow \text{ uniform convergence}. 	
		\end{align}
\end{defn}

\begin{prop}
	\begin{enumerate}
		\item \( (C^0(\Omega), || \cdot ||_\infty ) \) is a Banach space. 
		\item If \( \Omega \subseteq \R^n \) is bounded, then \( C^0(\overline{\Omega})\) is separable.
	\end{enumerate}
\end{prop}
We will only give a sketch of the proof.
\begin{proof}
	\begin{enumerate}
		\item The uniform limit of continuous functions is continuous. 
		\item Follows from the Weierstrass approximation theorem: polynomials are dense in \( C^0(\overline{\Omega}) \); then, consider the polynomials with rational coefficients.
	\end{enumerate}
\end{proof}

\subsubsection{Higher-Order Derivatives}
Recall some notation from advanced calculus: 
\begin{align}
	\boxed{Du = \nabla u = \text{ gradient of } u = \begin{bmatrix}
		\partial_1 u \\
		\vdots \\
		\partial_n u 
	\end{bmatrix}}.	
\end{align}
We consider the \textbf{multi-index} \( \alpha = (\alpha_1, ..., \alpha_n ) \), where \( |\alpha| \coloneqq \alpha_1 + ... + \alpha_n \), and \( \forall k \in \R^n \), define \( k^\alpha \coloneqq k_1^{\alpha_1} \cdots h_n^{\alpha_n} \). Then, in this notation, 
\begin{align*}
	D^\alpha u &  = \partial_1^{\alpha_1} \cdots \partial_n^{\alpha_n} u  \\
			& = \frac{\partial^{|\alpha|} u}{\partial x_1^{\alpha_1} \cdots \partial x_n^{\alpha_n}} \hspace{1cm} \text{(partial derivative)}.
\end{align*}

\begin{defn}[\(C^r(\Omega) \)]
	\begin{align} 
C^r(\Omega) := \{ f \given D^\alpha f \in C^0(\Omega)\ \forall\ |\alpha| < r \} 		
	\end{align}
\end{defn}
In words, this means that all partial derivatives less than or equal to \( r \) are continuous. Then, we can define the following space:
\begin{align}
	C^\infty (\Omega) \coloneqq \bigcap_{r=1}^\infty C^r(\Omega).	
\end{align}

\begin{defn}[Support of \( f \) ]
		The \textbf{support} of \( f \) is defined as the smallest closed set such that \( f \equiv 0 \) on \( \R^n \setminus \operatorname{supp}(f) \).
		\begin{align}
			\operatorname{supp} (f) \coloneqq \overline{\{ x \given f(x) \neq 0 \} }.
		\end{align}
\end{defn}

\begin{defn}[Compactly Contained]
	A set \( K \subset \subset \Omega \) means that \( K \subseteq \Omega \) is compact. We say that \( K \) is \textbf{compactly contained} in \( \Omega \) if \( K \subset \subset \Omega \), \( \Omega\) is bounded, and that there exists an \( \eps > 0 \) such that \( B(x, \eps ) \subseteq \Omega \) for all \( x \in K \). This is equivalent to for all \( x \in K \), 
	\begin{align}
		\exists\ \eps > 0 \text{ s.t. } d(x, \partial \Omega) \coloneqq \inf_{y \in \partial \Omega} |x-y| > \eps.
	\end{align}
\end{defn}

\begin{defn}[\( C_c^r ( \Omega) \)]
	\begin{align}
		C_c^r(\Omega) \coloneqq \{ f \given f \in C^r(\Omega), \text{supp}(f) \subset \subset \Omega \}.	
	\end{align}
\end{defn}

\begin{defn}[Norm on \( C^r(\overline{\Omega}) \)]
		Let \( \Omega \) be bounded. Then, 
		\begin{align}
			\boxed{|| 
			f||_{C^r} \coloneqq \sum_{|\alpha|\leq r} \sup_{ x \in \Omega} |D^\alpha f(x)|.}	
		\end{align}
\end{defn}

\begin{prop}
Let \( \Omega \subseteq \R^n \) be bounded. Then, \( C^r(\Omega) \) is a separable Banach space
 (in fact, all you need for separable is that it is bounded) for all \( r < \infty \).
\end{prop}

\begin{rems}
	\( C_c^r(\Omega) \) is not complete. \( C^\infty (\Omega) \) is not complete. However, subspaces of \( C^\infty \) is still complete with some norm.
\end{rems}
We also introduce, 
\begin{defn}[Hölder Continuous \( C^{0, \gamma} (\Omega) \)]
	\( f: \Omega \rightarrow \R \) is \textbf{Hölder Continuous} with exponent \( \gamma \in [0, 1[ \) if there exists a \( C \) such that
	\begin{align}
		\boxed{| f(x) - f(y) | \leq C || x-y||^\gamma.	}
	\end{align}
	If \( \gamma =1 \Rightarrow  f \) is \textbf{Lipschitz Continuous}.
\end{defn}

Also, 
\begin{align*}
	\boxed{[f]_{C^{0, \gamma}(\Omega)} \coloneqq \sup_{x, y \in \Omega} \frac{|f(x) - f(y) |}{|| x- y ||^\gamma}}
\end{align*}
is called the \textbf{Hölder seminorm}. This is nor a norm, but we can make it a norm: 
\begin{defn}[\( || \cdot ||_{C^{0,\gamma}} \)]
	\begin{align}
		\boxed{||f||_{C^{0, \gamma}(\Omega)} \coloneqq ||f||_\infty + [f]_{C^{0, \gamma} (\Omega)}.} 	
	\end{align}
\end{defn}
On the homework, you'll show that 
\begin{align*}
	(C^{0, \gamma}(\Omega), ||f||_{C^{0, \gamma}(\Omega)}) 
\end{align*}
is complete.
\begin{defn}[\(C^{r, \gamma} \)]
\begin{align}
		\boxed{C^{r, \gamma}(\Omega) \coloneqq \{ f \given f \in C^r(\Omega) \text{ and } |D^\alpha f(x) - D^\alpha f(y) | \leq C || x-y||^\gamma\ \forall \ |\alpha| = r \}}.
\end{align} 
The norm of this space is given by, 
\begin{align}
	\boxed{|| f||_{C^{r, \gamma}} \coloneqq ||f||_{C^r} + \sup_{|\alpha| = r} [D^\alpha f]_{C^{0,r}}.}	
\end{align}
\end{defn}

\begin{rem}
	If \( f \in C^{0, \gamma}(\Omega) \), \( \Omega \) bounded, then \( f \in C^{0, \alpha}(\Omega) \) for all \( 0 < \alpha \leq \gamma \)	
\end{rem}


\begin{rem}
	\textbf{(Rademacher's Theorem)}. If \( f \in C^{0,1} \), then \( f \) is differentiable a.e.	
\end{rem}

\subsection{Integration Theorems}

\begin{thm}[Monotone Convergence Theorem]
		If \( f_n \uparrow f \) pointwise for almost every \( x \), then 
		\begin{align}
			\boxed{\lim_{n \rightarrow \infty} \int_{\Omega} f_n(x) dx = \int_{\Omega} f(x) dx}.
		\end{align}
\end{thm}

\begin{thm}[Fatou's Lemma]
	Let \( \{ f_n \} \) be a sequence of measurable functions that are all positive. Then, 
	\begin{align}
		\boxed{\int_{\Omega} \liminf_{n \rightarrow \infty} f_n(x) \leq \liminf_{n \rightarrow \infty} \int_\Omega f_n(x) dx}	
	\end{align}
\end{thm}

\begin{thm}[Dominated Convergence Theorem]
		Assume that \( \{ f_n \} \) are measurable, \( f_n \rightarrow f \) pointwise a.e. Then, if \( |f_n(x)| \leq g(x) \) for all \( n \in \N \) for almost every \( x \), where \( g \in L^1(\Omega) \), then
		\begin{align}
			\boxed{\lim_{n \rightarrow \infty} \int_\Omega f_n(x) dx = \int_\Omega f(x) dx}	
		\end{align}
\end{thm}
This is the theorem that you use when you want to differentiate under integrals.

\begin{thm}
	The space \( C_c^0(\R^n ) \) is dense in \( L^1(\R^n) \).	
\end{thm}

\begin{thm}[Fubini-Tonelli] 
	For all \( f: X \times Y \to \R^n \), 
	\begin{align}
		\boxed{\int_X \int_Y | f(x,y)| dy dx = \int_Y \int_X |f(x,y)|dx dy = \int_{X \times Y } |f(x,y)| d(x,y).} 
	\end{align}
	If, moreover, \( f \in L^1(X \times Y) \), 
	\begin{align}
		\int_X \int_Y f(x, y) dy dx = \int_Y \int_X f(x, y) dx dy = \int_{X \times y} f(x,y) d(x,y) 	
	\end{align}
\end{thm}

\subsection{Elementary \( L^p \) Spaces}

\begin{defn}[\(L^p \)] 
		Fix \( 1 \leq p < \infty \), let \( \Omega \subseteq \R^n \). Then, we define the \(L^p\) space to be
		\begin{align}
			\boxed{L^p(\Omega) \coloneqq \{ f: \Omega \to \R \given f \text{ measurable } |f|^p \in L^1(\Omega) \}}
		\end{align}
		with the following norm, 
		\begin{align}
			|| f ||_{L^p} \coloneqq \left[ \int_{\Omega} |f(x)|^p \right]^{1/p}.	
		\end{align}
\end{defn}

\begin{defn}[\( L^\infty \)]
		We define \(L^\infty\) to be: 
		\begin{align}
			L^\infty(\Omega) \coloneqq \{ f: \Omega \to \R \given f \text{ measurable }, \exists\ C \text{ s.t. } |f(x)| \leq C \text{ a.e. } \}, 	
		\end{align}
		with the following norm, 
		\begin{align}
			||f||_{L^\infty} = || f||_\infty = \inf \{ c\ |\ |f(x)| \leq c\ a.e. \} 	
		\end{align}
\end{defn}
This definition implies that \( f(x) \leq || f||_\infty \) almost everywhere. Below are some fundamental tools that we'll be using
\begin{thm}[Hölder's Inequality]
		Let \( 1 \leq p, p' \leq \infty \). If \( f \in L^p(\Omega) \), \( g \in L^{p'}(\Omega) \) and \( 1/p + 1/p' = 1 \), then \( fg \in L^1 \) and
		\begin{align}
			\boxed{\int|fg|dx \leq ||f||_p ||g||_{p'}}	
		\end{align}
\end{thm}

\begin{thm}[Minkowski's Inequality]
	For all \( p \in [1, \infty] \), 
	\begin{align}
		\boxed{||f+g||_p \leq ||f||_p + ||g||_p.} 	
	\end{align}
\end{thm}
As a consequence of [Minkowski's Inequality, \(L^p \) is a vector space.








\end{document}