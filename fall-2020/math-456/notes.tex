\documentclass[psamsfonts]{amsart}
\newcommand\hr{\par\vspace{-.5\ht\strutbox}\noindent\hrulefill\par}
\usepackage{hyperref}
\usepackage{amssymb,amsfonts}
\usepackage[all,arc]{xy}
\usepackage{enumerate}
\usepackage{mathrsfs}
\usepackage{stackengine}
\stackMath
\usepackage{graphicx}
\usepackage{graphics}
\usepackage[margin=2cm]{geometry}
\newtheorem{thm}{Theorem}[section]
\newtheorem{cor}[thm]{Corollary}
\newtheorem{prop}[thm]{Proposition}
\newtheorem{lem}[thm]{Lemma}
\newtheorem{conj}[thm]{Conjecture}
\newtheorem{quest}[thm]{Question}

\theoremstyle{definition}
\newtheorem{defn}[thm]{Definition}
\newtheorem{defns}[thm]{Definitions}
\newtheorem{con}[thm]{Construction}
\newtheorem{exmp}[thm]{Example}
\newtheorem{exmps}[thm]{Examples}
\newtheorem{notn}[thm]{Notation}
\newtheorem{notns}[thm]{Notations}
\newtheorem{addm}[thm]{Addendum}
\newtheorem{exer}[thm]{Exercise}
\newtheorem{ques}[thm]{Question}

\theoremstyle{remark}
\newtheorem{rem}[thm]{Remark}
\newtheorem{rems}[thm]{Remarks}
\newtheorem{warn}[thm]{Warning}
\newtheorem{sch}[thm]{Scholium}

\newcommand{\T}[0]{\mathbb{T}}
\newcommand{\N}[0]{\mathbb{N}}
\newcommand{\R}[0]{\mathbb{R}}
\newcommand{\Z}[0]{\mathbb{Z}}
\newcommand{\Sph}[0]{\mathbb{S}}
\newcommand{\Hyp}[0]{\mathbb{H}}
\newcommand{\christoffel}[4][\Gamma]{#1^{\hspace{5pt}#3}_{#2 \hspace{5pt} #4}}
\newcommand{\no}{\noindent}
\usepackage{xcolor,cancel}
\newcommand\Ccancel[2][black]{\renewcommand\CancelColor{\color{#1}}\cancel{#2}}
\newcommand{\M}{\mathcal{M}}
\usepackage{framed}
\usepackage{appendix}

\makeatletter
\let\c@equation\c@thm
\makeatother
\numberwithin{equation}{section}

\bibliographystyle{plain}

\title[Math 456: Algebra 3]{Math 456: Algebra 3 (Fall 2020 Semester)}
\author{Shereen Elaidi, based on the lectures of Prof. goren}
\date{Fall 2020 Semester}

\begin{document}


\maketitle


\section{Basic Concepts and Key Examples}
First we'll review some notions from Algebra 1. 

\begin{defn}[Group]
	A \textbf{group} \( G \) is a non-empty set with a set function \( m: G \times G \rightarrow G \). This can be abbreviated by \( g * h \). This function satisfies the following: 
	\begin{enumerate}
		\item \textbf{(Associativity)}: \( f(gh) = (fg)h \) for all \(f, g h \in G \).
		\item \textbf{(Identity)}: there is an element \( g \in G \) such that for all \( g \in G \) we have \( eg = ge = g \). 
		\item \textbf{(Inverse)}: for every \( g \in G \), there is an element \( h \in G \) such that \( gh = hg = e \).
	\end{enumerate}
\end{defn}
A group with finite element is called of \textbf{finite order}. A group is called \textbf{abelian} if its commutative. 

\subsection{Subgroup and Order}
\begin{defn}
	A \textbf{subgroup} \( H \) of \( G \) is a subset of \( G \) which obeys the following:
	\begin{enumerate}
		\item \( e \in H \). 
		\item \textbf{(Closed under multiplication)}: \( g, h \in H \) implies that \( gh \in H \).
		\item \textbf{(Closed under inversion)}: if \( g \in H \) then \( g^{-1} \in H \).
	\end{enumerate}
\end{defn}

A \textbf{cyclic subgroup} is a subgroup \( H \) for which there is an element \( h \in H \) such that \( H = \{ h^n\ |\ n \in \Z \} \). We denote the set \( \{ h^n\ |\ n \in \Z \} \) by \( \langle h \rangle \); it is the \textbf{cyclic subgroup generated by \( h \)}. The \textbf{order} of an element \( h \in G \), denoted by \( \operatorname{ord}(h) \), is the minimal \( n \in \Z^+ \) such that \( h^n = e \). \( h \in G \) has \textbf{infinite order} if no such \( n \) exists.

\begin{prop}
	Let \( H \) be a group and \( h \in H \). Then, \( \operatorname{ord}(h) = \# \langle h \rangle \).
\end{prop}

\begin{proof}
	Suppose that \( h \) has finite order \( n \). To prove this, we'll first show that
	\begin{align*}
		\langle h \rangle = \{ 1, h, ..., h^{n-1} \}.
	\end{align*}
	This will allow us to conclude that \( \# \langle h \rangle = n \). Let's prove this statement. 
	\newline
	``\( \supseteq \)'': Clear from the definition of \( \langle h \rangle \).
	\newline
	``\( \subseteq \)'': To prove this inclusion, we need to show that \( h^n = h^i \) for \( 0 \leq i \leq n -1 \) and that none of the elements \( 1, h, ..., h^{n-1} \) are equal. Write \( r = tn + i \), where \( 0 \leq i \leq n-1 \). Then, this means that we can write
	\begin{align}
		h^r = (h^n)^t h^i = (1)^t h^i = h^i. \checkmark
	\end{align}
	Now we need to show that none of the elements \( 1, h, ..., h^{n-1} \) are equal. For a contradiction, suppose that \( h^i = h^j \) are the same, where \( 0 \leq i \leq j \leq n-1 \). This implies that \( h^{j-i} = 1_G \), which is a contradiction because \( 0 < j-i < n \). This contradicts that \( \operatorname{ord}(h) = n \).
	
	Now we need to prove the infinite case 
\end{proof}



We have the following examples of groups: \( \Z, \Z / n\Z \) and \( (\Z / n \Z)^+ \). Let's investigate these groups further.
\begin{itemize}
	\item For \( \Z \), we have that \( (\Z, + ) \) is a group. The elements are given by \( \Z = \{ ..., -3, -2, -1, 0, 1, 2, 3, ... \} \). The number of elements in this group, denoted by \( \# \Z \), is infinite.
	\item If \( n \geq 1 \) is an integer, then we have the group \( \Z / n \Z \); this can also be denoted by \( \Z_n \). There is also a notion of the order of an element in the group. The \textbf{order} is the minimum integer \( n > 0 \) such that \( g^n = 1_G \). That is for a multiplicative group; for an additive group, the order is the minimal \( n > 0 \) such that \( ng = 0 \) in \( G \).
\end{itemize}



\end{document}