\documentclass[reqno,11pt]{amsart}
\usepackage[margin=2cm]{geometry}                
\usepackage{color}
\definecolor{darkblue}{rgb}{0.0,0.0,0.4}
\usepackage[pdfauthor={Shereen Elaidi},pdftitle={Sobolev Spaces},pdfsubject={PDE},colorlinks,citecolor=darkblue,linkcolor=darkblue,urlcolor=darkblue]{hyperref}
%\usepackage{hyperref}
\usepackage{graphicx}
\usepackage{amssymb}
\usepackage{enumerate}
\usepackage{subcaption}
\usepackage{mathrsfs}
\usepackage{xlop}
\usepackage{MnSymbol}
\usepackage[ruled]{algorithm2e}

\makeatletter
\newcommand{\LeftEqNo}{\let\veqno\@@leqno}
\makeatother

\newtheorem{theorem}{Theorem}
\newtheorem{proposition}[theorem]{Proposition}
\newtheorem{lemma}[theorem]{Lemma}
\newtheorem{corollary}[theorem]{Corollary}
\newtheorem{assumption}[theorem]{Assumption}
\newtheorem{axiom}{Axiom}
\setcounter{axiom}{-1}
\theoremstyle{definition}
\newtheorem{definition}[theorem]{Definition}
\newtheorem{example}[theorem]{Example}
\newtheorem{exercise}{Exercise}
\newtheorem{remark}[theorem]{Remark}
\theoremstyle{remark}

\newcommand{\reals}{{\mathbb{R}}}
\newcommand{\TT}{{\mathbb{T}}}
\newcommand{\sphere}{{\mathbb{S}}}
\newcommand{\hyp}{{\mathbb{H}}}
\usepackage{cite}
\newcommand{\C}{\mathbb{C}}
\newcommand{\R}{\mathbb{R}}
\newcommand{\Q}{\mathbb{Q}}
\newcommand{\Z}{\mathbb{Z}}
\newcommand{\N}{\mathbb{N}}
\newcommand{\D}{\mathbb{D}}
\newcommand{\eps}{\varepsilon}

\newcommand\Hol{\mathscr{O}}
\newcommand\Con{\mathscr{C}}
\renewcommand\Re{\mathrm{Re}}
\renewcommand\Im{\mathrm{Im}}
\newcommand\exd{\mathrm{d}}
\newcommand{\myref}[2]{{\hyperref[#2]{#1\thinspace\ref*{#2}}}}



\usepackage{amsmath}
\usepackage{enumitem}
\usepackage{graphicx}
\usepackage{amssymb}
\usepackage{float}
\usepackage{hyperref}
%\hypersetup{
%    colorlinks,
%    citecolor=black,
%    filecolor=black,
%    linkcolor=black,
%    urlcolor=black
%}

\title{Calculus: Single Variable, Multivariable, Differential Equations, and Vector Calculus Summary}
\author{Shereen Elaidi}
\date{8 June 2020}
% HEADERS
\usepackage{fancyhdr}
 
\pagestyle{fancy}
\fancyhf{}
\fancyhead[LE,RO]{Page \thepage}
\fancyhead[RE,LO]{Calculus Summary}
\fancyhead[CE,CO]{June 2020}
\fancyfoot[LE,RO]{}

\newcommand{\dfn}[1]{\underline{\textbf{#1}}}
\newcommand{\deriv}[1]{\frac{d}{dx} \left[ #1 \right]}
\DeclareMathOperator{\arcsec}{arcsec}
\DeclareMathOperator{\arccot}{arccot}
\DeclareMathOperator{\arccsc}{arccsc}
\DeclareMathOperator{\csch}{csch}
\DeclareMathOperator{\sech}{sech}
\DeclareMathOperator{\arcsinh}{arcsinh}
\DeclareMathOperator{\arccosh}{arccosh}
\DeclareMathOperator{\arctanh}{arctanh}
\DeclareMathOperator{\arcsech}{arcsech}
\DeclareMathOperator{\arccsch}{arccsch}
\DeclareMathOperator{\arccoth}{arccoth}

\begin{document}
\maketitle 
\begin{abstract}
	The purpose of this document is to review Calculus. The content here should be equivalent to Math 140, Math 141, Math 222, and Math 248/358 at McGill. 
\end{abstract}
\tableofcontents
\section{Single Variable Calculus}
\subsection{Limits and Derivatives}
\begin{itemize}[noitemsep]
	\item \dfn{Precise Definition of a Limit}: Let $f$ be a function defined on an open interval $]a,c[$ that contains the number $a$. Then, we say that the limit of $f(x)$ as $x$ approaches $a$ is $L$, and we write $\lim_{x \rightarrow a} f(x) = L$ if for every $\varepsilon > 0$, $\exists$ a $\delta > 0$ such that $0 < | x - a | < \delta$ $\Rightarrow$ $|f(x) - L| < \varepsilon$.
	\begin{itemize}[noitemsep]	
		\item Heuristically, this means that if any small interval $] L - \varepsilon, L +  \varepsilon [$ is given around $L$, then we can find an interval $]a - \delta, a + \delta [$ around $a$ such that $f$ maps the points in $]a - \delta, a + \delta [$ (except possibly $a$) into the interval $] L - \varepsilon, L+ \varepsilon[$. 
	\end{itemize}
	\item \dfn{Continuous}: A function $f$ is said to be continuous at a number $ a \in \R$ if $\lim_{x \rightarrow a} f(x) = f(a)$. 
	\item \dfn{Intermediate Value Theorem}: Let $f$ be continuous on the interval $[a,b]$ and let $N$ be any number between $f(a)$ and $f(b)$ where $f(a) \neq f(b)$. Then, there exists a number $c \in ]a,b[$ for which $f(c) = N$. 
	\item \dfn{Tangent Line}: The tangent line to the curve $y = f(x)$ at the point $P = (a, f(a))$ is the line through $P$ with the slope 
	\begin{align}
		m = \lim_{x \rightarrow a} \frac{f(x) - f(a)}{x - a} \iff m = \lim_{h \rightarrow 0} \frac{f(a+h) - f(a)}{h}	
	\end{align}
	\item \dfn{Velocity / Instantaneous Velocity}: the instantaneous velocity $v(a)$ at the time $t=a$ is the limit of the average velocities: 
	\begin{align}
		v(a) := \lim_{h \rightarrow 0} \frac{f(a+h) - f(a)}{h}	
	\end{align}
	\item \dfn{Derivative}: The \dfn{derivative} of a function $f$ at a number $a \in \R$, denoted by $f'(a)$, is 
	\begin{align}
		\lim_{h \rightarrow 0} \frac{f(a+h) - f(a)}{h} = f'(a)	
	\end{align}
\end{itemize}

\subsection{Differentiation Rules}
\begin{itemize}[noitemsep]
	\item Derivative of a constant function: $\deriv{c} = 0$. 
	\item \dfn{Power Rule}: if $n \in \R$, $\deriv{x^n} = nx^{n-1}$. One can prove this using geometric series. 
	\item \dfn{Constant Multiple Rule}: if $c \in \R$ and $f$ differentiable, then $\deriv{cf(x)} = cf'(x)$. 
	\item \dfn{Constant Sum Rule}: if $f, g$ are differentiable, then $\deriv{f(x) + g(x)} = \deriv{f(x)} + \deriv{g(x)}$. 
	\item The rate of change of any exponential function is proportional to the function itself: for $f(x) := b^x$:
	\begin{align}
		f'(x) = f'(0) b^x	
	\end{align}
	\item \dfn{Derivative of the Natural Exponential Function}: 
	\begin{align}
		\deriv{e^x} = e^x 	
	\end{align}
	\item \dfn{Product Rule}: if $f, g$ are differentiable, then: 
	\begin{align}
		\deriv{f(x) g(x) } = f(x) \deriv{g(x)} + g(x) \deriv{f(x)} 	
	\end{align}
	\item \dfn{Quotient Rule}: If $f, g$ are differentiable, then: 
	\begin{align}
		\deriv{ \frac{f(x)}{g(x)} } = \frac{g(x) \deriv{f(x)} - f(x) \deriv{g(x)}}{[g(x)]^2}	
	\end{align}
	\item \dfn{Derivatives of Trigonometric Functions}: 
	\begin{itemize}[noitemsep]
		\item $\deriv{ \sin(x) } = \cos (x) $, $\deriv{ \csc(x) } = - \csc(x) \cot (x) $ 
		\item $\deriv{\cos(x) } = - \sin(x)$, $\deriv{\sec(x)} = \sec(x) \tan (x)$
		\item $\deriv{ \tan(x) } = \sec^2 (x)$, $ \deriv{\cot(x)} = - \csc^2 (x)$.
	\end{itemize}
	\item \dfn{Chain Rule}: If $g$ is differentiable at $x$ and if $f$ is differentiable at $g(x)$, then the composite function $F := f \circ g$ defined by $F(x) := f(g(x))$ is differentiable at $x$ and $F'$ is given by the product: 
	\begin{align}
		F'(x) = f'(g(x)) \cdot g'(x)	
	\end{align}
	or, in Leibnitz notation, 
	\begin{align}
		\frac{dy}{dx} = \frac{dy}{du} \frac{du}{dx}	
	\end{align}
	\item \dfn{Method of Implicit Differentiation}: Differentiating both sides of the equation with respect to $x$, and then solving the resulting equation for $y'$. 
	\begin{itemize}[noitemsep]
		\item Application: finding the derivatives of inverse trigonometric functions: 
		\begin{itemize}[noitemsep]
			\item $\deriv{\arcsin(x)} = \frac{1}{\sqrt{1-x^2}}$, $\deriv{\arcsec(x)} = \frac{-1}{x \sqrt{x^2 - 1}}$
			\item $\deriv{\arccos(x)} = \frac{-1}{\sqrt{1-x^2}}$, $\deriv{\arcsec(x)} = \frac{1}{x \sqrt{x^2-1}}$ 
			\item $\deriv{\arctan(x)} = \frac{1}{1+x^2}$, $\deriv{\arccot(x)} = \frac{-1}{x^2 +1}$ 
		\end{itemize}
		\item Application: derivatives of logarithmic functions, $y = \log_b(x)$ and $y = \ln(x)$. 
		\begin{itemize}[noitemsep]
			\item $\deriv{ \log_b(x)} = \frac{1}{x \ln (b)}$
			\item $\deriv{ \ln (x) } = \frac{1}{x}$
		\end{itemize}
		\item \dfn{Method of Logarithmic Differentiation}: the calculation of complex functions involving products, quotients, or powers can be simplified by taking logarithms. 
	\end{itemize}
	\item \dfn{Hyperbolic Trigonometric Functions}: hyperbolic functions $\sim$ hyperbola like trigonometric functions $\sim$ circle. They are defined as: 
	\begin{itemize}[noitemsep]
		\item $\sinh(x) := \frac{e^x - e^{-x}}{2}$, $\csch(x) := \frac{1}{\sinh(x)}$
		\item $\cosh(x) := \frac{e^x + e^{-x}}{2}$, $\sech(x) := \frac{1}{\cosh(x)}$
		\item $\tanh(x) := \frac{\sinh(x)}{\cosh(x)}$, $\coth(x) := \frac{\cosh(x)}{\sinh(x)}$
		\item Applications: whenever an entity such as light, velocity, electricity, or radioactivity is gradually absorbed or extinguished. 
		\item Hyperbolic identities: 
		\begin{itemize}
			\item $\sinh(-x) = - \sinh(x)$, $\cosh(-x) = \cosh(x)$ 
			\item $\cosh^2(x)  - \sinh^2(x) = 1$, $1- \tanh^2(x) = \sech^2(x)$ 
			\item $\sinh(x+y) = \sinh(x) \cosh(y) + \cosh(x) \sinh(y)$
			\item $\cosh(x+y) = \cosh(x) \cosh(y) + \sinh(x) \sinh(y)$
		\end{itemize}
		\item Derivatives of Hyperbolic Functions: 
		\begin{itemize}[noitemsep]
			\item $\deriv{\sinh(x)} = \cosh(x)$, $\deriv{\csch(x)} = - \csch(x) \coth(x)$, 
			\item $\deriv{\cosh(x)} = \sinh(x)$, $\deriv{\sech(x)} = - \sech(x) \tanh(x)$, 
			\item $\deriv{\tanh(x)} = \sech^2(x)$, $\deriv{\coth(x)} = - \csch^2(x)$. 
		\end{itemize}
		\item \dfn{Inverse Hyperbolic Functions}: 
		\begin{itemize}[noitemsep]
			\item $\arcsinh(x) := \ln(x + \sqrt{x^2 + 1})$ for $x \in \R$. 
			\item $\arccosh(x) := \ln ( x + \sqrt{x^2 -1})$ for $x \geq 1$. 
			\item $\arctanh(x) := \frac{1}{2} \ln \left( \frac{1+x}{1-x} \right)$ for $x \in [-1, 1]$ 
		\end{itemize}
		\item Derivatives of Inverse Hyperbolic Functions: 
		\begin{itemize}[noitemsep]
			\item $\deriv{\arcsinh(x)} = \frac{1}{\sqrt{1+x^2}}$, $\deriv{ \arccsch(x)} = \frac{-1}{|x| \sqrt{x^2 +1}}$
			\item $\deriv{\arccosh(x)} = \frac{1}{\sqrt{x^2 - 1}}$, $\deriv{\arcsech(x)} = \frac{-1}{x \sqrt{1-x^2}}$
			\item $\deriv{\arctanh(x)} = \frac{1}{1-x^2}$, $\deriv{ \arccoth(x) } = \frac{1}{1-x^2}$
		\end{itemize}	
	\end{itemize}
\end{itemize}

\subsection{Applications of Differentiation}
\begin{itemize}[noitemsep]
	\item \dfn{Extreme Value Theorem}: Let $f$ be continuous on the closed and bounded interval $[a,b]$. Then, $f$ attains an absolute maximum value $f(x)$ and an absolute minimum value $f(d)$ at some numbers $c, d \in [a,b]$.
	\item \dfn{Fermat's Theorem}: If $f$ has a local maximum or minimum at $c$, and if $f'9x)$ exists, then $f'(x) =0$. 
	\item \dfn{Closed Interval Method}: To find the absolute maximum and minimum values of a continuous function $f$ on a closed interval $[a,b]$, 
	\begin{enumerate}[noitemsep]
		\item Find the values of $f$ at the critical points of $f$ in the open interval $]a,b[$. 
		\item Compute $f(a)$ and $f(b)$. 
		\item The max between (1) and (2) is the absolute max; the min between (1) and (2) is the absolute min. 
	\end{enumerate}
	\item \dfn{Rolle's Theorem}: Let $f: [a,b] \rightarrow \R$ satisfy: 
	\begin{enumerate}[noitemsep]
		\item $f$ is continuous on $[a,b]$
		\item $f$ is differentiable on $]a,b[$
		\item $f(a) = f(b)$. 
	\end{enumerate}
	Then, there exists a number $c \in ]a,b[$ such that $f'(c) = 0$. 
	\item \dfn{Mean Value Theorem}: Let $f: [a,b] \rightarrow \R$ satisfy: 
	\begin{enumerate}[noitemsep]
		\item $f$ is continuous on $[a,b]$ 
		\item $f$ is differentiable on $]a,b[$
	\end{enumerate}
	Then, there exists a number $c \in ]a,b[$ such that
	\begin{align}
		f'(c) = \frac{f(b) - f(a)}{b-a} \iff f(b) -f(a) = f'(c) [b-a]	
	\end{align}
	\item \dfn{Theorem} (Consequence of MVT): If $f'(x) =0$ $\forall x \in ]a,b[$, then $f$ is constant on $]a,b[$. 
	\begin{itemize}[noitemsep]
		\item \dfn{Corollary}: If $f'(x) = g'(x)$ $\forall x \in ]a,b[$, then $f-g$ is constant on $]a,b[$, i.e., $\exists c \in \R$ such that $f(x) = g(x) +c$. 
	\end{itemize}
	\item \dfn{L'Hopital's Rule}: Suppose $f$ and $g$ are differentiable and that $g(x) \neq 0$ on an open interval containing $a$. Suppose that
	\begin{align}
		\lim_{x \rightarrow a} f(x) = 0 \text{ and } \lim_{x \rightarrow a} g(x) = 0 	
	\end{align}
	or
	\begin{align}
		\lim_{x \rightarrow a} f(x) = \pm \infty \text{ and } \lim_{x \rightarrow a} f(x)  = \pm \infty 	
	\end{align}
	then
	\begin{align}
		\lim_{x \rightarrow a} \left( \frac{f(x)}{g(x)} \right) = \lim_{ x \rightarrow a} \left( \frac{f'(x)}{g'(x)}	\right) 
	\end{align}
	\item \dfn{Antiderivative}: A function $F$ is called an \dfn{anti-derivative} of $f$ on an interval $I$ if $F'(x) = f(x)$ $\forall x \in I$. 
\end{itemize}

\subsection{Integrals}
\begin{itemize}[noitemsep]
	\item \dfn{Area}: The \dfn{area} $A$ of a region $S$ that lies under the graph of a continuous function $f$ is the limit of the sum of the approximating rectangles 
	\begin{align}
		A = \lim_{n \rightarrow \infty} R_n = \lim_{n \rightarrow \infty} [ f(x_1) \Delta x + ...  + f(x_n) \Delta x ]	
	\end{align}
	\item \dfn{Definite Integral}: Let $f: [a,b] \rightarrow \R$. Divide $[a,b]$ into $n$ subintervals of equal width $\Delta x = \frac{b-a}{n}$. Let $a = x_0 < x_1 < ... < x_n =b$ be the endpoints and let $x_1^*, ..., x_n^*$ be any sample points in these subintervals such that $x_i^*  \in [x_{i-1}, x_i]$. Then, the definite integral of $f$ from $a$ to $b$ is: 
	\begin{align}
		\int_a^b f(x) dx := \lim_{n \rightarrow \infty} \sum_{i=1}^n f(x_i^*) \Delta x	
	\end{align}
	provided that the limit exists and is the same for all possible choices of sample points. If it does exist, then we say that $f$ is \dfn{integrable} on $[a,b]$.
	\item Formulae for the sums of positive integers: 
	\begin{itemize}[noitemsep]
		\item $ \sum_{i=1}^n i = \frac{n(n+1)}{2} $
		\item $ \sum_{i=1}^n i^2 = \frac{n(n+1)(2n+1)}{6} $
		\item $ \sum_{i=1}^n i^3 = \left[  \frac{n(n+1)}{2} \right]^2 $
	\end{itemize}
	\item \dfn{Fundamental Theorem of Calculus} connects differential calculus and integral calculus. Deals with equations of the form
	\begin{align}
		g(x) = \int_a^x f(t) dt 	
	\end{align}
	\begin{itemize}[noitemsep]
		\item \dfn{Fundamental Theorem of Calculus Part 1}: let $f: [a,b] \rightarrow \R$ be continuous on $[a,b]$. Then, the function $g$ defined by
		\begin{align}
			g(x) := \int_a^x f(t) dt 	
		\end{align}
		is continuous on $[a,b]$ and differentiable on $]a,b[$. Moreover, $g'(x) = f(x)$. 
		\item \dfn{Fundamental Theorem of Calculus Part 2}: If $f$ is continuous on $[a,b]$, then 
		\begin{align}
			\int_a^b f(x) dx = F(b) - F(a) 	
		\end{align}
		where $F$ is any anti-derivative of $f$. 
		\item Alternative expression for the FoC Part 1: 
		\begin{align}
			\deriv{ \int_a^x f(t) dt } = f(x) 	
		\end{align}
	\end{itemize}
	\item \dfn{Table of Integration Formulae}: 
	\begin{itemize}
\begin{minipage}{0.4\linewidth}
    \item $\int x^n dx = \frac{x^{n+1}}{n+1}$ for $n \neq -1$. 
    \item $\int e^x dx = e^x$
    \item $\int \sin (x) dx = - \cos(x) $
    \item $\int \sec^2(x) dx = \tan (x) $
    \item $\int \sec (x) \tan (x) dx = \sec (x) $
    \item $\int \sec (x) dx = \ln | \sec (x) + \tan (x) | $
    \item $\int \tan (x) dx = \ln | \sec (x) |$
    \item $\int \sinh (x) dx = \cosh (x) $
    \item $\int \frac{1}{x^2 + a^2} dx = \frac{1}{a} \arctan \left( \frac{x}{a} \right)$ 
    \item $\int \frac{1}{x^2 - a^2} dx = \frac{1}{2a} \ln \left|  \frac{ x-a}{x+a} \right| $
\end{minipage}
\begin{minipage}{0.4\linewidth}
    \item $\int \frac{1}{x} dx = \ln |x |$
    \item $\int b^x dx = \frac{b^x}{\ln b}$
    \item $\int \cos (x) dx = \sin (x) $
    \item $\int \csc^2 (x) dx = - \cot (x) $
    \item $\int \csc (x) \cot (x) dx = - \csc (x) $ 
    \item $\int \csc (x) dx = \ln | \csc (x) - \cot (x) | $
    \item $\int \cot (x) dx = \ln \left| \sin (x) \right| $
    \item $\int \cosh(x) dx = \sinh(x)$
    \item $\int \frac{1}{\sqrt{a^2 - x^2}} dx = \arcsin \left( \frac{x}{a} \right) $ $a > 0$
    \item $\int \frac{1}{\sqrt{ x^2 \pm a^2}} dx = \ln | x + \sqrt{x^2 \pm a^2 }| $
\end{minipage}
\end{itemize}
\end{itemize}
\subsection{Applications of Integration}
\begin{itemize}[noitemsep]
	\item  The \dfn{average value} of $f$ on the interval $[a,b]$ is: 
	\begin{align}
		f_{avg} := \frac{1}{b-a} \int_a^b f(x) dx 	
	\end{align}
	\item \dfn{Mean Value Theorem for Integrals}: If $f$ is continuous on $[a,b]$ then there exists a $c \in [a,b]$ such that
	\begin{align}
		f(x) = f_{avg} = \frac{1}{b-a} \int_a^b f(x) dx \iff \int_a^b f(x) dx = f(c) (b-a) 	
	\end{align}
\end{itemize}

\subsection{Integration Techniques}
\begin{itemize}[noitemsep]
	\item \dfn{Integration by Parts}: 
	\begin{align}
		\int f(x) g'(x) dx = f(x) g(x) - \int g(x) f'(x) dx 	
	\end{align}
	\item \dfn{Trigonometric Integrals}: 
	\begin{enumerate}[noitemsep]
		\item Strategy for evaluating $\int \sin^m(x) \cos^n (x) dx$: 
		\begin{enumerate}[noitemsep]
			\item If $n$ is odd: save one cosine, use $\cos^2(x) = 1-\sin^2(x)$ to express the remaining factors in terms of sine: 
			\begin{align}
				\int \sin^m(x) \cos^{2k+1}(x) dx = \int \sin^m(x) (1-\sin^2(x))^k \cos(x) dx 	
			\end{align}
			and make the substitution $u=\sin(x)$. 
			\item If $m$ is odd: save one sine, use $\sin^2(x) = 1 - \cos^2(x)$ to express the remaining factors in terms of cosine: 
			\begin{align}
				\int \sin^{2k+1}(x) \cos^n (x) dx = \int (1-\cos^2(x))^k \cos^n (x) \sin (x) dx 	
			\end{align}
			and make the substitution $u = \cos(x)$.
			\item If sine and cosine are even, then use the half-angle identities: 
			\begin{align}
				\sin^2(x) = \frac{1}{2} ( 1 - \cos (2x)) \text{ and } \cos^2(x) = \frac{1}{2} ( 1 + \cos(2x)) 	
			\end{align}
			A helpful identity is $\sin (x) \cos(x) = \frac{1}{2} \sin (2x)$. 
		\end{enumerate}
		\item Strategy for evaluating $\int \tan^m (x) \sec^n(x) dx$: 
		\begin{enumerate}[noitemsep]
			\item If $n$ is even: save one secant squared, use the identity $\sec^2(x) = 1 + \tan^2(x)$ to express the remaining factors in terms of $\tan (x)$: 
			\begin{align}
				\int \tan^m(x) \sec^{2k} (x) dx = \int \tan^m(x) (1+\tan^2(x))^{k-1} \sec^2(x) dx 	
			\end{align}
			and make the substitution $u= \tan(x)$. 
			\item If $m$ is odd: save one $\sec(x) \tan(x)$, use $\tan^2(x) = \sec^2(x) - 1$ to express the remaining factors in terms of $\sec(x)$: 
			\begin{align}
				\int \tan^{2k+1} (x) \sec^n(x) dx = \int ( \sec^2(x) - 1)^k \sec^{n-1}(x) \sec(x) \tan(x) dx 	
			\end{align}
			substitute $u= \sec(x)$. 
		\end{enumerate}
	\item Important product identities to remember: 
	\begin{enumerate}[noitemsep]
		\item $\sin (A) \cos (B) = \frac{1}{2} [ \sin (A-B) + \sin (A+B) ]$
		\item $\sin (A) \sin (B) = \frac{1}{2} [ \cos (A-B) + \cos (A+B) ]$ 
		\item $\cos (A) \cos (B) = \frac{1}{2} [ \cos (A-B) + \cos (A+B) ]$
	\end{enumerate}
	\end{enumerate}
	\item \dfn{Trigonometric Substitution} 
	\begin{itemize}[noitemsep]
		\item $\sqrt{a^2 - x^2} \rightarrow x = a \sin ( \theta)$, $\theta \in \left[ - \frac{\pi}{2}, \frac{\pi}{2} \right]$ $\rightarrow$ $1 - \sin^2 (\theta ) = \cos^2 (\theta)$. 
		\item $\sqrt{a^2 + x^2} \rightarrow x = a \tan ( \theta)$, $ \theta \in \left] - \frac{\pi}{2}, \frac{\pi}{2} \right[$ $\rightarrow 1 + \tan^2 (\theta) = \sec^2 (\theta)$. 
		\item $\sqrt{x^2 - a^2} \rightarrow x = a \sec (\theta)$, $ \theta \in \left[ 0, \frac{\pi}{2} \right[ \cup \left[ \pi, \frac{3 \pi}{2} \right[$ $\rightarrow \sec^2 (\theta) - 1 = \tan^2(\theta)$. 
	\end{itemize}
	\item \dfn{Partial Fractions}: 
	\begin{enumerate}[noitemsep]
		\item Case I: Denominator $Q(x)$ is a product of distinct linear factors: 
		\begin{align}
			\frac{R(x)}{Q(x)} = \frac{A_1}{a_1x+b_1} + ... + \frac{A_k}{a_k x + b_k}	
		\end{align}
		\item Case II: Denominator $Q(x)$ is a product of linear factors, some of which are repeated $r$ times: 
		\begin{align}
			\frac{R(x)}{Q(x)} = \frac{A_1}{a_1 x + b_1}	 + ... + \frac{A_k}{(a_1 x + b_1)^r}
		\end{align}
		\item Case III: $Q(x)$ contains irreducible quadratic factors, none of which is repeated. Then, expression will have a term of the form
		\begin{align}
			\frac{Ax + B}{ax^2 + bx + c}	
		\end{align}
		which can be integrated by completing the square and using the formula: 
	\begin{align}
		\int \frac{1}{x^2 + a^2} dx = \frac{1}{a} \arctan \left( \frac{x}{a} \right) + C	
	\end{align}
	\item Case IV: $Q(x)$ contains a repeated irreducible factor. Then, the expression will have a term of the form: 
	\begin{align}
		\frac{A_1 x + B_1}{ax^2 + bx + c}	+ .... + \frac{
		A_r x + B_r}{(ax^2 + bx + c)^r}
	\end{align}
	\end{enumerate}
	\item General Strategy for Integrating: 
	\begin{enumerate}[noitemsep]
		\item Simplify the integrand if possible using algebraic manipulation and trigonometric identities. 
		\item Look for obvious substitutions. 
		\item Classify integrand according to its form
		\begin{enumerate}[noitemsep]
			\item Trigonometric functions
			\item Rational functions ($\rightarrow$ partial fractions) 
			\item Integration by parts
			\item Radicals
			\begin{enumerate}[noitemsep]
				\item $\sqrt{ \pm x^2 \pm a^2}$ $\rightarrow$ trigonmetric substitution 
				\item $(ax+b)^{1/n}$ $\rightarrow$ rationalising substitution $u = (ax+b)^{1/n}$
			\end{enumerate}
		\end{enumerate}
	\end{enumerate}
	\item \dfn{Improper Integral}: if in the definite integral, $\int_a^n f(x) dx$, either $[a,b]$ is an unbounded interval or $f(x)$ has an infinite discontinuity in $[a,b]$
\end{itemize}

\subsection{Further Applications of Integration}
\begin{itemize}[noitemsep]
	\item \dfn{Arc-length formula}: If $f'$ is continuous on $[a,b]$, then the length of the curve $y=f(x)$, $a \leq x \leq b$ is: 
	\begin{align}
		L = \int_a^b \sqrt{1+ [ f'(x)]^2}	
	\end{align}
	\item \dfn{Arc-Length Function}: If a smooth curve $C$ has the equation $y=f(x)$, $ a \leq x \leq b$, let $s(x)$ be the distance along $C$ from the initial point $P_0(a, f(a))$ to the point $Q(x, f(x))$. Then, $s$ is a function given by: 
	\begin{align}
		s(x) = \int_a^x \sqrt{1+ [f'(t)]^2} dt 	
	\end{align}
\end{itemize}
\subsection{Parameter Equations and Polar Coordinates}
\textbf{Motivation:} some curves are best handeled when both $x$ and $y$ are given as a function of a third variable $t$: $x=f(t)$, $y=g(t)$. 
\begin{itemize}
	\item Suppose $f$, $g$ are differentiable functions and suppose we want to find the tangent line at a point on the parametric curve $x=f(t)$, $y=g(t)$, where $y$ is also a differentiable function of $x$. If $\frac{dx}{dt} \neq 0$, then the slope of the parametric curve is given by: 
	\begin{align}
		\frac{dy}{dx} = \frac{\frac{dy}{dt}}{\frac{dx}{dt}}	
	\end{align}
	\begin{itemize}[noitemsep]
		\item We can consider $\frac{d^2 y}{d^2x}$: 
		\begin{align}
			\frac{d^2 y}{dx^2} = \frac{d}{dx} \left( \frac{dy}{dx} \right) = \frac{\frac{d}{dt} \left( \frac{dy}{dx} \right)}{\frac{dx}{dt}}	
		\end{align}
	\item \dfn{Areas}: if a curve is traced out by the parametric equations $x = f(t)$ and $y=g(t)$ for $t \in [\alpha, \beta]$, then using the substitution rule for integrals one has the following formula: 
	\begin{align}
		\mathcal{A} = \int_a^b y dx = \int_\alpha^\beta g(t) f'(t) dt	
	\end{align}
	\item \dfn{Arc Length} if a curve $C$ is described by parametric equations $x = f(t)$, $y= g(t)$, $\alpha \leq t \leq \beta$, where $f'$ and $g'$ are continuous on $[a,b]$ and $C$ is traversed exactly once as $t$ travels from $\alpha$ to $\beta$, then the length of $C$ is: 
	\begin{align}
		L = \int_\alpha^\beta \sqrt{ \left( \frac{dx}{dt} \right)^2 + \left( \frac{dy}{dt} \right)^2} dt 	
	\end{align}
	\item \dfn{Surface Area}: similar to the conditions in the previous theorem, the surface area of a curve obtained by rotating it about the $x$-axis is given by: 
	\begin{align}
		S = \int_\alpha^\beta 2 \pi y \sqrt{ \left( \frac{dx}{dt} \right)^2 + \left( \frac{dy}{dt} \right)^2}	
	\end{align}

	\end{itemize}
	\item Equations to convert between cartesian and polar coordinates: 
	\begin{align}
		x = r \cos (\theta)\ & y = r \sin (\theta) \\
		r^2 = x^2 + y^2\ & \tan ( \theta) = \frac{y}{x}	
	\end{align}
	\item If $r = f(\theta)$ is a polar curve, then we can find the tangent line to a polar curve by regarding $\theta$ as a parameter: 
	\begin{align*}
			& x = f(\theta) \cos ( \theta) \\
			& y = f(\theta) \sin ( \theta) 
	\end{align*}
	and the tangent is given by: 
	\begin{align}
		\frac{dy}{dx} = \frac{\frac{dy}{d \theta}}{\frac{dx}{d \theta}} = \frac{\frac{dr}{d \theta} \sin (\theta) + r \cos ( \theta)}{\frac{dr}{d \theta} \cos ( \theta) - r \sin (\theta)}
	\end{align}
	\item Areas and lengths in polar coordinates
	\begin{itemize}[noitemsep]
		\item The area of a sector of a circle: $A = \frac{1}{2} r^2 \theta$. The area of a polar region $\mathcal{R}$: 
		\begin{align}
			A(\mathcal{R}) = \int_a^b \frac{1}{2} [ f(\theta) ]^2 d \theta 
		\end{align}
		\item The arc-length of a polar curve with the equation $r = f( \theta)$, $ a \leq \theta \leq b$ is: 
		\begin{align}
			L = \int_a^b \sqrt{r^2 + \left( \frac{dr}{d \theta} \right)^2} d \theta 	
		\end{align}
	\end{itemize}
\end{itemize}

\subsection{Infinite Sequences and Series}

\begin{itemize}[noitemsep]
	\item A \dfn{Sequence} is a list of numbers written in a definite order
	\begin{align}
		a_1, a_2, a_3, ....	
	\end{align}
	\item A sequence has a \dfn{limit} $L$ and we write
	\begin{align*}
		\lim_{n \rightarrow \infty} a_n = L 
	\end{align*}
	if we can make the terms $a_n$ as close to $L$ as we'd like by taking $n$ sufficiently large. if $\lim_{n \rightarrow \infty} a_n$ exists, then we say that $\{ a_n \}$ is \dfn{convergent}. Else, it is \dfn{divergent}. 
	\begin{itemize}
		\item A sequence $\{ a_n \}$ has a limit $L$ if $\forall$ $\varepsilon > 0$, $\exists$ an $N \in \mathbb{N}$ such that $\forall n \geq N$, one has that $|a_n - L| < \varepsilon$. 
	\end{itemize}
	\item \dfn{Squeeze Theorem}: if $a_n \leq b_n \leq c_n$ $\forall n \geq n_0$, and if $\lim_{n \rightarrow \infty} a_n = \lim_{n \rightarrow \infty} c_n =L$, then $\lim_{n \rightarrow \infty} b_n =L$. 
	\begin{itemize}[noitemsep]
		\item If $\lim_{n \rightarrow \infty} |a_n| = 0$, then $\lim_{n \rightarrow \infty} a_n = 0$. 
		\item If $\lim_{n \rightarrow \infty} a_n = L$ and if $f$ is a continuous function at $L$, then
		\begin{align*}
			\lim_{n \rightarrow \infty} f(a_n) = f(L) 
		\end{align*}
	\end{itemize}
	\item The sequence $\{ r^n \}$ is convergent if $r \in ]-1, 1]$ and divergent for all other values of $r$. If $r \in ]-1, 1]$: 
	\begin{align*}
		\lim_{n \rightarrow \infty} r^n = \begin{cases}
			0 & \text{ if } r \in ]-1, 1[ \\
			1 & \text{ if } r = 1
		\end{cases}
	\end{align*}
	\item \dfn{Monotonic Sequence Theorem}: every bounded, monotonic sequence converges. 
	\item \dfn{Series}: Given a series
	\begin{align*}
		\sum_{n=1}^\infty a_n = a_1 + a_2 + a_3 + a_4 + ... 
	\end{align*}
	let $s_n$ denote the $n$th partial sum: 
	\begin{align*}
		s_n := \sum_{i=1}^n a_i = a_1 + ... + a_n 
	\end{align*}
	if the \emph{sequence} $\{ s_n \}$ is convergent and $\lim_{n \rightarrow \infty} s_n = s \in \R$, then the series $\sum_{n=1}^\infty a_n$ is convergent. 
	\item \dfn{Geometric Series}: is an important example of an infinite series.
	\begin{align}
		a + ar + ar^2 + ... + ar^{n-1} + ... = \sum_{n=1}^\infty ar^{n-1} \text{ ($a \neq 0$)} 	
	\end{align}
	\begin{itemize}[noitemsep]
		\item If $r=1$, then $s_n = na \rightarrow \infty$. 
		\item If $r \neq 1$, then $s_n = \frac{a(1-r^n)}{1-r}$. If $ r \in ]-1, 1[$, then $r^n \rightarrow 0 $ as $n \rightarrow \infty$, and so 
		\begin{align}
			\lim_{n \rightarrow \infty} s_n = \frac{a}{1-r}	
		\end{align}
		otherwise the geometric series diverges. 
	\end{itemize}
	\item \dfn{Harmonic Series} is defined as $\sum_{n=1}^\infty \frac{1}{n}$. It's divergent. 
	\item If the series $\sum_{n=1}^\infty a_n$ is convergent, then $\lim_{n \rightarrow \infty} a_n =0$. 
	\begin{itemize}
		\item \dfn{Test for Divergence}: if $\lim_{n \rightarrow \infty} a_n$ does not exist or if $\lim_{n \rightarrow \infty} a_n \neq 0$, then the series $\sum_{n=1}^\infty a_n$ is divergent.
	\end{itemize}
	\item \dfn{Integral Test}: Suppose $f$ is a continuous, positive, decreasing function on $[1, \infty[$ and let $a_n = f(n)$. Then, the series $\sum_{n=1}^\infty a_n$ is convergent $\iff$ the improper integral $\int_1^\infty f(x) dx $ is convergent. 
	\begin{itemize}
		\item \dfn{p-series}: the $p$-series $\sum_{n=1}^\infty \frac{1}{n^p}$ converges $\iff$ $p >1$. 
	\end{itemize}
	\item \dfn{Remainder Estimate for the Integral Test}: Suppose that $f(k) = a_k$, where $f$ is a continuous, decreasing, positive function for $x \geq n$ and suppose that $\sum a_n$ is convergent. If $R_n := S - S_n$, then
	\begin{align}
		\int_{n+1}^\infty f(x) dx \leq R_n \leq \int_n^\infty f(x) dx 	
	\end{align}
	\item \dfn{Comparison Test}: Suppose that $\sum a_n$ and $\sum b_n$ are series with positive terms. 
	\begin{enumerate}[noitemsep]
		\item If $\sum b_n$ is convergent and $a_n \leq b_n $ $\forall n$, then $\sum a_n$ converges. 
		\item If $\sum b_n$ is divergent and $b_n \leq a_n$ $\forall n$, then $\sum a_n$ diverges. 
	\end{enumerate}
	\item \dfn{Limit Comparison Test}: Suppose $\sum a_n$ and $\sum b_n$ are series with positive terms. If 
	\begin{align*}
		\lim_{n \rightarrow \infty} \frac{a_n}{b_n} = c
	\end{align*}
	where $c \in ]0, \infty[$, then both series have the same behaviour; i,e, either both series converge or both diverge. 
	\item \dfn{Alternating Series Test}: If the alternating series $\sum_{n=1}^\infty (-1)^{n-1} b_n$ for $b_n > 0$ satisfies: 
	\begin{enumerate}[noitemsep]
		\item $b_{n+1} \leq b_n$ $\forall n \in \mathbb{N}$
		\item $\lim_{n \rightarrow \infty} b_n = 0$
	\end{enumerate}
	then, the series converges. 
	\item \dfn{Absolutely Convergent}: A series $\sum a_n$ is called \dfn{absolutely convergent} if the series of absolute values $\sum |a_n|$ is convergent. 
	\item \dfn{Conditionally Convergent}: A series $\sum a_n$ is called \dfn{conditionally convergent} if its convergent but not absolutely convergent. 
	\item If $\sum a_n$ is absolutely convergent then it is convergent.
	\item \dfn{Ratio Test} Let $\{ a_n \}$ be a sequence.
	\begin{enumerate}[noitemsep]
		\item If $\lim_{n \rightarrow \infty} \left| \frac{a_{n+1}}{a_n} \right| = L < 1$, then $\sum_{n=1}^\infty a_n $ absolutely converges (and thus converges).
		\item If $\lim_{n \rightarrow \infty} \left| \frac{a_{n+1}}{a_n} \right| = L > 1$ or $\lim_{n \rightarrow \infty} \left| \frac{a_{n+1}}{a_n} \right| - L = \infty $, then $\sum_{n=1}^\infty a_n $ diverges. 
		\item If $\lim_{n \rightarrow \infty} \left| \frac{a_{n+1}}{a_n} \right| = L = 1$ then the Ratio test is inconclusive. 
	\end{enumerate}
	\item \dfn{Root Test} Let $\{ a_n \}$ be a sequence. 
	\begin{enumerate}[noitemsep]
		\item If $\lim_{n \rightarrow \infty} \sqrt[n]{|a_n|} = L < 1$, then the series absolutely converges (and therefore converges). 
		\item If $\lim_{n \rightarrow \infty} \sqrt[n]{|a_n|} = L > 1$ or if $\lim_{n \rightarrow \infty} \sqrt[n]{|a_n|} = \infty$ then the series diverges. 
		\item If $\lim_{n \rightarrow \infty} \sqrt[n]{|a_n|} = L = 1$, then the root test will be inconclusive. 
	\end{enumerate}
\end{itemize}
\subsection{Strategy for Testing Series}
\begin{enumerate}[noitemsep]
	\item If the series is of the form $\sum \frac{1}{n^p}$, then apply the p-series rule. 
	\item If the series is of the form $\sum ar^{n-1}$ or $\sum ar^n$, then apply the geometric series rule. 
	\item If the series is similar to a p-series or a geometric series, then use a comparison test. 
	\item If $\lim_{n \rightarrow \infty} a_n \neq 0$, use the divergence test to conclude that the series diverges. 
	\item If the series is of the form $\sum (-1)^{n-1} b_n$ or $\sum (-1)^n b_n$, then use the alternating series test. 
	\item If the series has factorials in it, consider applying a ratio test. 
	\item If the series is of the form $(b_n)^n$, then consider the root test. 
	\item If $a_n = f(n)$ where $\int_1^\infty f(x) dx$ is easily evaluated, then consider the integral test.
\end{enumerate}

\subsection{Power Series}
\begin{itemize}[noitemsep]
	\item \dfn{Power Series}: a power series is of the form 
	\begin{align}
		\sum_{n=0}^\infty c_n x^n = c_0 + c_1 x + c_2 x^2 + ... 	
	\end{align}
	A series of the form 
	\begin{align}
		\sum_{n=0}^\infty c_n (x-a)^n = c_0 + c_1 (x-a) + c_2 (x-a)^2 
	\end{align}
	is called a \dfn{power series in $(x-a)$}
	\item \dfn{Theorem}: for a given power series $\sum_{n=0}^\infty c_n (x-a)^n$, there are only three possibilities: 
	\begin{enumerate}[noitemsep]
		\item The series converges only when $x=a$. 
		\item The series converges for all $x$. 
		\item $\exists$ an $R>0$ such that the series converges if $|x-a| < R$ and diverges if $|x-a| > R$. 
	\end{enumerate}
	\item \dfn{Theorem (Term-by-term Differentiation and Integration)}: If the power series $\sum c_n (x-a)^n$ has a radius of convergence $R>0$, then the function defined by: 
	\begin{align}
		f(x) := c_0 + c_1 (x-a) + c_2 (x-a)^2 + ... = \sum_{n=0}^\infty c_n (x-a)^n 	
	\end{align}
	is differentiable (and thus continuous) on the interval $]a-R, a+R[$ and:
	\begin{enumerate}[noitemsep]
		\item $f'(x) = c_0 + 2c_s(x-a) + 3c_3 (x-a)^2 + ... = \sum_{n=1}^\infty nc_n (x-a)^{n-1}$. 
		\item $\int f(x) dx = c + c_0 (x-a) + c_a \frac{(c-a)^2}{2} + c_2 \frac{(x-a)^3}{3} + .... $
		\begin{align}
			\int f(x) dx = c + \sum_{n=0}^\infty c_n \frac{(x-a)^{n+1}}{n+1}	
		\end{align}
	The radius of convergence of both (1) and (2) remain $R$. 
	\end{enumerate}
	\item \dfn{Theorem (Taylor Series Representation)}: If $f$ has a power series representation (expansion) at $a$, that is, if 
	\begin{align}
		f(x) = \sum_{n=0}^\infty c_n (x-a)^n 	
	\end{align}
	for $|x-a|<R$, then the coefficients are given by the formula: 
	\begin{align}
		c_n = \frac{f^{(n)}(a)}{n!	}	
	\end{align}
	Then, the \dfn{Taylor Series} for $f$ about $a$ is given by: 
	\begin{align}
		f(x) = \sum_{n=0}^\infty \frac{f^{(n)} (a)}{n!} (x-a)^n	
	\end{align}
	For the special case of $a=0$, then the above becomes: 
	\begin{align}
		f(x) = \sum_{n=0}^\infty \frac{f^{(n)} (0)}{n!}x^n = f(0) + \frac{f'(0)}{1!} x + \frac{f''(0)}{2!} x^2 + ... 	
	\end{align}
	\item \dfn{Theorem (Remainder)}: Define the remainder of the Taylor series by $R_n := f(x) - T_n(x)$. If $f(x) = T_n(x) + R_n(x)$, where $T_n$ is the nth degree Taylor polynomial of $f$ at $a$ and
	\begin{align}
		\lim_{n \rightarrow \infty} R_n(x) = 0 	
	\end{align}
	for $|x-a| < R$, then $f$ is equal to the sum of its Taylor series on the interval $|x-a|<R $.
	\item The following theorem is often used when trying to show that $\lim_{n \rightarrow \infty} R_n =0$ for a specific function $f$: 
	\begin{itemize}[noitemsep]
		\item \dfn{Taylor's Inequality}: If $|f^{(n+1)}(x)| \leq M$ $\forall $ $|x-a| \leq d$, then the remainder $R_n(x)$ of the Taylor series satisfies the following inequality: 
		\begin{align}
			|R_n(x) | \leq \frac{M}{(n+1)!}	|x-a|^{n+1} 
		\end{align}
		for $|x-a| \leq d$. 
	\end{itemize}
	\item \dfn{Important MacLaurin Series and their Radii of Convergence}: 
	\begin{align}
		\frac{1}{1-x} & = \sum_{n=0}^\infty x^n = 1 + x + x^2 + x^3 + ... \text{ ($R=1$) } \\
		e^x & = \sum_{n=0}^\infty \frac{x^n}{n!} = 1 + \frac{x}{1!} + \frac{x^2}{2!} + \frac{x^3}{3!} + .... \text{ ($R= \infty$)} \\
		\sin (x) &  = \sum_{n=0}^\infty (-1)^n \frac{x^{2n+1}}{(2n+1)!} = x - \frac{x^3}{3!} + \frac{x^5}{5!} - \frac{x^7}{7!} _ ... \text{ ($R =\infty$)} \\
		\cos (x) & = \sum_{n=0}^\infty (-1)^n \frac{x^{2n}}{(2n)!} = 1 - \frac{x^2}{2!} + \frac{x^4}{4!} - \frac{x^6}{6!} + ... \text{ ($R=\infty$) } \\
		\arctan(x) & = \sum_{n=0}^\infty \frac{x^{2n+1}}{(2n+1)} = 1 - \frac{x^3}{3} + \frac{x^5}{5!} - \frac{x^7}{7!} + ... \text{ ($R =1$)} \\
		\ln (1+x) & = \sum_{n=1}^\infty (-1)^{n-1} \frac{x^n}{n} = x - \frac{x^2}{2} + \frac{x^3}{3} - \frac{x^4}{4} + ... \text{ ($R=1$) } \\
		(1+x)^k & = \sum_{n=0}^\infty \binom{k}{n} x^n = 1 + kx + \frac{k(k-1)}{2!}x^3 + \frac{k(k-1)(k-2)}{3!}x^3 + ... \text{ ($R=1$)} 
	\end{align}

\end{itemize}

\section{Multi Variable Calculus}
\subsection{Vectors and Geometry of Space}
\begin{itemize}[noitemsep]
	\item If $\theta$ is the angle between vectors $\mathbf{a}$ and $\mathbf{b}$, then $\mathbf{a} \cdot \mathbf{b} = ||\mathbf{a}|| || \mathbf{b} || \cos (\theta)$ 
	\begin{itemize}[noitemsep]
		\item Two vectors are orthogonal $\iff$ $\mathbf{a} \cdot \mathbf{b} = 0$. 
		\item \dfn{Scalar Projection of $\mathbf{b}$ onto $\mathbf{a}$}: 
		\begin{align}
			\text{comp}_{ \mathbf{a}} ( \mathbf{b}) : \frac{\mathbf{a} \cdot \mathbf{b} }{||\mathbf{a}|| }	
		\end{align}
		\item \dfn{Vector projection of $\mathbf{b}$ onto $\mathbf{a}$}: 
		\begin{align}
			\text{proj}_{\mathbf{a}} (\mathbf{b}) 	:= \left( \frac{\mathbf{a} \cdot \mathbf{b}}{|| \mathbf{a}|| } \right) \frac{\mathbf{a}}{|| \mathbf{a}||}
		\end{align}
	\end{itemize}
	\item \dfn{Cross Product}: if $\mathbf{a} := \langle a_1, a_2, a_3 \rangle$ and $\mathbf{b} := \langle b_1, b_2, b_3 \rangle$, then their \dfn{cross product} is:
	\begin{align}
		\mathbf{a} \times \mathbf{b} := \det \begin{bmatrix}
			\mathbf{\hat{i}} & \mathbf{\hat{j}} & \mathbf{\hat{k}} \\
			a_1 & a_2 & a_3 \\
			b_1 & b_2 & b_3 
		\end{bmatrix}	
	\end{align}	
	\begin{itemize}[noitemsep]
		\item If $\theta$ is the angle between $\mathbf{a}$ and $\mathbf{b}$, then 
		\begin{align}
			||\mathbf{a} \times \mathbf{b} || := || \mathbf{a}|| || \mathbf{b} || \sin (\theta) 	
		\end{align}
		\item Two non-zero vectors $\mathbf{a}$ and $\mathbf{b}$ are parallel $\iff$ $\mathbf{a} \times \mathbf{b} = 0$. 
		\item The volume of the parallelepiped spanned by the vectors $\mathbf{a}$, $\mathbf{b}$, $\mathbf{c}$ is $| \mathbf{a} \cdot ( \mathbf{b} \times \mathbf{c} ) | $
	\end{itemize}
	\item A parametric equation for a line going through the point $(x_0, y_0, z_0)$ parallel to the direction vector $\langle a, b, c \rangle $ are: 
	\begin{align*}
		x & = x_0 + at \\
		y & = y_0 + bt \\
		z & = z_0 + ct 
	\end{align*}
	\item The scalar equation of the plane through the point $P_0(x_0, y_0, z_0) $ with the normal vector $\mathbf{n} := \langle a, b, c \rangle$ is: 
	\begin{align}
		a(x-x_0) + b(y-y_0) + c(z-z_0) = 0 	
	\end{align}
	\item The distance $D$ from a point $P_1(x_1, y_1, z_1)$ to the plane $ax+ by + cz + d =0$ is: 
	\begin{align}
		D = \frac{|ax_1 + by_1 + cz_1 + d |}{\sqrt{a^2 + b^2 + z^2}}	
	\end{align}
	\item \dfn{Vector-Valued Function}: a function whose domain is the set of real numbers and whose range is a set of vectors. 
	\begin{itemize}[noitemsep]
		\item The \dfn{limit} of a vector-valued function $\mathbf{r}$ is defined by taking the limits of the component functions as follows: 
		\begin{align}
			\lim_{ t \rightarrow a} \mathbf{r}(t) := \langle \lim_{t \rightarrow a} f(t), 	\lim_{t \rightarrow a} g(t), \lim_{t \rightarrow a} h(t) \rangle 
		\end{align}
	\item \dfn{Space-curve}: the set $C$ of all points $(x,y,z)$ in space where $x= f(t)$, $y=g(t)$, and $z = h(t)$, where $t \in I$, is called a \dfn{space-curve}. 
	\end{itemize}
	\item The \dfn{derivative} $\mathbf{r}'$ of a vector-valued function $\mathbf{r}$ is defined as: 
	\begin{align}
		\frac{d \mathbf{r}}{dt} = \mathbf{r}'(t) = \lim_{h \rightarrow 0} \frac{\mathbf{r}(t+h) - \mathbf{r}(t)}{h}	
	\end{align}
	\begin{itemize}
		\item The \dfn{unit tangent} vector $\mathbf{T}(t)$ is defined as:
		\begin{align}
				\mathbf{T}(t) = \frac{\mathbf{r}'(t)}{|| \mathbf{r}'(t) ||}
		\end{align}
		\item The \dfn{definite integral} of a vector-valued function is exactly what one would expect: 
		\begin{align}
			\int_a^b \mathbf{r}(t) dt = \left( \int_a^b f(t) dt \right) \mathbf{\hat{i}} + \left( \int_a^b g(t) dt \right) \mathbf{\hat{j}} + \left( \int_a^b h(t) dt \right) \mathbf{\hat{k}}  
		\end{align}
	\end{itemize}
	\item The length $L$ of a space-curve between the points $a$ and $b$ is parameterisation-independent and is given by:
	\begin{align}
		L = \int_a^b || \mathbf{r}'(t) || dt 	
	\end{align}
	\begin{itemize}[noitemsep]
		\item The \dfn{arc-length function} of a curve, $s$, is defined as: 
		\begin{align}
			s(t) := \int_a^t || \mathbf{r}'(u) || du 	
		\end{align}
		We can use the above equation to parameterise a curve with respect to arc-length by differentiating both sides of the equation above with respect to $t$ and applying the fundamental theorem of calculus: 
		\begin{align}
			\frac{ds}{dt} = || \mathbf{r}'(t) ||	
		\end{align}
		Advantages of an arc-length parametrisation include: it arises naturally from the shape of the curve and it's coordinate-system independent. 
	\end{itemize}
	\item \dfn{Smooth}: a parameterisation $\mathbf{r}(t)$ is called \dfn{smooth} on an interval $I$ if $\mathbf{r}'$ is continuous and $\mathbf{r}'(t) \neq 0$ on $I$. 
	\item \dfn{Curvature}: the \dfn{curvature} of a curve is given by: 
	\begin{align}
		\kappa := \left| \left| \frac{d \mathbf{T}}{ds} \right| \right| 	
	\end{align}
	where $\mathbf{T}$ is the unit tangent vector. We have three other formulae for curvature: 
	\begin{enumerate}[noitemsep]
		\item 
		\begin{align}
			\kappa(t) = \frac{|| \mathbf{T}'(t) ||}{|| \mathbf{r}'(t) ||}	
		\end{align}
		\item 
		\begin{align}
			\kappa (t) = \frac{|| \mathbf{r}'(t) \times \mathbf{r}''(t) ||}{|| \mathbf{r}'(t) ||^3}
		\end{align}
		\item For plane curves, write $\mathbf{r}(x) = x \mathbf{\hat{i}} + f(x) \mathbf{\hat{j}}$ 
		\begin{align}
				\kappa (x) = \frac{|f''(x)|}{[1+ (f'(x)^2]^{3/2} }
		\end{align}
	\end{enumerate}
	\item When $\kappa(t) \neq 0$, one can define the \dfn{principle unit normal} $\mathbf{N}(t)$: 
	\begin{align}
		\mathbf{N}(t) = \frac{\mathbf{T}'(t)}{|| \mathbf{T}'(t) ||}	
	\end{align}
	\begin{itemize}
		\item \dfn{Normal plane}: the plane determined by the normal and binormal vectors. Consists of all lines orthogonal to the unit tangent vector $\mathbf{T}$. 
		\item \dfn{Osculating circle}: the plane determined by the vectors $\mathbf{T}$ and $\mathbf{N}$. 
		\begin{itemize}
			\item Closest plane to containing the part of the curve near $P$. 
			\item \dfn{Osculating circle}: the circle that lies on the osculating plane of $C$ at $P$, has the same tangent as $C$ at $P$, and lies on the concave side of $C$ (towards where $\mathbf{N}$ is pointing). This best describes the behaviour of $C$ near $P$. 
		\end{itemize}
	\end{itemize}
	\item The \dfn{velocity vector} $\mathbf{v}(t)$ at time $t$: 
	\begin{align}
		\mathbf{v}(t) := \lim_{h \rightarrow 0} \frac{\mathbf{r}(t+h) - \mathbf{r}(t)}{h} = \mathbf{r}'(t) 	
	\end{align}
	\begin{itemize}
		\item \dfn{Speed}: the magnitude of the velocity vector $|| \mathbf{v}(t) ||$. 
		\begin{align}
			|| \mathbf{v}(t) || = || \mathbf{r}'(t) || = \frac{ds}{dt}	
		\end{align}
		\item \dfn{Acceleration}: the derivative of the velocity
		\begin{align}
			\mathbf{a}(t) = \mathbf{v'}(t) = \mathbf{r}''(t) 	
		\end{align}
	\end{itemize}
	\item Often times it can be useful to resolve the acceleration of a particle into its tangential and normal components: 
	\begin{align}\label{accdecomp}
		\mathbf{a} = \underbrace{v'}_{:= \mathbf{a}_T}  \mathbf{T} + \underbrace{ \kappa v^2}_{:= \mathbf{a}_N}	\mathbf{N}
	\end{align}
	where $v := || \mathbf{r}'(t) ||$. One can re-write Equation (\ref{accdecomp}) so that it only depends on $\mathbf{r}$, $\mathbf{r}'$, and $\mathbf{r}''$: 
	\begin{align}
		\mathbf{a}_T & = \frac{\mathbf{r}'(t) \cdot \mathbf{r}''(t)}{|| \mathbf{r}'(t) ||}	\\
		\mathbf{a}_N & = \frac{|| \mathbf{r}'(t) \times \mathbf{r}''(t) || }{|| \mathbf{r}'(t) ||}
	\end{align}
\end{itemize}
\subsection{Partial Derivatives}
\begin{itemize}[noitemsep]
	\item \dfn{Graph}: let $f$ be a function in two variables with domain $\Omega$. Then, the \dfn{graph} of $f$ is the set of all points $(x,y,z) \in \R^3$ such that $z = f(x,y)$ for $(z,y) \in \Omega$. 
	\item \dfn{Level Curves}: The \dfn{level curves} of a function $f$ in two variables are the curves with equations $f(x,y) = k$, where $k \in \R$ is a constant.
	\item \dfn{Limit}: Let $f$ be a function of two variables whose domain $\Omega$ includes points arbitrarily close to $(a,b)$. Then, we say that the \dfn{limit of $f(x,y)$ as $(x,y) \rightarrow (a,b)$} is $L$ and we write:
	\begin{align}
		\lim_{(x,y) \rightarrow (a,b)} f(x,y) = L	
	\end{align}
 	if $\forall$ $\varepsilon > 0$, $\exists \delta > 0$ such that if $(x,y) \in \Omega$, $0 < \sqrt{(x-a)^2 + (y-b)^2} < \delta$, then $|f(x,y) - L | < \varepsilon$. 
 	\item \dfn{Partial Derivative}: the partial derivative of $f$ with respect to $x$ at $(a,b)$ is: 
 	\begin{align}
 		f_x(a,b) := \lim_{h \rightarrow 0} \frac{f(a+h) - f(a,b)}{h}	
 	\end{align}
	\item \dfn{Claircut's Theorem}: suppose $f$ is defined on a disk $D$ that contains the point $(a,b)$. If the functions $f_{xx}$ and $f_{yy}$ are both continuous on $D$, then, 
	\begin{align}
		f_{xy} (a,b) = f_{yx} (a,b) 	
	\end{align}
 	\item \dfn{Tangent Plane}: Suppose $f$ has continuous partial derivatives. An equation for the tangent plane to the surface $z = f(x,y)$ at the point $P(x_0, y_0, z_0) $ is given by: 
 	\begin{align}
 		z - z_0 = f_x (x_0, y_0) (x-x_0) + f_y (x_0, y_0) (y-y_0 ) 	
 	\end{align}
	\begin{itemize}[noitemsep]
		\item \dfn{Linearisation}: an equation for the tangent plane to the graph $f$ at the point $(a,b, f(a,b))$ is given by: 
		\begin{align}
			z = f(a,b) + f_x(a,b) (x-a) + f_y(a,b) (y-b) 	
		\end{align}
		the graph of this tangent plane is called the \dfn{linearisation} of $f$ at $(a,b)$: 
		\begin{align}
			L(x,y) := f(a,b) + f_x(a,b) (x-a) + f_y(a,b) (y-b)	
		\end{align}
	\end{itemize}
	\item \dfn{Differentiable}: If $z = f(x,y)$, then $f$ is \dfn{differentiable} at $(a,b)$ if $\Delta z$ can be expressed in the form: 
	\begin{align}
		\Delta z = f_x (a,b) \Delta x + f_y (a,b) \Delta y + \varepsilon_1 \Delta x + \varepsilon_2 \Delta y	
	\end{align}
	where $\varepsilon_1, \varepsilon_2 \rightarrow 0$ as $(\Delta x, \Delta y) \rightarrow  (0,0)$.
	\begin{itemize}[noitemsep]
		\item If the partial derivatives $f_x$ and $f_y$ exist near $(a,b)$, and are continuous at $(a,b)$, then $f$ is differentiable at $(a,b)$.
		\item \dfn{Total Differential}: for a differentiable function of two variables $z = f(x,y)$, then the \dfn{total} \dfn{differential} is defined as: 
		\begin{align*}
			dz & := f_x (x,y) dx + f_y (x,y) dy 	\\
			   & = \frac{\partial z}{\partial x} dx + \frac{\partial z}{\partial y} dy
		\end{align*} 
		and so in the language of differentials, we write: 
		\begin{align*}
			f(x,y) \approx f(a,b) + dz
		\end{align*}
	\end{itemize}
	\item \dfn{Chain Rule (1. Case)}: Suppose that $z= f(x,y)$ is a differentiable function of $x$ and $y$, and suppose that both $x$ and $y$ are differentiable functions of $t$ (i.e, $x=x(t)$, $y=y(t)$) so that $z = (f(x(t), y(t))$. Then, $z$ is a differentiable function of $t$ and: 
	\begin{align}
		\frac{dz}{dt} = \frac{\partial f}{\partial x} \frac{dx}{dt} + \frac{\partial f}{\partial y}\frac{dy}{dt}	
	\end{align}
	\item \dfn{Chain Rule (General Version)}: Suppose that $u$ is a differentiable function of $n$ variables ($x_1, ..., x_n$) and each $x_j$ is a differentiable function of $m$ variables $t_1, ..., t_m$. Then, $u$ is a function of $t_1, ..., t_m$ and one has: 
	\begin{align}
		\frac{\partial u}{\partial t_i} = \frac{\partial u}{\partial x_i} \frac{\partial x_1}{\partial t_i}	+ \frac{\partial u}{\partial x_i} \frac{\partial x_2}{\partial t_i} + ... + \frac{\partial u}{\partial x_i} \frac{\partial x_n}{\partial t_i}
	\end{align}
	$\forall i = 1, 2, ..., m$. 
	\item \dfn{Implicit Function Theorem}: Let $y=f(x)$. If $F$ is defined on a disc containing $(a,b)$, where $F(a,b) = 0$ $F_y(a,b) \neq 0$, and $F_y$ and $F_x$ are continuous on the disc, then the equation $F(x,y) = 0$ defines $y$ as a function of $x$ near the point $(a,b)$ whose derivative is given by: 
	\begin{align}
		\frac{dy}{dx} = - \frac{\frac{\partial F}{\partial x}}{\frac{\partial F}{\partial y}} = - \frac{F_x}{F_y}	
	\end{align}
	Now, for $z=f(x,y)$, if $z$ is implicitly given as a function by an equation of the form $F(x,y,z) =0$, then the derivative of the implicitly defined function is: 
	\begin{align}
		\frac{\partial z}{\partial x} & = - \frac{\frac{\partial F}{\partial x}}{\frac{\partial F}{\partial z}}	\\
		\frac{\partial z}{\partial y} & = - \frac{\frac{\partial F}{\partial y}}{\frac{\partial F}{\partial z}}
	\end{align}
	\item \dfn{Directional Derivative}: the \dfn{directional derivative} of $f$ at $(x_0, y_0)$ in the direction of the unit vector $\mathbf{u} = \langle a, b \rangle$ is:
	\begin{align}
		D_uf(x_0, y_0) := \lim_{h \rightarrow 0 } \frac{f(x_0 + ha, y_0 + hb) - f(x_0, y_0)}{h}	
	\end{align}
	\begin{itemize}[noitemsep]
		\item If $f$ is a differentiable function of $x$ and $y$, then $f$ has a directional derivative in the direction of any unit vector $\mathbf{u} = \langle a, b \rangle$ and 
		\begin{align}	
			D_f(x,y) = f_x(x,y) a + f_y(x,y) b 	
		\end{align}
	\end{itemize}
	\item \dfn{Gradient}: Let $f: \R \times \R \rightarrow \R$ be a function of two variables. Then, the \dfn{gradient} of $f$ is the vector function $\nabla f$ defined by: 
	\begin{align*}
		\nabla f(x,y) & := \langle f_x(x,y), f_y(x,y) \rangle \\
					  & = \frac{\partial f}{\partial x} \mathbf{\hat{i}} + \frac{\partial f}{\partial y} \mathbf{\hat{j}} 
	\end{align*}
	Using this notation, we can re-write the directional derivative as a dot product: 
	\begin{align*}
		D_u f(x,y) = \nabla f(x,y) \cdot \mathbf{u} 
	\end{align*}
	Thus, we can interpret the directional derivative as the scalar projection of the gradient function onto $\mathbf{u}$. 
	\item Let $f: \R^n \rightarrow \R$ be a differentiable function of $n$ variables. The maximum value of the directional derivative $D_uf(x)$ is $|| \nabla f(x) ||$; this occurs when $\mathbf{u}$ is parallel to $\nabla f(x)$. 
	\item The \dfn{tangent planet to the level surface} $F(x,y,z) = k$ at the point $P(x_0, y_0, z_0)$ is the plane that passes through $P$ with the normal vector $\nabla F(x_0, y_0, z_0)$. This is written as:
	\begin{align}
		F_x(x_0, y_0, z_0) (x-x_0) + F_y(x_0, y_0, z_0) (y-y_0) + F_z (x_0, y_0, z_0) (z-z_0) = 0 	
	\end{align}
\end{itemize}

\subsection{Multiple Integrals}
\begin{itemize}
	\item \dfn{Double Integral}: the \dfn{double integral} of $f$ over the rectangle $\mathcal{R}$ is: 
	\begin{align}
		\iint_{\mathcal{R}} f(x,y) dA := \lim_{m,n \rightarrow \infty} \sum_{i=1}^m \sum_{j=1}^n f(x_{i,j}^*, y_{i,j}^*) \Delta A 	
	\end{align}
	if the limit exists.
	\begin{itemize}
		\item If $f \geq 0$, then the volume $V$ of the solid above the rectangle $\mathcal{R}$ and below the surface $z=f(x,y)$ is:
		\begin{align}
			V = \iint_{\mathcal{R}} f(x,y) dA	
		\end{align}
	\end{itemize}
	\item \dfn{Fubini's Theorem}: If $f$ is continuous on the rectangle $\mathcal{R} := \{ (x,y)\ |\ a \leq x \leq b, c \leq y \leq d \}$ then: 
	\begin{align}
		\iint_{\mathcal{R}} f(x,y) dA = \int_a^b \int_c^d f(x,y) dy dx = \int_c^d \int_a^b f(x,y) dx dy 	
	\end{align}
	\item \dfn{Average Value}: the \dfn{average value} of a function $f$ of two variables defined on the rectangle $\mathcal{R}$ is defined to be: 
	\begin{align}
		f_{\text{avg}} := \frac{1}{A(\mathcal{R})} \iint_{\mathcal{R}} f(x,y) dA 	
	\end{align}
	\item \dfn{Double Integral of $f$ over $D$}: let $D$ be a bounded region in $\R^n$. Define the following function $F$; let $\mathcal{R}$ be a rectangle such that $D \subseteq \mathcal{R}$ and 
	\begin{align}
		F(x,y) := \begin{cases}
			f(x,y) & (x,y) \in D \\
			0 	   & (x,y) \notin D 
		\end{cases}	
	\end{align}
		then, the \dfn{double integral of $f$ over $D$} is given by:
		\begin{align}
			\iint_{D} f(x,y) dA := \iint_{\mathcal{R}} F(x,y) dA 
	    \end{align}
	 \item We have various ``types'' of domains/regions: 
	 \begin{itemize}
	 	\item \dfn{Type I}: A plane region $D$ is Type I if it lies between the graphs of two continuous functions of $x$: 
	 	\begin{align}
	 		D := \{ (x,y) \in \R\ |\ a \leq x \leq b, g_1(x) \leq y g_2(x) \} 	
	 	\end{align}
		in this case, the double integral is given by: 
		\begin{align}
			\iint_D f(x,y) dA := \int_a^b \int_{g_1(x)}^{g_2(x)} f(x,y) dy dx 	
		\end{align}
		\item \dfn{Type II}: a plane region of Type II is: 
		\begin{align}
			D := \{ (x,y) \in \R^2\ |\ c \leq y \leq d,\ h_1(y) \leq x \leq h_2(y) \}	
		\end{align}
		where $h_1, h_2 \in C(\R)$. In this case, the double integral is given by: 
		\begin{align*}
			\iint_D f(x,y) dA := \int_c^d \int_{h_1(y)}^{h_2(y)} f(x,y) dx dy
		\end{align*}
	 \end{itemize}
	 \item If $m \leq f(x,y) \leq M$ $\forall$ $(x,y) \in D$, then: 
	 \begin{align}
	 	mA(D) \leq \iint_D f(x,y) dA \leq M A(D) 	
	 \end{align}
	\item \dfn{Polar Rectangle}: $\mathcal{R} := \{ (r, \theta)\ |\ a \leq r \leq b, \alpha \leq \theta \leq \beta \}$ 
	\begin{itemize}
		\item \dfn{Change to Polar Coordinates in a Double Integral}: Let $f: \R \rightarrow \R$ be continuous, $\mathcal{R}$ a polar rectangle, $f$ continuous. If $\mathcal{R}$ is given by $0 \leq a \leq r \leq b$, $\alpha \leq \theta \beta$, $0 \leq \beta - \alpha \leq 2 \pi $. Then: 
		\begin{align}
			\iint_{\mathcal{R}} f(x,y) dA := \int_\alpha^\beta \int_a^b f ( r \cos (\theta), r \sin (\theta)) r dr d \theta 	
		\end{align}
		if $f$ is continuous on a polar region of the form $D := \{ (r, \theta)\ |\ \alpha \leq \theta \leq \beta, h_1(\theta) \leq r \leq h_2(\theta) \}$, then: 
		\begin{align}
			\iint_D f(x,y) dA = \int_{\alpha}^{\beta} \int_{h_1(\theta)}^{h_2(\theta)} f(r \cos (\theta), r \sin (\theta) ) r dr d \theta 	
		\end{align}
	\end{itemize}
	\item \dfn{Surface Area}: the area of the surface $z = f(x,y)$ for $(x,y) \in D $ where $\frac{\partial z}{\partial x}$ and $\frac{\partial z}{\partial y}$ are continuous is given by: 
	\begin{align}
		A(S) = \iint_D \sqrt{1 +  \left( \frac{\partial z}{\partial x} \right)  + \left( \frac{\partial z}{\partial y} \right) }	
	\end{align}	
\end{itemize}
\subsection{Triple Integrals}
\begin{itemize}
	\item \dfn{Triple Integrals over a Box}: The \dfn{triple integral} of $f$ over the box $B$ is: 
	\begin{align}
		\iiint_B f(x,y,z) dV := \lim_{\ell, m, n \rightarrow \infty} \sum_{i=1}^\ell \sum_{j=1}^m \sum_{k=1}^n f(x_{ijk}^*, y_{ijk}^*, z_{ijk}^*) \Delta V	
	\end{align}
	\item \dfn{Cylindrical Coordinate System}: a point $P \in \R^3$ is represented by the ordered triple $(r, \theta, z)$; the equations to convert are given by: 
	\begin{align*}
		& x = r \cos (\theta)\ &  y = r \sin (\theta)\ & z = z \\
		& r^2 = x^2 + y^2 + z^2\	& \tan (\theta) = y/z\ & z = z 
	\end{align*} 
	Often useful in problems involving symmetry about an axis. 
	\item \dfn{Triple Integration in Polar Coordinates}: Suppose $E$ is a Type I region whose projection $D$ onto the $xy$-plane is described in polar coordinates: 
	\begin{align*}
		E & = \{ (x, y, z)\ |\ (x,y) \in D\ |\ u_1(x,y) \leq z \leq u_2(x,y)\ \} \\
		D & = \{ (r, \theta)\ |\ \alpha \leq \theta \leq \beta,\ h_1(\theta) \leq r \leq h_2(\theta) \} 	
	\end{align*}
	Then, 
	\begin{align}
		\iint_E f(x,y,z) dV = \int_\alpha^\beta \int_{h_1(\theta)}^{h_2(\theta)} \int_{u_1(r \cos (\theta), r \sin (\theta) }^{u_2(r \cos (\theta), r \sin (\theta)}	f (r \cos (\theta), r \sin (\theta)) r dz dr d \theta 
	\end{align}
	\item \dfn{Spherical Coordinates}: the spherical coordinates $(\rho, \theta, \varphi)$ of a point $p \in \R^3$ are given by: 
	\begin{align*} 
		& x = \rho \sin (\varphi) \cos (\theta),\ y = \rho \sin (\theta) \sin (\theta),\ z = \rho \cos (\varphi) \\
		& \rho^2 = x^2 + y^2 + z^2 	
	\end{align*} 
	\begin{itemize}[noitemsep]
		\item The formula for a triple integral in spherical coordinates is given by:
		\begin{align}
			\iiint_E f(x,y,z) dV = \int_c^d \int_\alpha^\beta \int_a^b f(\rho \sin (\varphi) \cos (\theta), \rho \sin (\theta) \sin (\theta), \rho \cos (\varphi) ) \rho^2 \sin (\varphi) d \rho d \theta d \varphi 	
		\end{align}
	\end{itemize}
	\item \dfn{Jacobian}: the \dfn{Jacobian} of a transformation $T$ given by $x = g(u,v)$ and $y=h(u,v)$ is given by: 
	\begin{align}
		\frac{\partial (x,y)}{\partial (u,v)} = \det \begin{bmatrix}
			\frac{\partial x}{\partial u} & \frac{\partial x}{\partial v} \\
			\frac{\partial y}{\partial u} & \frac{\partial y}{\partial v}
		\end{bmatrix}	
	\end{align}
	\begin{itemize}[noitemsep]
		\item Suppose $T$ is a $C^1$-transformation whose Jacobian is non-zero and that $T$ maps $S$ in the $uv$-plane to region $\mathcal{R}$ in the $xy$-plane. Suppose that $f$ is continuous on $\mathcal{R}$ and that both $\mathcal{R}$ and $S$ are Type I or Type II regions. Moreover, assume that $T$ is bijective, except for potentially on $\partial S$. Then: 
		\begin{align}
			\iint_{\mathcal{R}} f(x,y) dA = \iint_S f(x(u,v), y(u,v)) | \det (\mathbf{Jac}) | du dv 	
		\end{align}
	\end{itemize}
\end{itemize}

\section{Vector Calculus}
\subsection{Vector Fields}
\begin{itemize}[noitemsep]
	\item \dfn{Vector Field on $\R^n$}: Let $D \subseteq \R^n$. A \dfn{vector field} on $\R^n$ is a function $\mathbf{F}$ that assigns to every point $(x_1, ..., x_n) \in D$ an n-dimensional vector $\mathbf{F}(x_1, ..., x_n)$. One can write this in terms of component functions, e.g. in $\R^2$: 
	\begin{align}
		\mathbf{F}(x,y) = P(x,y) \mathbf{\hat{i}} + Q(x,y) \mathbf{\hat{j}} + Q(x,y) \hat{\mathbf{k}}  = \langle P(x,y), Q(x,y), Q(x,y) \rangle 	
	\end{align}
	\begin{itemize}
		\item \dfn{Gradient Vector Field}: if $f: \R^2 \rightarrow \R$, recall that $\nabla f$ is: 
		\begin{align}
			\nabla f(x,y) = f_x(x,y) \hat{\mathbf{i}} + f_y(x,y) \hat{\mathbf{j}} 	
		\end{align}
		which means that $\nabla f$ is a vector field on $\R^2$ (we call this vector field a \dfn{gradient vector field}).
	\end{itemize}
	\item \dfn{Conservative Vector Field}: A vector field $\mathbf{F}$ is a \dfn{conservative vector field} if there exists a scalar function $f$ such that $\nabla f = \mathbf{F}$. 
\end{itemize}
\subsection{Line Integrals}
\begin{itemize}[noitemsep]
	\item \dfn{Line Integral}: parameterise a smooth curve $C$ by
	\begin{align*}
		x = x(t),\ y = y(t)\ t \in [a,b] 
	\end{align*}
	or, equivalently, 
	\begin{align*}
		r(t) = x(t) \hat{\textbf{i}} + y(t) \hat{\textbf{j}} 
	\end{align*}
	then, the \dfn{line integral} of $f$ along $C$ is: 
	\begin{align}
		\int_C f(x,y) ds = \lim_{n \rightarrow \infty} \sum_{i=1}^n f(x_i^*, y_i^*) \Delta s_i 	
	\end{align}
	the line integral is given by: 
	\begin{align}
		\int_C f(x,y) ds = \int_a^b f(x(t), y(t)) \sqrt{\left( \frac{dx}{dt} \right)^2 + \left( \frac{dy}{dt} \right)^2 } dt 	
	\end{align}
	\begin{itemize}[noitemsep]
		\item A more compact notation for line integrals can be given by: 
		\begin{align}
			\int_a^b f(\mathbf{r}(t)) || \mathbf{r}'(t) || dt 	
		\end{align}
	\end{itemize}
	\item \dfn{Line Integrals over Vector Fields}: Let $\mathbf{F}$ be a continuous vector field defined on a smooth curve $C$ be given by a vector function $\mathbf{r}(t)$, $ t \in [a,b]$. Then, the \dfn{line integral of $\mathbf{F}$ along $C$} is: 
	\begin{align}
		\int_C \mathbf{F} \cdot d \mathbf{r} = \int_a^b F(\mathbf{r}(t)) \cdot \mathbf{r}'(t) dt = \int_C \mathbf{F} \cdot \mathbf{T} ds 	
	\end{align}
	Suppose a vector field $\mathbf{F}$ on $\R^3$ is given in compact-form by the equation $\mathbf{F} = P \hat{\mathbf{i}} + Q \hat{\mathbf{j}} + R \hat{\mathbf{k}} $. Then, 
	\begin{align}
		\int_C \mathbf{F} \cdot dr = \int_C P dx + Q dy + K dz 	
	\end{align}
	\item \dfn{Fundamental Theorem for Line Integrals}: Let $C$ be a smooth curve given by the vector function $\mathbf{r}(t)$ for $t \in [a,b]$. Let $f$ be a differentiable function $f$ of two or three variables whose gradient vector $\nabla f$ is continuous on $C$. Then:
	\begin{align}
		\int_C \nabla f \cdot d \mathbf{r}  = f (\mathbf{r}(b)) - f (\mathbf{r}(a)) 	
	\end{align}
	\item Path Independence
	\begin{itemize}[noitemsep]
		\item $\int_C \mathbf{F} \cdot d \mathbf{r}$ is independent of path in $D$ $\iff$ $\int_C \mathbf{F} \cdot d \mathbf{r} = 0$ for every closed path $C$ in $D$.
		\item Let $D$ be an open, connected domain. Suppose that $F$ is a vector field on $D$. If $\int_C \mathbf{F} \cdot d \mathbf{r}$ is independent of path on $D$, then $\mathbf{F}$ is a conservative vector field on $D$, that is, $\exists$ an $f$ such that $\nabla f = \mathbf{F}$.
		\item Let $F(x,y) = P(x,y) \hat{\mathbf{i}} + Q(x,y) \hat{\mathbf{j}} $ be a conservative vector field, where $P $ and $Q$ have continuous first-order partial derivatives on a domain $D$. Then, throughout $D$, we have
		\begin{align}
			\frac{\partial P}{\partial y} = \frac{\partial Q}{\partial x}	
		\end{align}
		\item \dfn{Simple Curve}: a curve that doesn't intersect itself anywhere between its endpoints. 
		\item \dfn{Simply-Connected Region}: a connected region $D$ such that every simple closed curve in $D$ encloses points that are only in $D$.
	\end{itemize}
	\item Let $\mathbf{F} = P \hat{\mathbf{i}} + Q \mathbf{\hat{j}}$ be a vector field of an open, simply-connected region $D$. Suppose that $P$ and $Q$ have continuous first-order partial derivatives and 
	\begin{align}
		\frac{\partial P}{\partial y} = \frac{\partial Q}{\partial x}	
	\end{align}
	throughout $D$. Then, $\mathbf{F}$ is conservative. 
\end{itemize}

\subsection{Green's Theorem}
Counterpart of the Fundamental Theorem of Calculus for double integrals.
\begin{itemize}[noitemsep]
	\item \dfn{Green's Theorem}: Let $C$ be a positively-oriented, piece-wise smooth simple closed curve in the plane and let $D$ be a region bounded by $C$. If $P$ and $Q$ have continuous partial derivatives on an open region containing $D$, then:
	\begin{align}
		\oint_C Pdx + Qdy = \iint_D \left( \frac{\partial Q}{\partial x} - \frac{\partial P}{\partial y} \right) dA 	
	\end{align}
	\item Can be used to calculate areas: 
	\begin{align}
		A(D) = \oint_C x dy = - \oint_C y dx = \frac{1}{2} \oint_C x dy - y dx 	
	\end{align}
\end{itemize}

\subsection{Curl and Divergence}
Each of the following operations resemble differentiation, but one produces a vector field and the other produces a scalar field. 

\begin{itemize}[noitemsep]
	\item \dfn{Curl}: Let $\mathbf{F} := P \hat{\mathbf{i}} + Q \hat{\mathbf{j}} + R \hat{\mathbf{k}}$ be a vector field on $\R^3$. Assume that the partial derivatives $P$, $Q$, and $R$ all exist. Then, the \dfn{curl} of $\mathbf{F}$ is the vector field on $\R^3$ defined by: 
	\begin{align*}
		\operatorname{curl}(F) & := \left( \frac{\partial R}{\partial y} - \frac{\partial Q}{\partial z}   \right) \hat{\mathbf{i}} + \left( \frac{\partial P}{\partial z} - \frac{\partial R}{\partial x} \right) \hat{\mathbf{j}} + \left( \frac{\partial Q}{\partial x} - \frac{\partial P}{\partial y} \right) \hat{\mathbf{k}} \\
		& = \nabla \times \mathbf{F} \\
		& = \det \begin{bmatrix}
			\hat{\mathbf{i}} & \hat{\mathbf{j}} & \hat{\mathbf{k}} \\
			\frac{\partial}{\partial x} & \frac{\partial }{\partial y} & \frac{\partial }{\partial z} \\
			P & Q & R
		\end{bmatrix}
	\end{align*}
	\item If $f$ is a function of three variables and has continuous, second-order partial derivatives, then:
	\begin{align}
		\text{curl}(\nabla f) = 0 	
	\end{align}
	\begin{itemize}[noitemsep]
		\item Conservative vector fields have $\mathbf{F} = \nabla f$, and so $\text{curl}(\mathbf{F}) = 0$ for conservative vector fields.
		\item Let $\mathbf{F}$ be a vector field defined on all of $\R^3$ whose component functions have continuous partial derivatives and $\text{curl}(\mathbf{F}) = 0$. Then, $\mathbf{F}$ is a conservative vector field. 
		\item We say that a vector field $\mathbf{F}$ is irrotational if $\text{curl}(\mathbf{F})=0$.
	\end{itemize}
	\item \dfn{Divergence}: Let $\mathbf{F} := P \hat{\mathbf{i}} + Q \hat{\mathbf{j}} + R \hat{\mathbf{k}}$ be a vector field on $\R^3$ and assume that $\frac{\partial P}{\partial x }$, $\frac{\partial Q}{\partial y}$, and $\frac{\partial R}{\partial z}$ all exist. Then, the \dfn{divergence} of $\mathbf{F}$ is defined by:
	\begin{align}
		\operatorname{div}(\mathbf{F}) & := \frac{\partial P}{\partial x} + \frac{\partial Q}{\partial y} + \frac{\partial R}{\partial z}	\\
			& = \nabla \cdot \mathbf{F} 
	\end{align}
	\item If $\mathbf{F} := P \hat{\mathbf{i}} + Q \hat{\mathbf{j}} + R \hat{\mathbf{k}}$ be a vector field on $\R^3$ and if $P$, $Q$, and $R$ have continuous second-order partial derivatives, then 
	\begin{align}
		\operatorname{div}( \operatorname{curl}(\mathbf{F})) = 0 	
	\end{align}
	\begin{itemize}[noitemsep]
		\item A vector field $\mathbf{F}$ is called \dfn{incompressible} if $\operatorname{div}(\mathbf{F}) =0$. 
	\end{itemize}
	\item We can use what we've built up here to formulate Green's Theorem in terms of vector forms. 
	\begin{itemize}[noitemsep]	
		\item The first vector form of Green's theorem is: 
		\begin{align}
			\oint_C \mathbf{F} \cdot d \mathbf{r} 	= \iint_D ( \operatorname{curl}(\mathbf{F})) \cdot \mathbf{\hat{k}} d A 
		\end{align}
		\item The second vector form of Green's Theorem is: 
		\begin{align}
			\oint_C \mathbf{F} \cdot \mathbf{\hat{n}}	ds = \iint_D \operatorname{div}(\mathbf{F}) dA
		\end{align}
	\end{itemize}
\end{itemize}

\subsection{Parametric Surfaces and their Areas}
A vector function $\mathbf{r}(u,v) := x(u,v) \mathbf{\hat{i}} + y(u,v) \mathbf{\hat{j}} + z(u,v) \hat{\mathbf{k}}$ can trace out a surface. We call this surface a \dfn{parametric surface}; consider the following tangent vectors: 
\begin{align}
	\mathbf{r}_v &  = \frac{\partial x}{\partial v}(u_0, v_0) \hat{\mathbf{i}} + \frac{\partial y}{\partial v}(u_0, v_0) \hat{\mathbf{j}} + \frac{\partial z}{\partial v}(u_0, v_0) \hat{\mathbf{k}} 	 \\
	\mathbf{r}_u &  = \frac{\partial x}{\partial u}(u_0, v_0) \hat{\mathbf{i}} + \frac{\partial y}{\partial u}(u_0, v_0) \hat{\mathbf{j}} + \frac{\partial z}{\partial u}(u_0, v_0) \hat{\mathbf{k}} 	 	
\end{align}
If $\mathbf{r}_u \times \mathbf{r}_v \neq \mathbf{0}$, then the surface is \dfn{smooth} and the tangent plane is the plane containing the vectors $\mathbf{r}_u$ and $\mathbf{r}_v$.

\begin{definition}[Surface Area]
	Let $S$ be a smooth parametric surface given by the equation:
	\begin{align}
		\mathbf{r}(u,v) = x(u,v) \hat{\mathbf{i}} + y(u,v) \mathbf{\hat{j}} + z(u,v) \mathbf{\hat{k}} 	
	\end{align}
	for $(u,v) \in D$. Then, the \dfn{surface area} of $S$ is:
	\begin{align}
		A(S) := \iint_D || \mathbf{r}_u \times \mathbf{r}_v || dA 	
	\end{align}
\end{definition}
If, moreover, the surface $S$ has the equation $z= f(x,y)$, $(x,y) \in D$ and $f \in C^1$, then:
\begin{align}
	A(S) = \iint_D \sqrt{1 + \left( \frac{\partial z}{\partial x} \right)^2  + \left( \frac{\partial z}{\partial y} \right)^2} dA 	
\end{align}

\subsection{Surface Integrals}
\begin{definition}[Surface Integral over $S$] Suppose that a surface $S$ has the following vector equation:
\begin{align}
	\mathbf{r}(u,v) = x(u,v) \hat{\mathbf{i}} + y(u,v) \hat{\mathbf{j}} + z(u,v) \hat{\mathbf{k}} 	
\end{align}
Then, the \dfn{surface integral of $f$ over $S$} is: 
\begin{align}
	\iint_S f(x,y,z) dS := \lim_{n, m \rightarrow \infty} \sum_{i=1}^m \sum_{j=1}^n f(P^*_{ij}) \Delta S_{ij} 	
\end{align}
where
\begin{align*}
	\Delta S_{ij} \approx || \mathbf{r}_u \times \mathbf{r}_v || \Delta u \Delta v
\end{align*}
If the components are continuous and $\mathbf{r}_u$ and $\mathbf{r}_v$ are non-zero and non-parallel on int$(D)$, then:
\begin{align}
	\iint_S f(x,y,z) dS = \iint_D f( \mathbf{r}(u,v)) || \mathbf{r}_u \times \mathbf{r}_v || dA
\end{align}	
\end{definition}
If, moreover, $S$ has an equation $z=g(x,y)$, then:
\begin{align*}
	\iint_S f(x,y,z) dS = \iint_D f(x,y,g(x,y)) \sqrt{ \left( \frac{\partial z}{\partial x} \right)^2 + \left( \frac{\partial z}{\partial y} \right)^2  + 1 } dA 
\end{align*}

\subsection{Oriented Surface}
\begin{definition}[Oriented Surface]
	Let $S$ be a surface with a with a tangent plane at every point $(x,y,z) \in S$. If it's possible to choose a unit normal vector $\mathbf{n}$ at every point $(x,y,z) \in S$ such that $\mathbf{n}$ varies continuously over $S$, then we say that $S$ is an \textbf{orientable surface}.
\end{definition}
\begin{itemize}
	\item For a surface $z = g(x,y)$, the natural orientation will be given by the following unit normal vector:
	\begin{align}
		\mathbf{n} = \frac{- \frac{\partial g}{\partial x} \mathbf{\hat{i}} - \frac{\partial g}{\partial y}\mathbf{\hat{j}} + \mathbf{\hat{k}} }{\sqrt{1 + \left( \frac{\partial g}{\partial x} \right)^2 + \left( \frac{\partial g}{\partial y} \right)^2}}	
	\end{align}
	\item For a smooth orientable surface $S$ given in parametric form by a vector $\mathbf{r}(x,y)$, the unit normal points in the same direction as the position vector:
	\begin{align}
		\mathbf{n} = \frac{\mathbf{r}_u \times \mathbf{r}_v}{||\mathbf{r}_u \times \mathbf{r}_v ||}	
	\end{align}

\end{itemize}


\subsection{Surface Integrals over Vector Fields}
\begin{definition}[Surface Integral]
	Let $\mathbf{F}$ be a continuous vector field defined on an oriented surface $S$ with unit normal vector $\mathbf{n}$. Then, the \dfn{surface integral of $\mathbf{F}$ over $S$} is: 
	\begin{align}
		\iint_S \mathbf{F} \cdot \mathbf{dS} = \iint_S \mathbf{F} \cdot \mathbf{n} dS 	
	\end{align}
\end{definition}

\begin{enumerate}
	\item If $S$ is given by a vector function $\mathbf{r}(u,v)$, then: 
	\begin{align}
		\iint_S \mathbf{F} \cdot \mathbf{dS} = \iint_D \mathbf{F} \cdot [\mathbf{r}_u \times \mathbf{r}_v] dA 	
	\end{align}
	\item If $S$ is given by a graph $z = g(x,y)$, then let $\mathbf{F} := (P, Q, R)$. Then, the surface integral is: 
	\begin{align}
		\iint_S \mathbf{F} \cdot \mathbf{dS} = \iint_D \left( - P \frac{\partial g}{\partial x} - Q \frac{\partial g}{\partial y} + R \right) dA 	
	\end{align}
\end{enumerate}

\subsubsection{Physical Interpretations: Electric Flux and Heat Flow}
\begin{enumerate}
	\item Electricity: Let $\mathbf{E}$ be an electric field. Then, the meaning of the surface integral $\iint_S \mathbf{E} \cdot \mathbf{dS}$ is that it represents the \dfn{electric flux of $\mathbf{E}$ through $S$}.Then, \dfn{Gauss' Law} states that the net charge enclosed by a closed surface $S$ is given by:
	\begin{align}
		Q = \epsilon_0 \iint_S \mathbf{E} \cdot \mathbf{dS} 	
	\end{align}
	\item Heat Flow: the \dfn{heat flow} is the vector field given by $\mathbf{F} := - k \nabla \mathbf{u}$, where $k$ is a constant that encodes the conductivity of the material. Then, the rate of heat flow across the surface $S$ in the body is given by:
	\begin{align}
		\iint_S \mathbf{F} \cdot \mathbf{dS} = -k \iint_S \nabla \mathbf{u} \cdot \mathbf{dS} 	
	\end{align}
\end{enumerate}
\subsection{Stokes' Theorem}
\begin{theorem}[Stokes' Theorem]
	Let $S$ be an oriented piece-wise smooth surface bounded by a simple, closed, piece-wise smooth boundary curve $C$ with positive orientation. Let $\mathbf{F}$ be a vector fields with components with partial derivatives which are $C^1$ on an open set $\Omega \subseteq \R^3$ such that $S \subseteq \Omega$. Then: 
	\begin{align}
		\int_C \mathbf{F} \cdot \mathbf{dr} = \iint_S \mathrm{curl}( \mathbf{F} ) \cdot \mathbf{dS}	
	\end{align}
	or, equivalently, 
	\begin{align}
		\int_{\partial S} \mathbf{F} \cdot \mathbf{dr} = \iint_S \mathrm{curl}(\mathbf{F}) \cdot \mathbf{dS} 	
	\end{align}
\end{theorem}
The physical meaning of $\int_{\partial S} \mathbf{F} \cdot \mathbf{dr}$ is that it represents the circulation of $\mathbf{F}$ around $C$.
\subsection{Divergence Theorem}
\begin{theorem}[Divergence Theorem]
	Let $E$ be a simple solid region. Let $S$ be the boundary surface of $E$, given with positive outward orientation. Let $\mathbf{F}$ be a vector field with component functions in $C^1$ on $\Omega \supseteq E$. Then: 
	\begin{align}
		\iint_S \mathbf{F} \cdot \mathbf{dS} = \iiint_E \mathrm{div}(\mathbf{F}) dV \end{align}
\end{theorem}



%
%
%\section{Ordinary Differential Equations}
%\subsection{Basic Concepts}
%\begin{itemize}[noitemsep]
%	\item A \dfn{differential equation} is an equation that involves an unknown function and its derivatives. We say that a function is an \dfn{ordinary differential equation} if the unknown function depends only on one independent variable. 
%	\item The \dfn{order} of a differential equation is the order of the highest derivative appearing in the equation.
%	\item A \dfn{solution} of an ODE in an unknown function $y$ and an independent variable $x$ on an interval $I$ is a function $y(x)$ satisfying the ODE for all $x \in I$. 
%\end{itemize}
%
%
%\subsection{Classification of First Order Differential Equations}
%First of all, the two forms that we can see ODEs come in: 
%\begin{itemize}[noitemsep]
%	\item \dfn{Standard Form}: $y' = f(x,y)$. 
%	\item \dfn{Differential Form}: $M(x,y) dx + N(x,y) dy =0$. 
%\end{itemize}
%And now, some types of first order differential equations that we will encounter: 
%\begin{itemize}[noitemsep]
%	\item \dfn{Linear Equation}: If $f(x,y)$ can be written as: 
%	\begin{align}
%		f(x,y) = -p(x) y + q(x) 	
%	\end{align}
%	then the ODE is \dfn{linear} and it can always be expressed as: 
%	\begin{align}
%		y' + p(x) y = q(x) 	
%	\end{align}
%	\item \dfn{Bernoulli Equation}: A \dfn{Bernoulli differential equation} is an equation of the form: 
%	\begin{align}
%		y' + p(x) y = q(x) y^n	
%	\end{align}
%	\item \dfn{Homogeneous Equation}: A differential equation in standard form is said to be \dfn{homogeneous} if:
%	\begin{align}
%		f(tx, ty) = f(x,y) 	
%	\end{align}
%	\item \dfn{Separable Equation}: A differential equation given in differential form 
%	\begin{align*}
%			M(x,y) dx + N(x,y) dy = 0 
%	\end{align*}
%	is said to be \dfn{separable} if $M(x,y) = A(x)$ and $B(x,y) = B(y)$. 
%	\item \dfn{Exact Equations}: A differential equation in the differential form is called \dfn{exact} if: 
%	\begin{align}
%		\frac{\partial M(x,y)}{\partial y} = \frac{\partial N(x,y)}{\partial x}	
%	\end{align}
%	\item Separable ODE $\Rightarrow$ exact ODE.
%\end{itemize}
%\subsection{Separable First Order Differential Equations}
%Suppose our ODE is of the form: 
%\begin{align}
%	A(x) dx + B(y) dy = 0 	
%\end{align}
%i.e., we have a separable first-order differential equation. Then, the solution is obtained by integrating: 
%\begin{align}
%	\int A(x) dx + \int B(y) dy = c	
%\end{align}
%where $c \in \R$ is our constant of integration. This is the \dfn{general solution}; if we have an initial value problem: 
%\begin{align*}
%	& A(x) dx + B(y) dy = 0 	 \\
%	& y(x_0) = y_0 
%\end{align*} 
%then we can obtain the particular solution by first obtaining the general solution, and then applying the initial condition to solve for $c \in \R$.
%\subsubsection{Reduction of a Homogeneous Equation}
%Suppose we have a homogeneous equation (i.e, $f(xt , yt) = f(x,t)$ $\forall t$. Then, there are two transformations which we can apply to obtain a separable equation: 
%\begin{enumerate}[noitemsep]
%	\item Make the following substitution: $y = xv$. This has the corresponding derivative:
%	\begin{align}
%		\frac{dy}{dx} = v + x \frac{dv}{dx}	
%	\end{align}
%	Then, the ODE is separable in $v$ and $x$. 
%	\item Re-write the differential equation as:
%	\begin{align}
%		\frac{dx}{dy} = \frac{1}{f(x,y)}	
%	\end{align}
%	and make the following substitution for $x$ $x:= yu$, which gives the corresponding derivative: 
%	\begin{align*}
%		\frac{dx}{dy} = u + y \frac{du}{dy}
%	\end{align*}
%	and substitute back into the ODE. 
%\end{enumerate}
%
%\subsection{Exact First-Order Differential Equations}
%We say than an ODE of the form 
%\begin{align*}
%	M(x,y) dx + N(x,y) dy = 0 
%\end{align*}
%is \dfn{exact} if there exists a function $g(x,y)$ such that
%\begin{align}
%	d g(x,y) = M(x,y) dx + N(x,y) dy 	
%\end{align}
%We can check to see if an ODE is exact by using the \dfn{test for exactness}: if $M(x,y)$ and $N(x,y)$ are continuous functions with continuous first partial derivatives on a rectangle in the $xy$-plane, then the ODE $M(x,y) dx + N(x,y) dy = 0 $ is exact $\iff$: 
%\begin{align}
%	\frac{\partial M(x,y)}{\partial y} = \frac{\partial N(x,y)}{\partial x}	
%\end{align}
%Then, the solution to the ODE is given by first solving the equations: 
%\begin{align*}
%	& \frac{\partial g(x,y)}{x} = M(x,y) \\
%	& \frac{\partial g(x,y)}{y} = N(x,y) 
%\end{align*}
%for $g(x,y)$; the solution is then implicitly given by:
%\begin{align}
%	g(x,y) = c,\ c \in \R 	
%\end{align}
%\subsubsection{Integrating Factors}
%We can transform an ODE into an exact ODE by using an \dfn{integrating factor} $I(x,y)$; we say that $I(x,y)$ is an \dfn{integrating factor} if the ODE
%\begin{align*}
%	I(x,y) [ M(x,y) dx + N(x,y) dy ] = 0 
%\end{align*}
%is exact. Below, we have some common integrating factors:
%\begin{enumerate}
%	\item If $ \frac{1}{N} \left( \frac{\partial M}{\partial y} - \frac{\partial N}{\partial x} \right) \equiv g(x)$, then:
%	\begin{align*}
%		I(x,y) = e^{\int g(x) dx } 
%	\end{align*}
%	\item If $\frac{1}{M} \left( \frac{\partial M}{\partial y} - \frac{\partial N}{\partial x} \right) \equiv h(y)$, then: 
%	\begin{align*}
%		I(x,y) = e^{-\int h(y) dy } 
%	\end{align*}
%\end{enumerate}
%Common Integrating Factors:
%\begin{table}[H]
%\centering
%\begin{tabular}{|c|c|c|}
%\hline 
%Group of Terms & Integrating Factor $I(x,y)$ & Exact Differential $dg(x,y)$  \\
%\hline 
%  $ydx - xdy$ &  $- \frac{1}{x^2}$ &   $\frac{xdy - ydx}{x^2} = d \left ( \frac{x}{y} \right)$ \\
%\hline 
% $ydx - xdy$  &  $\frac{1}{y^2}$    &  $\frac{ydx = xdy}{y^2}  = d \left( \frac{x}{y} \right)$    \\
% \hline 
% $ydx - xdy$  & $- \frac{1}{xy}$ & 	$\frac{xdy - ydx}{xy} = d \left(\ln  \frac{y}{x} \right) $	\\
% \hline 
% $ydx - xdy$  & $ - \frac{1}{x^2 + y^2}$	& 	$\frac{xdy - y dx}{x^2 + y^2} = d \left(  \arctan{ \frac{y}{x}}\right) $	\\     
% \hline            
% $ydx + xdy $  & 	$\frac{1}{xy}$	& $\frac{ydx + xdy}{xy} = d(\ln(x,y))$ \\
% \hline 
% $ydx + xdy $  &  $ \frac{1}{(xy)^n}$	& $\frac{ydx + xdy}{(xy)^n} = d \left[  \frac{-1}{(n-1)(xy)^{n-1}} \right] $\\      
% \hline 
% $ydy + xdx $ & $ \frac{1}{x^2+y^2}$ & $\frac{ydy + xdx}{(x^2 + y^2)^n} = d \left[ \frac{1}{2} \ln ( x^2 + y^2) \right] $ \\
% \hline 
% $ydy + xdx$ & $\frac{1}{(x^2+y^2)^n}$, $n > 1$& $\frac{ydy + xdx}{(x^2 + y^2)^n} = d \left[ \frac{-1}{2(n-1)(x^2+y^2)^{n-1}} \right] $ \\
% \hline 
%$ay dx + bx dy $ & $a^{a-1}y^{b-1}$ & $x^{a-1}y^{b-1}(ay dx + bx dy ) = d(x^a y^b)$ \\ 
%\hline 
%\end{tabular}
%\end{table}
%If $M= yf(xy)$ and $N=xg(xy)$, then
%\begin{align}
%I(x,y) = \frac{1}{xM - yN}	
%\end{align}

\end{document}