\documentclass[11pt]{article}
\usepackage{titlesec}
\titleformat{\section}[hang]{\normalfont\scshape}{\thesection.}{1em}{}

\usepackage[dvipsnames]{xcolor}
\usepackage[margin=2cm]{geometry}
\usepackage{amssymb} 
\usepackage{amsmath}
\usepackage{graphicx}
\usepackage{float}
\usepackage[normalem]{ulem} 
\usepackage{enumitem} 
\setlist[enumerate]{itemsep=0mm}
\usepackage{xcolor}
\setenumerate{label=(\roman*)}
\usepackage[utf8]{inputenc}
\usepackage[T1]{fontenc}
\usepackage{babel}
\usepackage{mathtools}
\usepackage{amsthm}
\usepackage{thmtools}
\usepackage{etoolbox}
\usepackage{fancybox}
\usepackage{framed}
\usepackage{tcolorbox}
% example environment
\theoremstyle{definition} 
\newtheorem{exmp}{Example}[section]


% question environment
\theoremstyle{definition}
\newtheorem{question}{Question}

%rd 
\newcommand{\rd}[0]{\mathbb{R}^d}
\newcommand{\R}[0]{\mathbb{R}}

% let E in R be measurable 
\newcommand{\EinR}[0]{Let $E \subseteq \R$ be measurable}

%integral 
\newcommand{\idx}[2]{\int_{#1}^{#2}}

% weak convergence 
\newcommand{\warrow}[0]{\rightharpoonup}
\newcommand{\fcvw}[0]{ \{f_n \} \warrow f \text{ in } L^p(E)} 

% Probability 
\DeclareRobustCommand{\bbone}{\text{\usefont{U}{bbold}{m}{n}1}}
\newcommand{\Var}[1]{\mathrm{Var[#1]}}			% variance
\newcommand{\EX}[1]{\mathbb{E}\mathrm{[#1]}}	 % expected value 
\newcommand{\seq}[1]{\{ #1_n	\}_{n \in \bb{N}}} % sequence of events
\newcommand{\pspace}[0]{( \Omega, F, P)}		% probability space
\newcommand{\msp}[0]{( \Omega, F)}		% measurable space
	
% Exercise environment 
\newenvironment{myleftbar}{%
\def\FrameCommand{\hspace{0.6em}\vrule width 2pt\hspace{0.6em}}%
\MakeFramed{\advance\hsize-\width \FrameRestore}}%
{\endMakeFramed}
\declaretheoremstyle[
spaceabove=6pt,
spacebelow=6pt
headfont=\normalfont\bfseries,
headpunct={} ,
headformat={\cornersize*{2pt}\ovalbox{\NAME~\NUMBER\ifstrequal{\NOTE}{}{\relax}{\NOTE}:}},
bodyfont=\normalfont,
]{exobreak}

\declaretheorem[style=exobreak, name=Exercise,%
postheadhook=\leavevmode\myleftbar, %
prefoothook = \endmyleftbar]{exo}

% Solution environment 
\newenvironment{mysolbar}{%
\def\FrameCommand{\hspace{0.6em}\vrule width 2pt\hspace{0.6em}}%
\MakeFramed{\advance\hsize-\width \FrameRestore}}%
{\endMakeFramed}
\declaretheoremstyle[
spaceabove=6pt,
spacebelow=6pt
headfont=\normalfont\bfseries,
headpunct={} ,
headformat={\cornersize*{2pt}\ovalbox{\NAME~\NUMBER\ifstrequal{\NOTE}{}{\relax}{\NOTE}:}},
bodyfont=\normalfont,
]{solbreak}

\declaretheorem[style=solbreak, name=Solution,%
postheadhook=\leavevmode\mysolbar, %
prefoothook = \endmysolbar]{sol}

% HEADERS
\usepackage{fancyhdr}
 
\pagestyle{fancy}
\fancyhf{}
\fancyhead[LE,RO]{Page \thepage}
\fancyhead[RE,LO]{Math 454: Analysis 3}
\fancyfoot[CE,CO]{}
\fancyfoot[LE,RO]{\thepage}

% Definitions
\newcommand{\dfn}[1]{\textbf{\textcolor{blue}{#1}}}
\newcommand{\im}[1]{\textbf{\textcolor{red}{#1}}}

% lower integral
\usepackage{accents}

\newcommand{\ubar}[1]{\underaccent{\bar}{#1}}
\def\avint{\mathop{\,\rlap{-}\!\!\int}\nolimits} 

% custom commands 
\newcommand{\bb}[1]{\mathbb{#1}}
\newcommand{\vc}[1]{\mathbf{#1}}
\newcommand{\step}[1]{\textbf{#1}\textbf{. Step:}}
\newcommand{\pdv}[2]{\frac{\partial #1}{\partial #2}}
\newcommand{\sets}[2]{ \left\{ #1\ |\ #2 \right\}}
\DeclareMathOperator{\Tr}{Tr}

% Proofs
\newcommand{\claim}[1]{\textbf{#1}\textbf{. Claim:}}

	% iff proofs
	\newcommand{\rhs}[0]{(\Rightarrow )}
	\newcommand{\lhs}[0]{(\Leftarrow )}

% sequence of functions
\newcommand{\funcseqx}{(f_n(x))_{n \in \bb{N}}}
\newcommand{\funcseq}{(f_n)_{n \in \bb{N}}}

% measurable sets 
\newcommand{\measurable}{f^{-1}([-\infty, c[)} 

% heat equation 
\newcommand{\pbdry}[2]{C^{(#1, #2)} (\Omega_T) \cap C (\overline{\Omega_T})}
\DeclareMathOperator\erf{erf}
\newcommand{\mbf}[1]{\mathbf{#1}}

% Laplace Equation 
\newcommand{\lapbdry}[1]{C^{#1} (\Omega) \cap C (\overline{\Omega})}


% math environments 
\usepackage[utf8]{inputenc}
\newtheorem{theorem}{\textcolor{blue}{Theorem}}
\newtheorem{corollary}{Corollary}
\newtheorem{lemma}[theorem]{Lemma}
\theoremstyle{definition}
\newtheorem{definition}{\textcolor{OliveGreen}{Definition}}
\newtheorem{prop}{\textcolor{red}{Proposition}}
\theoremstyle{remark}
\newtheorem*{remark}{Remark}

% cookbook proofs 
\newcommand{\cb}[3]{\underline{(#1 #2): #3:}}

\usepackage{tcolorbox}
\tcbuselibrary{theorems}

% theorems 
\newtcbtheorem[number within=section]{mytheo}{Theorem}%
{colback=blue!5,colframe=blue!35!black,fonttitle=\bfseries}{th}

% definitions 
\newtcbtheorem[number within=section]{defn}{Definition}%
{colback=black!5,colframe=black!35!black,fonttitle=\bfseries}{th}

% axioms
\newtcbtheorem[number within=section]{ax}{Axioms}%
{colback=OliveGreen!5,colframe=black!35!OliveGreen,fonttitle=\bfseries}{th}


% important examples
\newtcbtheorem[number within=section]{examp}{Example}%
{colback=Mahogany!5,colframe=black!35!Mahogany,fonttitle=\bfseries}{th}

% upper and lower riemann integrals
\newcommand{\upRiemannint}[2]{
  \overline{\int_{#1}^{#2}}
}
\newcommand{\loRiemannint}[2]{
  \underline{\int_{#1}^{#2}}
}


\setlength{\headheight}{20pt}
\setlength{\headsep}{0.25 in}
\setlength{\parindent}{0 in}
\setlength{\parskip}{0.1 in}
\sloppy

%% Begin the header/lecture structure 
\newcommand{\lecture}[5]{
   \pagestyle{fancy}
   \fancyhf{}
   \fancyhead[LE,RO]{\thepage}
   \fancyhead[CE,CO]{Lecture #1: #2}
   \thispagestyle{plain}
   \setcounter{lecnum}{#1}
   \setcounter{page}{1}
   \noindent
   \vspace*{-.5in}
   \setlength{\fboxsep}{3mm}
   \setlength{\fboxrule}{1.5pt}
   \begin{center}
   \framebox{
      \vbox{
      \hbox to 6.18in {\textsc{Winter 2020 Semester (Results, Definitions, and Theorems) \hfill Lecture: #1}}
      \vspace{6mm}
      \hbox to 6.18in {{\bf \Large \hfill #2  \hfill}}
      \vspace{6mm}
      \hbox to 6.18in {Class:~ Math 455 (Analysis 4) 
\hfill Date:~{#3} \hfill Shereen Elaidi}}}
   \end{center}
   \vspace*{4mm}
}
%% End the header structure
% number everything using the lecture number counter
% so there is never any question as to whether we
% are referring to an example, theorem, exercise, etc.
\newcounter{lecnum}
\renewcommand{\thepage}{\thelecnum-\arabic{page}}
\renewcommand{\thesection}{\thelecnum.\arabic{section}}
\renewcommand{\theequation}{\thelecnum.\arabic{equation}}
\renewcommand{\thefigure}{\thelecnum.\arabic{figure}}
\renewcommand{\thetable}{\thelecnum.\arabic{table}}

\begin{document}

\lecture{08}{Chapter 8: The $L^p$ Spaces: Duality and Weak Convergence}{7 January 2020}{Shereen Elaidi}{}

\begin{abstract}
	This document contains a summary of all the key definitions, results, and theorems from class. There are probably typos, and so I would be grateful if you brought those to my attention :-). 
	
	Syllabus: $L^p$ space, duality, weak convergence, Young, Holder, and Minkowski inequalities, point-set topology, topological space, dense sets, completeness, compactness, connectedness, path-connectedness, separability, Tychnoff theorem, Stone-Weierstrass Theorem, Arzela-Ascoli, Baire category theorem, open mapping theorem, closed graph theorem, uniform boudnedness principle, Hahn Banch theorem. 
\end{abstract}

\section{Riesz Representation Theorem for the Dual of $L^p$, $1 \leq p < \infty$}
\begin{definition}[Linear Functional]
	A \textbf{linear functional} on a linear space $X$ is a real-valued function $T$ on T such that for $g$ and $g$ in $X$ and $\alpha$ and $\beta$ real numbers, 
	\begin{align}
		T(\alpha \cdot g + \beta \cdot h ) = \alpha \cdot T(g) + \beta \cdot T(h) 
	\end{align}
\end{definition}

\begin{definition}[Bounded]
	For a normed linear space $X$, a linear functional $T$ on $X$ is said to be \textbf{bounded} provided there is an $M \geq 0$ for which 
	\begin{align}
		|T(f)| \leq M \cdot ||f|| \text{ for all } f \in X
	\end{align}
	The infimum of all such $M$ is called the \textbf{norm} of $T$ and is denoted by $||T||_*$. 
\end{definition}

\begin{prop}[Continuity Property of a Bounded Linear Functional] 
	Let $T$ be a bounded linear functional on the normed space $X$. Then, if $\{ f_n \} \rightarrow f$ in $X$, then $\{ T(f_n) \} \rightarrow \{ T(f) \} $. 
\end{prop}

\begin{prop}
	Let $X$ be a normed vector space. Then, the collection of bounded linear functionals on $X$ is a linear space which is normed by $|| \cdot ||_*$. This normed vector space is called the \textbf{dual space} of $X$, and is denoted by $X^*$. 
\end{prop}

\begin{prop}
	Let $E \subseteq \R$ be measurable, $1 \leq p < \infty$, q the conjugate of $p$, $g \in L^q(E)$. Define the functional $T$ on $L^p(E)$ by: 
	\begin{align}
		T(f) := \int_E g \cdot f \text{ 		} \forall f \in L^p(E) 
	\end{align}
	Then, $T$ is a bounded linear functional on $L^p(E)$ and $||T||_* = ||g||_q$. 
\end{prop}

\begin{prop}
	Let $T$, $S$ be bounded linear functionals on the normed vector space $X$. If $T=S$ on a dense subset $X_0$ of $X$, then $T=S$. 
\end{prop}

\begin{lemma}
	\EinR, $1 \leq p < \infty$. Suppose that $g$ is integrable over $E$ and there exists a $M \geq 0$ for which 
	\begin{align*}
		\left| 		\int_E g \cdot f \right| \leq M || f||_p 	\text{ 		} \forall f \in L^p(E),\ f \text{ simple} 
	\end{align*}
	Then, $g \in L^q(E)$, where $q$ is the conjugate of $p$. Moreover, $||g||_q \leq M$. 
\end{lemma}

\begin{theorem}
	Let $[a,b]$ be a closed, bounded interval, and $1 \leq p < \infty$. Suppose that $T$ is a bounded linear functional on $L^p[a,b]$. Then, there is a functional $g \in L^q[a,b]$, where $q$ is the conjugate of $p$, for which: 
	\begin{align}
		T(f)= \idx{a}{b} g \cdot f \text{ 	} \forall f \in L^p[a,b] 
	\end{align}
\end{theorem}

\begin{theorem}[Riesz-Representation Theorem for the Dual of $L^p(E)$] 
	\EinR, $1 \leq p < \infty$, and q the conjugate of p. For all $g \in L^q(E)$, define the bounded linear functional $\mathcal{R}_g$ on $L^p(E)$ by: 
	\begin{align}
		\mathcal{R}_g := \idx{E}{} g \cdot f \text{ 		} \forall f \in L^p(E) 
	\end{align}
	Then, for each bounded linear functional $T$ on $L^p(E)$, there exists a unique $g \in L^q(E)$ for which 
	\begin{enumerate}[noitemsep]
		\item $\mathcal{R}_g = T$ and 
		\item $||T||_* = ||g||_q$
	\end{enumerate}
\end{theorem}

\section{Weak Sequential Convergence in $L^p$}

\begin{definition}[Converge Weakly]
	Let $X$ be a normed vector space. A sequence $\{ f_n \}$ in $X$ is said to \textbf{converge weakly} in $X$ to $f$ provided that 
	\begin{align}
		\lim_{n \rightarrow \infty} T(f_n) = T(f) \text{ 		} \forall T \in X^* 
	\end{align}
	we write 
	\begin{align*}
		\{ f_n \} \warrow f 
	\end{align*}
	to mean that $f$ and each $f_n$ belong to $X$ and $\{ f_n \}$ converges weakly in $X$ to $f$. 
\end{definition}

\begin{definition}
	\EinR, $1 \leq p < \infty$, $q$ the conjugate of $p$. Then, $\fcvw$ $\iff$ 
	\begin{align}
		\lim_{n \rightarrow \infty} \idx{E}{} g \cdot f_n = \idx{E}{} g \cdot f \text{ 		} \forall g \in L^q(E) 
	\end{align}
\end{definition}

\begin{theorem}
	\EinR, $1 \leq p < \infty$. Suppose that $\fcvw$. Then: 
	\begin{align*}
		\text{ $\{f_n\}$ is bounded and } ||f||_p \leq \liminf ||f_n||_p 	
	\end{align*}
\end{theorem}

\begin{corollary}
	\EinR, $1 \leq p < \infty$, $q$ the conjugate of $p$. Suppose $\{f_n\}$ converges weakly to $f$ in $L^p(E)$ and $\{ g_n \}$ converges strongly to $g \in L^q(E)$. Then: 
	\begin{align}
		\lim_{n \rightarrow \infty} \idx{E}{} g_n \cdot f_n = \idx{E}{} g \cdot f
	\end{align}
\end{corollary}

\begin{definition}[Linear Span] 
	Let $X$ be a normed vector space, and let $S \subseteq X$. Then, the \textbf{linear span of $S$} is the vector space consisting of all linear functionals of the form: 
	\begin{align}
		f = \sum_{k=1}^n \alpha_k \cdot f_k 
	\end{align}
	where each $\alpha_k \in \R$ and $f_k \in S$. It is the set of all \emph{finite linear combinations of elements in $S$}. We care about this since $L^p$ is an infinite dimensional  space, so we want to find a way to approximate it with finitely many elements. 
\end{definition}

\begin{prop}[Characterisation of Weak Convergence in $L^p(E)$] 
	\EinR, $1 \leq p < \infty$, q the conjugate of $P$. Assume that $\mathcal{F} \subseteq L^q(E)$ whose linear span is dense in $L^q(E)$. Let $\{f_n \}$ be a bounded sequence in $L^p(E)$, and let $f \in L^p(E)$. Then, $\fcvw$ $\iff$ 
	\begin{align}
		\lim_{n \rightarrow \infty} \idx{E}{} f_n \cdot g = \idx{E}{} f \cdot g \text{ 		} \forall g \in \mathcal{F} 
	\end{align}
\end{prop}

\begin{theorem}
	\EinR, $1 \leq p < \infty$. Suppose that $\{f_n \}$ is a bounded sequence in $L^p(E)$ and $f$ belongs to $L^p(E)$. Then, $\fcvw$ $\iff$ $\forall$ measurable sets $A \subseteq E$: 
	\begin{align}
		\lim_{n \rightarrow \infty} \idx{A}{} f_n = \idx{A}{} f
	\end{align}
	if $p > 1$, then it is sufficient to consider sets $A$ of finite measure. 
\end{theorem}

\begin{theorem}
	Let $[a,b]$ be a closed and bounded interval, $1 < p < \infty$. Suppose that $\{ f_n \}$ is a bounded sequence in $L^p[a,b]$ and $f \in L^p[a,b]$. Then, $\fcvw$ in $L^p[a,b]$ $\iff$ 
	\begin{align}
		\lim_{n \rightarrow \infty} \left[		\idx{a}{x} f_n	\right] = \idx{a}{x} f \text{ 			} \forall x \in [a,b] 
	\end{align}
\end{theorem}

\begin{lemma}[Riemann-Lebesgue Lemma; used in Fourier Series :-)] 
	Let $I= [-\pi, \pi]$, $1 \leq p < \infty$. $\forall n \in \mathbb{N}$, define $f_n(x):= \sin(nx)$ for $x \in I$. Then, $\{f_n\}$ converges weakly in $L^p(I)$ to $f \equiv 0$. 
\end{lemma}

\begin{theorem}
	\EinR, $1 < p < \infty$. Suppose that $\{ f_n \}$ is a bounded sequence in $L^p(E)$ that converges pointwise a.e. on $E$ to $f$. Then, $\fcvw$. 
\end{theorem}
This theorem was used in the proof but was not covered in Analysis 3: 
\begin{quote}
	\begin{theorem}[Vitali Convergence Theorem] 
		\EinR\  and of finite measure. Suppose that the sequence of functions $\{ f_n \}$ is uniformly integrable over $E$. Then, if $\{f_n \} \rightarrow f$ pointwise a.e. on $E$, then $f$ is integrable over $E$  and $\lim_{n \rightarrow \infty} \idx{E}{} f_n = f$. 
	\end{theorem}
\end{quote}

\begin{theorem}[Radon-Riesz Theorem]
	\EinR, $1 < p < \infty$. Suppose that $\fcvw$. Then: 
	\begin{align}
		\{f_n \} \rightarrow f \text{ in } L^p(E) \iff \lim_{n \rightarrow \infty} ||f_n||_p = ||f||_p 
	\end{align}
\end{theorem}

\begin{corollary} (Not Covered in Class): \EinR and $1 < p < \infty$. Suppose that $\fcvw$. Then, a subsequence of $\{f_n \}$ converges strongly to $f$ $\iff$ $||f||_p = \liminf||f_n||_p$. 
\end{corollary}

\section{Weak Sequential Compactness (``Compactness Found!'')} 

\begin{theorem}
	\EinR, $1 < p < \infty$. Then, every bounded sequence in $L^p(E)$ has a subsequence that converges weakly in $L^p(E)$ to a function in $L^p(E)$. 
\end{theorem}

\begin{theorem}[Helly's Theorem]
	Let $X$ be a \emph{SEPARABLE} normed vector space and $\{ T_n \}$ a sequence in its dual space $X^*$ that is bounded; that is, $\exists$ a $M > 0$ for which 
	\begin{align*}
		|T_n(f)| \leq M \cdot ||f|| \text{ 		} \forall f \in X,\ \forall n \in \mathbb{N} 	
	\end{align*}
	Then, there is a subsequence $\{ T_{n_k} \}$ of $\{ T_n \}$ and $T \in X^*$ for which 
	\begin{align}
		\lim_{k \rightarrow \infty} T_{n_k} (f) = T(f) \text{ 		} \forall f \in X 
	\end{align}
\end{theorem}

\begin{definition}[Weakly Sequentially Compact (Compact in the ``weak topology''] Let $X$ be a normed vector space. Then, a subset $K \subseteq X$ is \textbf{weakly sequentially compact} in $X$ provided that every sequence $\{ f_n \}$  in $K$ has a subsequence that converges weakly to $f \in K$. 
\end{definition}

\begin{theorem}[The Unit Ball is Weakly Sequentially Compact] \EinR, $1 < p < \infty$. Define: 
\begin{align}
	B_1:= \sets{f \in L^p(E)}{||f||_p \leq 1}. 
\end{align}
$B_1$ is weakly sequentially compact in $L^p(E)$. 

	
\end{theorem}
\end{document} 