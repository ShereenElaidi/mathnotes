\documentclass[11pt]{article}

% Packages
\usepackage[margin=2cm]{geometry}
\usepackage{amsmath} 
\usepackage{enumitem} 
\usepackage{amsfonts}
\usepackage{amsthm}


% Examples, definitions, theorems, etc
\theoremstyle{definition}
\newtheorem{ex}{Example}[section]
\newtheorem{defn}{Definition}[section]
\newtheorem{rmk}{Remark}[section]
\newtheorem{prop}{Proposition}[section]
\newtheorem{lem}{Lemma}[section]

\theoremstyle{theorem}
\newtheorem{thm}{Theorem}[section]

% Short-cuts
\newcommand{\R}[0]{\mathbb{R}}
\newcommand{\N}[0]{\mathbb{N}}

\newcommand{\prob}[1]{\mathbb{P}(#1)}
\newcommand{\comp}[1]{{#1}^{\texttt{C}}}
\newcommand{\borel}[0]{\mathcal{B}(\R)}
\newcommand{\pisys}[0]{\mathcal{I}}
\newcommand{\dsys}[0]{\mathcal{D}}


% physics short-cuts
\newcommand{\ket}[1]{| #1 \rangle}
\newcommand{\bra}[1]{\langle #1|}


\begin{document}
\begin{center}
	\textbf{PHYS 551: Quantum Theory} \\
	\textbf{Shereen Elaidi}
\end{center}

\section{Fundamentals}
The objective of this part of the course is to review mathematical language. 
\subsection{Hilbert Spaces}
A \textbf{Hilbert Space} is a generalized vector space. In Quantum mechanics, a physical state is represented by a \textbf{state vector}. A quantum state corresponds to a vector in the Hilbert Space. In Dirac notation, we describe a quantum state with a \textbf{ket}:
\begin{align}
	\ket{\psi}.
\end{align}
Consider \( 2 \ket{\psi} \). This disrupts the normalization, but it still represents the same state. Hence, we don't think of quantum states as vectors, but rather as a ``ray in Hilbert Space.'' It's the direction that counts. We usually will normalize it to unit length.
\newline
\newline
The dimension of our Hilbert Space depends on the system we're studying. If we are studying spin \(1/2 \) systems, then the Hilbert space would be two-dimensional:
\begin{align*}
	\ket{\uparrow} \text{ and } \ket{\downarrow}. 	
\end{align*}
Here, \( N = 2 \) and \( \ket{\uparrow} \text{ and } \ket{\downarrow} \) are the eigenvectors of the spin operator \( \hat{S}_z \). In general, the dimension of our Hilbert Space is the amount of numbers you need to describe the state. For example, we can consider a particle in a square box. What would the dimension of that Hilbert Space be? Since there are infinitely-many stationary states, \( N = \infty \). We would write: 
\begin{align*}
	\ket{\psi} = \sum c_n \ket{n}, 	
\end{align*}
where \( \ket{\psi} \) represents the state of the system and \( c_n \) is the weight corresponding to the energy level \( \ket{n} \). The \( \ket{n} \) correspond to the eigenstates of the Hamiltonian operator. We could also specify the wave function: 
\begin{align}
	\ket{\psi} = \int dx \psi(x) \ket{x}.
\end{align}
Here, \( \ket{x} \) corresponds to the position eigenstates: this is a continuous variable. This is an infinite Hilbert Space since \( \psi(x) \) is a \emph{continuous function}. 

\subsection{Inner Product and Dual Space}
For every ket \( \ket{\psi} \), there is an associated \textbf{bra} \( \bra{\psi} \), which lives in the dual space. Hence, there is a one-to-one correspondence between the Hilbert Space and the Dual space. 


\end{document}