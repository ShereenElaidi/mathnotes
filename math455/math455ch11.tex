\documentclass[11pt]{article}
\usepackage{titlesec}
\titleformat{\section}[hang]{\normalfont\scshape}{\thesection.}{1em}{}
\usepackage[T1]{fontenc}
\usepackage[dvipsnames]{xcolor}
\usepackage[margin=2cm]{geometry}
\usepackage{amssymb} 
\usepackage{amsmath}
\usepackage{graphicx}
\usepackage{float}
\usepackage[normalem]{ulem} 
\usepackage{enumitem} 
\setlist[enumerate]{itemsep=0mm}
\usepackage{xcolor}
\setenumerate{label=(\roman*)}
\usepackage[utf8]{inputenc}
\usepackage[T1]{fontenc}
\usepackage{babel}
\usepackage{mathtools}
\usepackage{amsthm}
\usepackage{thmtools}
\usepackage{etoolbox}
\usepackage{fancybox}
\usepackage{framed}
\usepackage{tcolorbox}
\usepackage{xcolor} 
% open 
\newcommand{\open}[0]{\mathcal{O}}
\newcommand{\topo}[0]{\mathcal{T}}
\newcommand{\hood}[0]{\mathcal{U}}
\newcommand{\base}[0]{\mathcal{B}} 
% example environmentq
\theoremstyle{definition} 
\newtheorem{exmp}{Example}[section]


% question environment
\theoremstyle{definition}
\newtheorem{question}{Question}

%rd 
\newcommand{\rd}[0]{\mathbb{R}^d}
\newcommand{\R}[0]{\mathbb{R}}

% let E in R be measurable 
\newcommand{\EinR}[0]{Let $E \subseteq \R$ be measurable}

%integral 
\newcommand{\idx}[2]{\int_{#1}^{#2}}

% weak convergence 
\newcommand{\warrow}[0]{\rightharpoonup}
\newcommand{\fcvw}[0]{ \{f_n \} \warrow f \text{ in } L^p(E)} 

% Probability 
\DeclareRobustCommand{\bbone}{\text{\usefont{U}{bbold}{m}{n}1}}
\newcommand{\Var}[1]{\mathrm{Var[#1]}}			% variance
\newcommand{\EX}[1]{\mathbb{E}\mathrm{[#1]}}	 % expected value 
\newcommand{\seq}[1]{\{ #1_n	\}_{n \in \bb{N}}} % sequence of events
\newcommand{\pspace}[0]{( \Omega, F, P)}		% probability space
\newcommand{\msp}[0]{( \Omega, F)}		% measurable space
	
% Exercise environment 
\newenvironment{myleftbar}{%
\def\FrameCommand{\hspace{0.6em}\vrule width 2pt\hspace{0.6em}}%
\MakeFramed{\advance\hsize-\width \FrameRestore}}%
{\endMakeFramed}
\declaretheoremstyle[
spaceabove=6pt,
spacebelow=6pt
headfont=\normalfont\bfseries,
headpunct={} ,
headformat={\cornersize*{2pt}\ovalbox{\NAME~\NUMBER\ifstrequal{\NOTE}{}{\relax}{\NOTE}:}},
bodyfont=\normalfont,
]{exobreak}

\declaretheorem[style=exobreak, name=Exercise,%
postheadhook=\leavevmode\myleftbar, %
prefoothook = \endmyleftbar]{exo}

% Solution environment 
\newenvironment{mysolbar}{%
\def\FrameCommand{\hspace{0.6em}\vrule width 2pt\hspace{0.6em}}%
\MakeFramed{\advance\hsize-\width \FrameRestore}}%
{\endMakeFramed}
\declaretheoremstyle[
spaceabove=6pt,
spacebelow=6pt
headfont=\normalfont\bfseries,
headpunct={} ,
headformat={\cornersize*{2pt}\ovalbox{\NAME~\NUMBER\ifstrequal{\NOTE}{}{\relax}{\NOTE}:}},
bodyfont=\normalfont,
]{solbreak}

\declaretheorem[style=solbreak, name=Solution,%
postheadhook=\leavevmode\mysolbar, %
prefoothook = \endmysolbar]{sol}

% HEADERS
\usepackage{fancyhdr}
 
\pagestyle{fancy}
\fancyhf{}
\fancyhead[LE,RO]{Page \thepage}
\fancyhead[RE,LO]{Math 455: Analysis 4}
\fancyfoot[CE,CO]{}
\fancyfoot[LE,RO]{\thepage}

% Definitions
\newcommand{\dfn}[1]{\textbf{\textcolor{blue}{#1}}}
\newcommand{\im}[1]{\textbf{\textcolor{red}{#1}}}

% lower integral
\usepackage{accents}

\newcommand{\ubar}[1]{\underaccent{\bar}{#1}}
\def\avint{\mathop{\,\rlap{-}\!\!\int}\nolimits} 

% custom commands 
\newcommand{\bb}[1]{\mathbb{#1}}
\newcommand{\vc}[1]{\mathbf{#1}}
\newcommand{\step}[1]{\textbf{#1}\textbf{. Step:}}
\newcommand{\pdv}[2]{\frac{\partial #1}{\partial #2}}
\newcommand{\sets}[2]{ \left\{ #1\ |\ #2 \right\}}
\DeclareMathOperator{\Tr}{Tr}

% Proofs
\newcommand{\claim}[1]{\textbf{#1}\textbf{. Claim:}}

	% iff proofs
	\newcommand{\rhs}[0]{(\Rightarrow )}
	\newcommand{\lhs}[0]{(\Leftarrow )}

% sequence of functions
\newcommand{\funcseqx}{(f_n(x))_{n \in \bb{N}}}
\newcommand{\funcseq}{(f_n)_{n \in \bb{N}}}

% measurable sets 
\newcommand{\measurable}{f^{-1}([-\infty, c[)} 

% heat equation 
\newcommand{\pbdry}[2]{C^{(#1, #2)} (\Omega_T) \cap C (\overline{\Omega_T})}
\DeclareMathOperator\erf{erf}
\newcommand{\mbf}[1]{\mathbf{#1}}

% Laplace Equation 
\newcommand{\lapbdry}[1]{C^{#1} (\Omega) \cap C (\overline{\Omega})}


% math environments 
\usepackage[utf8]{inputenc}
\newtheorem{theorem}{\textcolor{blue}{Theorem}}
\newtheorem{corollary}{Corollary}
\newtheorem{lemma}[theorem]{Lemma}
\theoremstyle{definition}
\newtheorem{definition}{\textcolor{OliveGreen}{Definition}}
\newtheorem{prop}{\textcolor{red}{Proposition}}
\newtheorem{ex}{\textcolor{Maroon}{Example}}
\theoremstyle{remark}
\newtheorem*{remark}{Remark}

% cookbook proofs 
\newcommand{\cb}[3]{\underline{(#1 #2): #3:}}

\usepackage{tcolorbox}
\tcbuselibrary{theorems}

% theorems 
\newtcbtheorem[number within=section]{mytheo}{Theorem}%
{colback=blue!5,colframe=blue!35!black,fonttitle=\bfseries}{th}

% definitions 
\newtcbtheorem[number within=section]{defn}{Definition}%
{colback=black!5,colframe=black!35!black,fonttitle=\bfseries}{th}

% axioms
\newtcbtheorem[number within=section]{ax}{Axioms}%
{colback=OliveGreen!5,colframe=black!35!OliveGreen,fonttitle=\bfseries}{th}


% important examples
\newtcbtheorem[number within=section]{examp}{Example}%
{colback=Mahogany!5,colframe=black!35!Mahogany,fonttitle=\bfseries}{th}

% upper and lower riemann integrals
\newcommand{\upRiemannint}[2]{
  \overline{\int_{#1}^{#2}}
}
\newcommand{\loRiemannint}[2]{
  \underline{\int_{#1}^{#2}}
}


\setlength{\headheight}{20pt}
\setlength{\headsep}{0.25 in}
\setlength{\parindent}{0 in}
\setlength{\parskip}{0.1 in}
\sloppy

%% Begin the header/lecture structure 
\newcommand{\lecture}[5]{
   \pagestyle{fancy}
   \fancyhf{}
   \fancyhead[LE,RO]{\thepage}
   \fancyhead[CE,CO]{Lecture #1: #2}
   \thispagestyle{plain}
   \setcounter{lecnum}{#1}
   \setcounter{page}{1}
   \noindent
   \vspace*{-.5in}
   \setlength{\fboxsep}{3mm}
   \setlength{\fboxrule}{1.5pt}
   \begin{center}
   \framebox{
      \vbox{
      \hbox to 6.18in {\textsc{Winter 2020 Semester (Results, Definitions, and Theorems) \hfill Lecture: #1}}
      \vspace{6mm}
      \hbox to 6.18in {{\bf \Large \hfill #2  \hfill}}
      \vspace{6mm}
      \hbox to 6.18in {Class:~ Math 455 (Analysis 4) 
\hfill Date:~{#3} \hfill Shereen Elaidi}}}
   \end{center}
   \vspace*{4mm}
}
%% End the header structure
% number everything using the lecture number counter
% so there is never any question as to whether we
% are referring to an example, theorem, exercise, etc.
\newcounter{lecnum}
\renewcommand{\thepage}{\thelecnum-\arabic{page}}
\renewcommand{\thesection}{\thelecnum.\arabic{section}}
\renewcommand{\theequation}{\thelecnum.\arabic{equation}}
\renewcommand{\thefigure}{\thelecnum.\arabic{figure}}
\renewcommand{\thetable}{\thelecnum.\arabic{table}}

\begin{document}

\lecture{011}{Chapter 11: Topological Spaces (General Properties)}{9 February 2020}{Shereen Elaidi}{}

\begin{abstract}
	This document contains a summary of all the key definitions, results, and theorems from class. There are probably typos, and so I would be grateful if you brought those to my attention :-). 
	
	Syllabus: $L^p$ space, duality, weak convergence, Young, Holder, and Minkowski inequalities, point-set topology, topological space, dense sets, completeness, compactness, connectedness, path-connectedness, separability, Tychnoff theorem, Stone-Weierstrass Theorem, Arzela-Ascoli, Baire category theorem, open mapping theorem, closed graph theorem, uniform boudnedness principle, Hahn Banch theorem. 
\end{abstract}

\section{Open Sets, Closed Sets, Bases, and Sub-bases}

\begin{definition}[Open Sets] 
	Let $X$ be a non-empty set. A \textbf{topology} $\topo$ for $X$ is a collection of subsets of $X$, called \textbf{open sets}, posessing the following properties: 
	\begin{enumerate}[noitemsep]
		\item The entire set $X$ and the empty set $\emptyset$ are open. 
		\item The finite intersection of open sets are open. 
		\item The union of any collection of open sets is open. 
	\end{enumerate}
	A non-empty set $X$, together with a topology on $X$, is called a \textbf{topological space}. For a point $x \in X$, an open set that contains $x$ is called a \textbf{neighbourhood} of $x$. 
\end{definition}

\begin{prop}
	A subset $E \subseteq X$ is open $\iff$ for each $x \in E$, there exists a neighbourhood of $x$ that is contained in $E$. 
\end{prop}

\begin{ex}[Metric Topology]
	Let $(X, \rho)$ be a metric space. Let $\open \subseteq X$ be  open if for all $x \in \open$, $\exists$ an open ball at $x$ that is contained in $\open$. This collection of open sets forms a topology; we call this the \textbf{metric topology} induced by $\rho$. 
\end{ex}

\begin{ex}[Discrete Topology] 
	This topology is ``too much.'' Let $X$ be a non-empty subset. Let $\topo := \mathcal{P}(X)$. Then, every set containing a point is a neighbourhood of that point. This is induced by the discrete metric. 
\end{ex}

\begin{ex}[Trivial Topology]
	Let $X$ be non-empty. Define $\topo := \{ X, \emptyset \}$. The only neighbourhood of any point is the whole set $X$. 	
\end{ex}

\begin{definition}[Topological Subspaces]
	Let $(X, \topo)$ be a topological space and let $E$ be a non-empty subset of $X$. The inherited topology $\mathcal{S}$ for $E$ is the set of all sets of the form $E \cap \topo$, where $\open \in \topo$. The topological space $(E, \mathcal{S})$ is called a \textbf{subspace} of $(X, \topo)$. 
\end{definition}

\begin{definition}[Base for the Topology]
	The building blocks of a topology is called a \textbf{base}. Let $(X, \topo)$ be a topological space. For a point $x \in X$, a collection of neighbourhoods of $x$, $B_x$, is called a \textbf{base for the topology at $X$} if $\forall$ neighbourhoods $\mathcal{U}$ of $x$, there exists a set $B$ in the collection $B_x$ for which $B \subseteq \mathcal{U}$. 
	
	A collection of open sets $\mathcal{B}$ is called a \textbf{base for the topology} $\topo$ provided it contains a base for the topology at each point. 
\end{definition}

\begin{center} 
\textbf{A base for a topology completely determines a topology, alongside $\emptyset$ and $X$.}
\end{center} 

\begin{prop}
	For a non-empty set $X$, let $\mathcal{B}$ be a collection of subsets of $X$. Then, $\mathcal{B}$ is a base for a topology for $X$ $\iff$: 
	\begin{enumerate}[noitemsep]
		\item $\mathcal{B}$ covers $X$. That is:
		\begin{align}
			X = \bigcup_{B \in \mathcal{B}} B
		\end{align}
		\item If $B_1, B_2 \in \mathcal{B}$, and $x \in B_1 \cap B_2$, then there is a set $B_3 \in \mathcal{B}$ for which $x \in B_3 \subseteq B_1 \cap B_2$. 
	\end{enumerate}
	The unique topology that has $\mathcal{B}$ as its base consists of $\emptyset$ and unions of sub-collections of $\mathcal{B}$. 
\end{prop}

\begin{definition}[Product Topology]
	Let $(X, \topo)$ and $(Y, \mathcal{S})$ be two topological spaces. In the cartesian product $X \times Y$, consider the collection of sets $\mathcal{B}$ containing the products $\open_1 \times \open_2$, where $\open_1$ is open in $X$ and $\open_2$ is open in $Y$. Then, $\mathcal{B}$ is a base for a topology on $X \times Y$, which we call the \textbf{product topology.}
\end{definition}

\begin{definition}[Sub-base]
	Let $(X, \topo)$ be a topological space. The collection of $\mathcal{S}$ of $\topo$ that covers $X$ is called a \textbf{sub-base} for the topology $\topo$ provided intersections of finite collections of $\mathcal{S}$ are a base for $\topo$. 
\end{definition}

\begin{definition}[Closure]
	Let $E \subseteq X$ be a subset of a topological space. A point $x \in E$ is called a \textbf{point of closure} of $E$ if every neighbourhood of $x$ contains a point in $E$. The collection of the points of closure of $E$ is called the \textbf{closure} of $E$, denoted $\overline{E}$. 
\end{definition}

\begin{prop}
	Let $X$ be a topological space, $E \subseteq X$. Then, $\overline{E}$ is closed. Moreover, $\overline{E}$ is the smallest closed subset of $X$ containing $E$ in the sense that if $F$ is closed and $E \subseteq F$, then $\overline{E} \subseteq F$. 
\end{prop}

\begin{prop}
	A subset of a topological space $X$ is open $\iff$ its complement is closed. 
\end{prop}

\begin{prop}
	Let $X$ be a topological space. Then, (a) $\emptyset$ and $X$ are closed, (b) the union of a finite collection of closed sets is closed, (c) the intersection of any collection of closed sets in $X$ is closed. 
\end{prop}

\section{Separation Properties}
\textbf{Motivation:} Separation properties for a topology allow us to discriminate between which topologies discriminate between certain disjoint pairs of sets, which will then allow us to study a robust collection of cts real-valued functions on $X$. 

\begin{definition}[Neighbourhood]
	A \textbf{neighbourhood} of $K$ for a subset $K \subseteq X$ is an open set that contains $K$. 
\end{definition} 

\begin{definition}[Separated by Neighbourhoods]
	We say that two disjoint sets $A$ and $B$ in $X$ can be separated by disjoint neighbourhoods provided that there exists neighbourhoods of $A$ and $B$, respectively, that are disjoint. 
\end{definition}

\begin{definition}[Separation Properties of Topological Spaces]. In the order of most general to least general, they are: 
\begin{enumerate}[noitemsep]
	\item \underline{\textbf{Tychonoff Separation Property}}: For each two points $u, v \in X$, there exists a neighbourhood of $u$ that does not contain $v$ and a neighbourhood of $v$ that does not contain $u$. 
	\item \underline{\textbf{Hausdorff Separation Property}}: Each two points in $X$ can be separated by disjoint neighbourhoods. 
	\item \underline{\textbf{Regular Separation Property}}: Tychonoff $+$ each closed set and a point not in the set can be separated by disjoint neighbourhoods.
	\item \underline{\textbf{Normal Separation Property}}: Tychonoff $+$ each two disjoint closed sets can be separated by disjoint neighbourhoods. 
\end{enumerate}
\end{definition}

\begin{prop}
	A topological space is Tychonoff $\iff$ every set containing a single point, $\{ x \}$, is closed. 
\end{prop}

\begin{prop}
	Every metric space is normal.
\end{prop}

\begin{lemma}
	$F$ is closed $\iff$ dist$(x, F) > 0$ $\forall$ $ x \notin F$. 
\end{lemma}

\begin{prop}
	Let $X$ be a Tychonoff topological space. Then, $X$ is normal $\iff$ whenever $\hood$ is a neighbourhood of a closed subset of $F$ of $X$, there is another neighbourhood of $F$ whose closure is contained in $\hood$. that is, there is an open set $\open$ for which: 
	\begin{align}
		F \subseteq \open \subseteq \overline{\open} \subseteq \hood 
	\end{align}
\end{prop}


\section{Countability and Separability}
\begin{definition}[Converge, Limit]
	A sequence $\{ x_n \}$ in a topological space $X$ is said to \textbf{converge} to the point $x \in X$ if for each neighbourhood $\hood$ of $x$, there exists an index $N \in \mathbb{N}$ such that if $n \geq N$, then $x_n$ belongs to $\hood$. This point is called a \textbf{limit} of the sequence. 
\end{definition}

\begin{definition}[First and Second Countable]
	A topological space $X$ is \textbf{first countable} if there is a countable base at each point. A space $X$ is said to be \textbf{second countable} if there is a countable base for the topology.
\end{definition}

\begin{ex}
Every metric space is first countable. 	
\end{ex}

\begin{prop}
	Let $X$ be a first countable topological space. For a subset $E \subseteq X$, a point $x \in X$ is called a point of closure of $E$ $\iff$ it is a limit of a sequence in $E$. Thus, a subset $E$ of $X$ is closed $\iff$ whenever a sequence in $E$ converges to $x \in X$, we have that $x \in E$. 
\end{prop}

\begin{definition}[Dense/Separable]
	A subset $E \subseteq X$ is \textbf{dense} in $X$ if every open set in $X$ contains a point of $E$. We call $X$ \textbf{separable} if it has a countable dense subset. 
\end{definition}

\begin{definition}[Metrisable]
	A topological space $X$ is said to be \textbf{metrisable} if the topology is induced by the metric. 
\end{definition}

\begin{theorem}
	Let $X$ be a second countable topological space. Then, $X$ is metrisable $\iff$ it is normal. 
\end{theorem}


\section{Continuous Mappings between Topological Spaces}
\begin{definition}[Continuous]
	For topological spaces $(X, \topo)$, $(Y, \mathcal{S})$, a mapping $f: X \rightarrow Y$ is said to be \textbf{continuous} at the point $x_0$ in $X$ if, for every neighbourhood $\open$ if $f(x_0)$, there is a neighbourhood $\hood$ of $x_0$ for which $f(\hood) \subseteq \open$. We say that $f$ is continuous provided it is continuous at each point in $X$.  
\end{definition}

\begin{prop}
	A mapping $f: X \rightarrow Y$ between topological spaces $X$ and $Y$ is continuous $\iff$ for any open subset $\open$ in $Y$, its inverse image under $f$, $f^{-1}(\open)$, is an open subset of $X$. 
\end{prop}

\begin{prop}
	The composition of continuous mappings between topological spaces, when defined, is continuous. 
\end{prop}

\begin{definition}[Stronger]
	Given two topologies $\topo_1$ and $\topo_2$ for a set $X$, if $\topo_2 \subseteq \topo_1$, then we say that $\topo_2$ is \textbf{weaker} than $\topo_1$, and that $\topo_1$ is \textbf{stronger} than $\topo_2$. 
\end{definition}

\begin{prop}
	Let $X$ be a non-empty set and let $\mathcal{S}$ be a collection of subsets of $X$ that covers $X$. The collection of subsets of $X$ consisting of intersections of finite collections of $\mathcal{S}$ is a base for a topology $\topo$ of $X$. It is the weakest topology containing $\mathcal{S}$ in the sense that if $\topo'$ is any other topology for $X$ containing $\mathcal{S}$, then $\topo \subseteq \topo'$. 
\end{prop}

\begin{definition}[Weak Topology]
	Let $X$ be a non-empty set and $\mathcal{F} := \{ f_\alpha\ |\ X \rightarrow X_\alpha \}_{\alpha \in \Lambda }$ a collection of mappings, where each $X_\alpha$ is a topological space. The weakest topology for $X$ that contains the collection of sets
	\begin{align}
		\{ f_\alpha^{-1} ( \open_\alpha)\ |\ f_\alpha \in \mathcal{F},\ \open_\alpha \text{ open in } X_\alpha \} 
	\end{align}
	is called the \textbf{weak topology for $X$ induced by $\mathcal{F}$.}
\end{definition}

\begin{prop} Let $X$ be a non-empty set, $\mathcal{F} := \{ f_\lambda\ |\ X \rightarrow X_\lambda \}_{\lambda \in \Lambda } $ a collection of mappings where each $X_\lambda$ is a topological space. The weak topology for $X$ induced by $\mathcal{F}$ is the topology on $X$ that has the fewest number of sets covering the topologies on $X$ for which each mapping $f_\lambda: X \rightarrow X_\lambda$ is continuous. 
\end{prop}

\begin{definition}[Homeomorphism]
	A mapping from a topological space $X \rightarrow Y$ is said to be a \textbf{homeomorphism} if it is bijective and has a continuous inverse $f^{-1}: Y \rightarrow X$. Two topological spaces are said to be \textbf{homeomorphic} if there exists a homeomorphism between them. The notion of homeomorphism induces a notion of an equivalence relation between spaces. 
\end{definition}

\section{Compact Topological Spaces}

\begin{definition}[Cover]
	A collection of sets $\{ E_\lambda \}_{\lambda \in \Lambda }$ is said to be a \textbf{cover} of a set $E$ if $E \subseteq \bigcup_{\lambda \in \Lambda } E_\lambda$. 
\end{definition}

\begin{definition}[Compact]
	A topological space $X$ is said to be \textbf{compact} if every open cover of $X$ has a finite sub-cover. A subset $K \subseteq X$ is compact if $K$, considered as a topological space with the subspace topology inherited from $X$, is compact. 
\end{definition}

\begin{prop}
	A topological space $X$ is compact $\iff$ every collection of closed subsets of $X$ that posesses the finite intersection property has non-empty intersection. 
\end{prop}

\begin{prop}
	A closed subset $K$ of a compact topological space is compact. 
\end{prop}

\begin{prop}
	A compact subspace $K$ of a Hausdorff topological space is a closed subset of $X$. 
\end{prop}

\end{document}