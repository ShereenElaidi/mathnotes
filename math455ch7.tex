\documentclass[11pt]{article}
\usepackage{titlesec}
\titleformat{\section}[hang]{\normalfont\scshape}{\thesection.}{1em}{}

\usepackage[dvipsnames]{xcolor}
\usepackage[margin=2cm]{geometry}
\usepackage{amssymb} 
\usepackage{amsmath}
\usepackage{graphicx}
\usepackage{float}
\usepackage[normalem]{ulem} 
\usepackage{enumitem} 
\setlist[enumerate]{itemsep=0mm}
\usepackage{xcolor}
\setenumerate{label=(\roman*)}
\usepackage[utf8]{inputenc}
\usepackage[T1]{fontenc}
\usepackage{babel}
\usepackage{mathtools}
\usepackage{amsthm}
\usepackage{thmtools}
\usepackage{etoolbox}
\usepackage{fancybox}
\usepackage{framed}
\usepackage{tcolorbox}
% example environment
\theoremstyle{definition} 
\newtheorem{exmp}{Example}[section]

% question environment
\theoremstyle{definition}
\newtheorem{question}{Question}

%rd 
\newcommand{\rd}[0]{\mathbb{R}^d}

% Probability 
\DeclareRobustCommand{\bbone}{\text{\usefont{U}{bbold}{m}{n}1}}
\newcommand{\Var}[1]{\mathrm{Var[#1]}}			% variance
\newcommand{\EX}[1]{\mathbb{E}\mathrm{[#1]}}	 % expected value 
\newcommand{\seq}[1]{\{ #1_n	\}_{n \in \bb{N}}} % sequence of events
\newcommand{\pspace}[0]{( \Omega, F, P)}		% probability space
\newcommand{\msp}[0]{( \Omega, F)}		% measurable space
	
% Exercise environment 
\newenvironment{myleftbar}{%
\def\FrameCommand{\hspace{0.6em}\vrule width 2pt\hspace{0.6em}}%
\MakeFramed{\advance\hsize-\width \FrameRestore}}%
{\endMakeFramed}
\declaretheoremstyle[
spaceabove=6pt,
spacebelow=6pt
headfont=\normalfont\bfseries,
headpunct={} ,
headformat={\cornersize*{2pt}\ovalbox{\NAME~\NUMBER\ifstrequal{\NOTE}{}{\relax}{\NOTE}:}},
bodyfont=\normalfont,
]{exobreak}

\declaretheorem[style=exobreak, name=Exercise,%
postheadhook=\leavevmode\myleftbar, %
prefoothook = \endmyleftbar]{exo}

% Solution environment 
\newenvironment{mysolbar}{%
\def\FrameCommand{\hspace{0.6em}\vrule width 2pt\hspace{0.6em}}%
\MakeFramed{\advance\hsize-\width \FrameRestore}}%
{\endMakeFramed}
\declaretheoremstyle[
spaceabove=6pt,
spacebelow=6pt
headfont=\normalfont\bfseries,
headpunct={} ,
headformat={\cornersize*{2pt}\ovalbox{\NAME~\NUMBER\ifstrequal{\NOTE}{}{\relax}{\NOTE}:}},
bodyfont=\normalfont,
]{solbreak}

\declaretheorem[style=solbreak, name=Solution,%
postheadhook=\leavevmode\mysolbar, %
prefoothook = \endmysolbar]{sol}

% HEADERS
\usepackage{fancyhdr}
 
\pagestyle{fancy}
\fancyhf{}
\fancyhead[LE,RO]{Page \thepage}
\fancyhead[RE,LO]{Math 454: Analysis 3}
\fancyfoot[CE,CO]{}
\fancyfoot[LE,RO]{\thepage}

% Definitions
\newcommand{\dfn}[1]{\textbf{\textcolor{blue}{#1}}}
\newcommand{\im}[1]{\textbf{\textcolor{red}{#1}}}

% lower integral
\usepackage{accents}

\newcommand{\ubar}[1]{\underaccent{\bar}{#1}}
\def\avint{\mathop{\,\rlap{-}\!\!\int}\nolimits} 

% custom commands 
\newcommand{\bb}[1]{\mathbb{#1}}
\newcommand{\vc}[1]{\mathbf{#1}}
\newcommand{\step}[1]{\textbf{#1}\textbf{. Step:}}
\newcommand{\pdv}[2]{\frac{\partial #1}{\partial #2}}
\newcommand{\sets}[2]{ \left\{ #1\ |\ #2 \right\}}
\DeclareMathOperator{\Tr}{Tr}

% Proofs
\newcommand{\claim}[1]{\textbf{#1}\textbf{. Claim:}}

	% iff proofs
	\newcommand{\rhs}[0]{(\Rightarrow )}
	\newcommand{\lhs}[0]{(\Leftarrow )}

% sequence of functions
\newcommand{\funcseqx}{(f_n(x))_{n \in \bb{N}}}
\newcommand{\funcseq}{(f_n)_{n \in \bb{N}}}

% measurable sets 
\newcommand{\measurable}{f^{-1}([-\infty, c[)} 

% heat equation 
\newcommand{\pbdry}[2]{C^{(#1, #2)} (\Omega_T) \cap C (\overline{\Omega_T})}
\DeclareMathOperator\erf{erf}
\newcommand{\mbf}[1]{\mathbf{#1}}

% Laplace Equation 
\newcommand{\lapbdry}[1]{C^{#1} (\Omega) \cap C (\overline{\Omega})}


% math environments 
\usepackage[utf8]{inputenc}
\newtheorem{theorem}{\textcolor{blue}{Theorem}}
\newtheorem{corollary}{Corollary}
\newtheorem{lemma}[theorem]{Lemma}
\theoremstyle{definition}
\newtheorem{definition}{\textcolor{OliveGreen}{Definition}}
\newtheorem{prop}{\textcolor{red}{Proposition}}
\theoremstyle{remark}
\newtheorem*{remark}{Remark}

% cookbook proofs 
\newcommand{\cb}[3]{\underline{(#1 #2): #3:}}

\usepackage{tcolorbox}
\tcbuselibrary{theorems}

% theorems 
\newtcbtheorem[number within=section]{mytheo}{Theorem}%
{colback=blue!5,colframe=blue!35!black,fonttitle=\bfseries}{th}

% definitions 
\newtcbtheorem[number within=section]{defn}{Definition}%
{colback=black!5,colframe=black!35!black,fonttitle=\bfseries}{th}

% axioms
\newtcbtheorem[number within=section]{ax}{Axioms}%
{colback=OliveGreen!5,colframe=black!35!OliveGreen,fonttitle=\bfseries}{th}


% important examples
\newtcbtheorem[number within=section]{examp}{Example}%
{colback=Mahogany!5,colframe=black!35!Mahogany,fonttitle=\bfseries}{th}

% upper and lower riemann integrals
\newcommand{\upRiemannint}[2]{
  \overline{\int_{#1}^{#2}}
}
\newcommand{\loRiemannint}[2]{
  \underline{\int_{#1}^{#2}}
}


\setlength{\headheight}{20pt}
\setlength{\headsep}{0.25 in}
\setlength{\parindent}{0 in}
\setlength{\parskip}{0.1 in}
\sloppy

%% Begin the header/lecture structure 
\newcommand{\lecture}[5]{
   \pagestyle{fancy}
   \fancyhf{}
   \fancyhead[LE,RO]{\thepage}
   \fancyhead[CE,CO]{Lecture #1: #2}
   \thispagestyle{plain}
   \setcounter{lecnum}{#1}
   \setcounter{page}{1}
   \noindent
   \vspace*{-.5in}
   \setlength{\fboxsep}{3mm}
   \setlength{\fboxrule}{1.5pt}
   \begin{center}
   \framebox{
      \vbox{
      \hbox to 6.18in {\textsc{Winter 2020 Semester (Results, Definitions, and Theorems) \hfill Lecture: #1}}
      \vspace{6mm}
      \hbox to 6.18in {{\bf \Large \hfill #2  \hfill}}
      \vspace{6mm}
      \hbox to 6.18in {Class:~ Math 455 (Analysis 4) 
\hfill Date:~{#3} \hfill Shereen Elaidi}}}
   \end{center}
   \vspace*{4mm}
}
%% End the header structure
% number everything using the lecture number counter
% so there is never any question as to whether we
% are referring to an example, theorem, exercise, etc.
\newcounter{lecnum}
\renewcommand{\thepage}{\thelecnum-\arabic{page}}
\renewcommand{\thesection}{\thelecnum.\arabic{section}}
\renewcommand{\theequation}{\thelecnum.\arabic{equation}}
\renewcommand{\thefigure}{\thelecnum.\arabic{figure}}
\renewcommand{\thetable}{\thelecnum.\arabic{table}}

\begin{document}

\lecture{07}{Chapter 7: $L^p$ Spaces: Completeness and Approximation}{7 January 2020}{Shereen Elaidi}{}

\begin{abstract}
	This document contains a summary of all the key definitions, results, and theorems from class. There are probably typos, and so I would be grateful if you brought those to my attention :-). 
	
	Syllabus: $L^p$ space, duality, weak convergence, Young, Holder, and Minkowski inequalities, point-set topology, topological space, dense sets, completeness, compactness, connectedness, path-connectedness, separability, Tychnoff theorem, Stone-Weierstrass Theorem, Arzela-Ascoli, Baire category theorem, open mapping theorem, closed graph theorem, uniform boudnedness principle, Hahn Banch theorem. 
\end{abstract}

\section{Normed Linear Spaces}

\begin{definition}[$\ell^p$ space]
	Let $(a_1, a_n, ...)$ be a sequence. Then, the $\ell^p$-space is: 
	\begin{align}
		\ell^p := \sets{(a_1, a_2, ...)}{\sum_{n=1}^\infty |a_n|^p < +\infty}
	\end{align}
\end{definition}

\begin{theorem}[Riesz-Fisher] 
	$L^p(X)$ is complete. 
\end{theorem}

\begin{definition}[$L^p$ space]
	Let $E$ be a measurable set and let $1 \leq p < \infty$. Then, $L^p(E)$ is the collection of measurable functions $f$ for which $|f|^p$ is Lebesgue integrable over $E$. 
\end{definition}

\begin{definition}[Equivalent Functions]
	Let $\mathcal{F}$ be the collection of all measurable extended real-valued functions on $E$ that are finite a.e. on $E$. Define two functions $f$ and $g$ to be equivalent, and write $f \sim g$ if $g(x) = f(x)$ a.e. on $E$.
\end{definition}

\begin{definition}[Essentially Bounded]
	We call a function $f \in \mathcal{F}$ to be \textbf{essentially bounded} if there exists some $M \geq 0$, called the \textbf{essential upper bound} for $f$, for which 
	\begin{align*}
		|f(x)| \leq M 	
	\end{align*}
	for almost every $x \in E$. $L^\infty(E)$ is the collection of equivalence classes $[f]$ for which $f$ is essentially bounded. 
\end{definition}

\begin{definition}[Norm]
	Let $X$ be a linear space. A real-valued functional $|| \cdot ||$ on $X$ is called a \textbf{norm} provided that for each $f$ and $g$ in $X$ and each real number $\alpha$, 
	\begin{enumerate}[noitemsep] 
		\item (The Triangle Inequality). 
		\begin{align*}
			|| f + g || \leq ||f|| + ||g|| 	
		\end{align*}
		\item (Positive Homogeneity). 	
		\begin{align*}
			|| \alpha f || = |\alpha | || f || 	
		\end{align*}
		\item (Non-Negativity). 
		\begin{align*}
			|| f || \geq 0 \text{ and } ||f|| = 0 \text{ if and only if } f = 0	
		\end{align*}
	\end{enumerate}
\end{definition}

\begin{definition}[Normed Linear Space]
	$X$ is said to be a \textbf{normed linear space} if $X$ is equipped with a norm. 
\end{definition}

\begin{definition}[Essential Supremum]
	Let $f \in L^\infty (E)$. $||f||_\infty$ is called the \textbf{essential supremum} and is defined as: 
	\begin{align*}
		||f ||_\infty := \sets{M}{M \text{ is an essential upper bound for } f}	
	\end{align*}
	\textbf{Theorem:} $|| \cdot ||_\infty$ is a norm on $L^\infty (E)$. 
\end{definition}

\section{The Inequalities of Young, Holder, and Minkowski}

\begin{definition}[p-norm]
	Let $E$ be a measurable set, $1 < p < \infty$, and let $f \in L^p(E)$. Then, define the \textbf{p-norm} to be: 
	\begin{align}
		||f ||_p := \left[	\int_E |f|^p		\right]^{\frac{1}{p}} 
	\end{align}
\end{definition}
	
\begin{definition}[Conjugate]
	The \textbf{conjugate} of a number $p \in ]1, \infty[$ is the number $q = p/(p-1)$, which is the unique number $q \in ]1, \infty[$ for which 
	\begin{align}
		\frac{1}{p} + \frac{1}{q} = 1
	\end{align}
\end{definition}
The conjugate of 1 is defined to be $\infty$ and the conjugate of $\infty$ is defined to be 1. 

\begin{definition}[Young's Inequality]
	For $1 < p < \infty$, $q$ the conjugate of $p$, and any two positive numbers $a$ and $b$, we have: 
	\begin{align}
			ab \leq \frac{a^p}{p} + \frac{b^q}{q}
	\end{align}
\end{definition}

\begin{theorem}[Hölder's Inequality]
	Let $E \subseteq \bb{R}$ be measurable, $1 \leq p < \infty$, and $q$ the conjugate of $p$. If $f$ belongs to $L^p(E)$, and $g$ belongs to $L^q(E)$, then their product $f \cdot g$ is integrable over $E$ and: 
	\begin{align}
		\int_E | f \cdot g | \leq || f ||_p \cdot ||g||_q. 
	\end{align}
	Moreover, if $f \neq 0$, then the function defined as: 
	\begin{align}
		f^* := ||f||_p^{1-p} \cdot \text{sgn}(f) \cdot |f|^{p-1}
	\end{align}
	belongs to $L^q(E)$, 
	\begin{align*}
		\int_E f \cdot f^* = ||f||_p \text{ and } ||f^*||_q = 1	
	\end{align*}
	We call $f^*$ defined as above to be called the \textbf{conjugate function} of $f$. 
\end{theorem}

\begin{theorem}[Minkowski's Inequality]
	Let $E$ be a measurable set and $1 \leq p \leq \infty$. If the functions $f$ and $g$ belong to $L^p(E)$, then so does their sum $f+g$. Moreover, 
	\begin{align}
		||f+g||_p \leq ||f||_p + ||g||_p 
	\end{align}
\end{theorem}

\begin{theorem}[Cauchy-Schwarz Inequality]
	Let $E$ be a measurable set and let $f$ and $g$ be measurable functions over $E$ for which $f^2$ and $g^2$ are integrable over $E$. Then, $f \cdot g$ is integrable over $E$ and 
	\begin{align}
		\int_E |f \cdot g| \leq \sqrt{\int_E f^2} \cdot  \sqrt{\int_E g^2}	
	\end{align}
\end{theorem}

\begin{corollary}
	Let $E$ be a measurable set and $1 < p < \infty$. Suppose $\mathcal{F}$ is a family of functions  in $L^p(E)$ that is bounded in $L^p(E)$ in the sense that there is a constant $M$ for which 
	\begin{align*}
		|| f ||_p \leq M \text{ for all } f \in \mathcal{F}	
	\end{align*}
	Then, the family $\mathcal{F}$ is uniformly integrable over $E$. 
\end{corollary}

\begin{corollary}
	Let $E$ be a measurable set of finite measure and $1 \leq p_1 < p_2 \leq \infty$. Then, $L^{p_2}(E) \subseteq L^{p_1} (E)$. Furthermore, 
	\begin{align*}
		||f ||_{p_1} \leq c ||f||_{p_2} 	
	\end{align*}
	for all $f$ in $L^{p_2}(E)$, where $ c = [m(E)]^{\frac{p_2 - p_1}{q_1 p_2}}$ if $p_2 < \infty$ and $c = [m(E)]^{\frac{1}{p_1}}$ if $p_2 = \infty$. 
\end{corollary}

\section{$L^p$ is complete: the Reisz-Fischer Theorem}

\begin{definition}[Converge]
	A sequence $\{ f_n \}$ in a linear space $X$ normed by $|| \cdot ||$ is said to \textbf{converge to $f$ in $X$} provided: 
	\begin{align*}
		\lim_{n \rightarrow \infty} || f - f_n || = 0 	
	\end{align*}
\end{definition}

\begin{definition}[Cauchy]
	A sequence $\{ f_n \}$ in a linear space $X$ that is normed by $|| \cdot ||$ is said to be \textbf{Cauchy} in $X$ provided for each $\varepsilon > 0$, there exists a $N \in \bb{N}$ such that 
	\begin{align}
		|| f_n - f_m || < \varepsilon\ \forall\ m, n \geq N
	\end{align}
\end{definition}

\begin{definition}[Complete]
	A normed linear space $X$ is called \textbf{complete} if every Cauchy sequence in $X$ converges to a function in $X$. A complete normed linear space is called a \textbf{Banach space}. 
\end{definition}


\begin{prop}
	Let $X$ be a normed linear space. Then, every convergent sequence in $X$ is Cauchy. Moreover, a Cauchy sequence in $X$ converges if it has a convergent subsequence. 
\end{prop}

\begin{definition}
	Let $X$ be a linear space normed by $|| \cdot ||$. A sequence $\{ f_n \}$ in $X$ is said to be \textbf{rapidly Cauchy} if there is a convergent series of positive numbers $\sum_{k=1}^\infty \varepsilon_k$ for which 
	\begin{align*}
		|| f_{k+1} - f_k || \leq \varepsilon_k^2 	\text{ for all $k$} 
	\end{align*}
	
\end{definition}

\begin{prop}
	Let $X$ be a normed linear space. Then, every rapidly Cauchy sequence in $X$ is Cauchy. Furthermore, every Cauchy sequence has a rapidly Cauchy subsequence. 
\end{prop}

\begin{prop}
	Let $E$ be a measurable set and $1 \leq p \leq \infty$. Then, every rapidly Cauchy sequence in $L^p(E)$ converges with respect to the $L^p(E)$ norm and pointwise a.e. on $E$ to a function in $L^p(E)$. 
\end{prop}

\begin{theorem}[Riesz-Fischer Theorem] 
	Let $E$ be a measurable set and $1 \leq p \leq \infty$. Then $L^p(E)$ is a Banach space. Moreover, if $\{f_n \} \rightarrow f$ in $L^p(E)$, a subsequence of $\{ f_n \}$ converges pointwise a.e. on $E$ to $f$. 
\end{theorem}

\begin{theorem}
	Let $E$ be a measurable set and $1 \leq p < \infty$. Suppose $\{ f_n \}$ is a sequence in $L^p(E)$ that converges pointwise a.e. on $E$ to the function $f$ which belongs to $L^p(E)$. Then: 
	\begin{align*} 
		\{ f_n \} \rightarrow f \text{ in } L^p(E) \iff \lim_{n \rightarrow \infty } \int_E |f_n | ^p = \int_E |f|^p 	
	\end{align*}
\end{theorem}

\begin{definition}[Tight]
	A family $\mathcal{F}$ of measurable functions on $E$ is said to be \textbf{tight} over $E$ provided that for each $\varepsilon > 0$, there exists a subset $E_0$ of $E$ of finite measure for which 
	\begin{align*}
		\int_{E \setminus E_0 } |f| < \varepsilon \text{ for all } f \in \mathcal{F}	
	\end{align*}

\end{definition}
\begin{theorem}
	Let $E$ be a measurable set and let $1 \leq p < \infty$. Suppose $\{ f_n \}$ is a sequence in $L^p(E)$ that converges pointwise a.e. on $E$ to the function $f$ which belongs to $L^p(E)$. Then, $\{ f_n \} \rightarrow f$ in $L^p(E)$ $\iff$ $\{ |f_n|^p \}$ is uniformly integrable and tight over $E$. 
\end{theorem}

\section{Approximation and Separability}

\begin{definition}[Dense]
	Let $X$ be a normed linear space with norm $|| \cdot ||$. Given two subsets $\mathcal{F}$ and $\mathcal{G}$ of $X$ with $\mathcal{F} \subseteq \mathcal{G}$, we say that $\mathcal{F}$ is \textbf{dense} in $\mathcal{G}$ provided for each function $g$ in $\mathcal{G}$ and $\varepsilon > 0$, there is a function $f \in \mathcal{F}$ for which $||f - g || < \varepsilon$. 
\end{definition}

\begin{prop}
	Let $E$ be a measurable set and let $1 \leq p \leq \infty$. Then, the subspace of simple functions in $L^p(E)$ is dense in $L^p(E)$. 
\end{prop}

\begin{prop}
	Let $[a,b]$ be a closed, bounded interval and $1 \leq p < \infty$. Then, the subspace of step functions on $[a,b]$ is dense in $L^p[a,b]$. 
\end{prop}

\begin{definition}[Separable] A normed linear space $X$ is said to be \textbf{separable} provided there is a countable subset that is dense in $X$. 
\end{definition}

\begin{theorem}
	Let $E$ be a measurable set and $1 \leq p < \infty$. Then, the normed linear space $L^p(E)$ is separable. 
\end{theorem}

\begin{theorem}
	Suppose $E$ is measurable and let $1 \leq p < \infty$. Then, $C_c(E)$ (the set of all continuous functions with compact support on $E$) is dense in $L^p(E)$. 
\end{theorem}

\end{document}